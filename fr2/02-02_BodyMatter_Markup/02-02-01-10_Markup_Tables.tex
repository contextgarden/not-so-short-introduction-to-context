\startcomponent 02-02-01-10_Markup_Tables

\environment introCTX_env_00

%==============================================================================

\startsection
  [title=Tableaux (tableaux naturels),
     reference=sec:tables,]

\TocChap


\tex{bTABLE}

\PlaceMacro{bTABLE}
\PlaceMacro{bTR}
\PlaceMacro{bTD}
\PlaceMacro{bTH}
\PlaceMacro{bTABLEhead}
\PlaceMacro{bTABLEnext}
\PlaceMacro{bTABLEbody}
\PlaceMacro{bTABLEfoot}
\PlaceMacro{setupTABLE}

Cette partie provient en grande partie des  \goto{\em Fiches à Bébert}[url(http://lesfichesabebert.fr/context/table.html)].

Les tableaux naturels offrent une configuration plus fine et une robustesse plus grande, parfois au prix d'une plus grande rigueur dans la définition.



% **** Subsubsection
\startsubsubsection[title={Décrire le contenu du tableau}]

\setupTABLE[r][2,3][color=darkred] Toutes les éléments d'un tableau naturel sont des environnements, ils commencent tous par un bTRUC (en fait beginTRUC) et s'achève par un eTRUC (endTRUC). Attention les majuscules sont essentielles. 

\placetable
  [here,force]
  [tbl:nattablecommands]
  {Commandes permettant de définir le contenu d'un tableau naturel}
{\bTABLE
\setupTABLE[frame=off] 
\setupTABLE[r] [first] [bottomframe=on]
\bTABLEhead
\bTR
\bTH Élément délimité \eTH
\bTH Commande début \eTH
\bTH Commande fin \eTH
\bTH Commentaire \eTH
\eTR
\eTABLEhead
\bTABLEbody
%----------------------
\bTR
\bTD Tableau\eTD 
\bTD \tex{bTABLE} \eTD 
\bTD \tex{eTABLE} \eTD 
\eTR
%----------------------
\bTR
\bTD Colonne \eTD 
\bTD \tex{bTR} \eTD 
\bTD \tex{eTR} \eTD 
\bTD Table Row\eTD 
\eTR
%----------------------
\bTR
\bTD Cellule\eTD 
\bTD \tex{bTD} \eTD 
\bTD \tex{eTD} \eTD 
\bTD Table Data\eTD 
\eTR
%----------------------
\bTR
\bTD Cellule d'en-tête\eTD 
\bTD \tex{bTH} \eTD 
\bTD \tex{eTH} \eTD 
\eTR
%----------------------
\bTR
\bTD En-tête de la table\eTD 
\bTD \tex{bTABLEhead} \eTD 
\bTD \tex{eTABLEhead} \eTD 
\eTR
%----------------------
\bTR
\bTD Second en-tête de la table\eTD
\bTD \tex{bTABLEnext} \eTD 
\bTD \tex{eTABLEnext} \eTD 
\bTD utilisé après un changement de page \eTD 
\eTR
%----------------------
\bTR
\bTD Corps de la table\eTD 
\bTD \tex{bTABLEbody} \eTD 
\bTD \tex{eTABLEbody} \eTD 
\eTR
%----------------------
\bTR
\bTD Fin de la table\eTD 
\bTD \tex{bTABLEfoot} \eTD 
\bTD \tex{eTABLEfoot} \eTD 
\eTR
\eTABLEbody
%----------------------
\eTABLE}

\placefigure [force,here,none]
  []
  {}
{\startDemoHN
\bTABLE
\bTABLEhead
\bTR \bTH Français     \eTH \bTH Chti         \eTH 
     \bTH Anglais      \eTH \bTH Italien      \eTH \eTR
\eTABLEhead
\bTABLEbody
\bTR \bTD Pantalon     \eTD \bTD Marrone      \eTD 
     \bTD Pants        \eTD \bTD Pantaloni    \eTD \eTR
\bTR \bTD Serpillière  \eTD \bTD Wassingue    \eTD 
     \bTD Swab         \eTD \bTD Strofinaccio \eTD \eTR
\bTR \bTD Boue         \eTD \bTD Berdoule     \eTD
     \bTD Mud          \eTD \bTD Fangos       \eTD \eTR
\eTABLEbody
\eTABLE 
\stopDemoHN}

\stopsubsubsection

\stopsection


%==============================================================================

\stopcomponent

%%% TeX-master: "../introCTX_fra.tex"
