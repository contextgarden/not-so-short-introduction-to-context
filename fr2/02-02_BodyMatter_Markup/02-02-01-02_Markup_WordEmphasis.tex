\startcomponent 02-02-01-02_Markup_WordEmphasis

\environment introCTX_env_00

%==============================================================================



\startsection
  [title=Emphase de mots,reference=sec:mkp:emphasew]

\TocChap

\tex{em}
\tex{highlight}
\tex{underbar}
\tex{overbar}
\tex{overstrike}


\PlaceMacro{underbar}\PlaceMacro{underbars}\PlaceMacro{overbar}
\PlaceMacro{overbars}\PlaceMacro{overstrike}\PlaceMacro{overstrikes}

Pour tracer des lignes sous, sur ou à travers du texte, les commandes suivantes sont utilisées~:
\placefigure [force,here,none] [] {}
{\startDemoVW
\setupinterlinespace[big]
\underbar{Ceci est un texte underbar} \\
\underbar{Ceci \underbar{est \underbar{un texte}} underbar} \\
\underbars{Ceci est un texte underbars} \\
\overbar{Ceci est un texte overbar} \\
\overbars{Ceci est un texte overbars} \\
\overstrike{Ceci est un texte overstrike} \\
\overstrikes{Ceci est un texte overstrikes}
\stopDemoVW}


Comme on peut le voir, il existe deux commandes pour chaque type de ligne (sous, sur ou à travers le texte). La version singulière de la commande trace une seule ligne sous, sur ou à travers tout le texte pris comme argument, tandis que la version plurielle de la commande ne trace la ligne que sur les mots, mais pas sur les espaces blancs.

Ces commandes ne sont pas compatibles entre elles, c'est-à-dire qu'on ne peut pas en appliquer deux au même texte. Si on essaie, c'est toujours la dernière qui l'emportera. En revanche, \tex{underbar} peut être imbriqué, soulignant ce qui a déjà été souligné.

\startSmallPrint

Le manuel de référence signale que \tex{underbar} désactive la césure des mots du texte qui constituent son argument. Il n'est pas clair pour moi si cette déclaration se réfère uniquement à \tex{underbar} ou aux six commandes que nous examinons.

\stopSmallPrint

Autre méthode de mise en valeur, nous définissons une commande spécifique avec \tex{definehighlight} auquel nous associons une couleur. Pour exemple~:

\PlaceMacro{definehighlight}

\placefigure [force,here,none] [] {}{
\startDemoVW
\definehighlight[important][color=red]
\important{Very important text}
\stopDemoVW}

Ainsi, pour modifer de façon cohérente l'ensemble des textes indiqués comme \quotation{important} dans le code source, il suffira de modifier la déclaration de \tex{definehighlight}.


\stopsection


%==============================================================================

\stopcomponent

%%% TeX-master: "../introCTX_fra.tex"
