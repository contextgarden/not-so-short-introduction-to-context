\startcomponent 02-04_SourceFile

\environment introCTX_env_00

\startchapter
  [reference=cap:sourcefile, title=Fichiers sources et projets]

\TocChap

Comme nous le savons déjà, lorsque nous travaillons avec \ConTeXt\, nous commençons toujours par un fichier texte dans lequel sont incluses, outre le contenu du texte, un certain nombre d'instructions indiquant à \ConTeXt\ les transformations qu'il doit appliquer pour générer notre document final correctement formaté en \PDF.

En pensant aux lecteurs qui, jusqu'à présent, n'ont su travailler qu'avec des traitements de texte, je pense qu'il vaut la peine de passer un peu de temps avec le fichier source lui-même. Ou plutôt les fichiers sources, car il y a des moments où il n'y a qu'un seul fichier source et d'autres où nous utilisons plusieurs fichiers sources pour arriver au document final. Dans ce dernier cas, nous pouvons parler de \quotation{projets multifichiers}.



% * 4.1 - Fichiers unique ou multiples

\startsection
  [title=Projet simple et projet multi-fichiers]


En \ConTeXt\, nous pouvons utiliser un seul fichier source qui inclut absolument tout le contenu de notre document final ainsi que tous les détails s'y rapportant, dans ce cas nous parlons de \quotation{projet simple}, ou, au contraire, nous pouvons utiliser un certain nombre de fichiers sources qui partagent le contenu de notre document final, et dans ce cas nous parlons de \quotation{projet multi-fichier}.

Les scénarios dans lesquels il est typique de travailler avec plus d'un fichier source sont les suivants~:

\startitemize

\item Si nous écrivons un document dans lequel plusieurs auteurs ont collaboré, chacun ayant sa propre partie différente des autres~; par exemple, si nous écrivons un livre hommage avec des contributions de différents auteurs, ou un numéro de magazine, etc.

\item Si nous sommes en train d'écrire un long document où chaque partie (chapitre) a une autonomie relative, de sorte que l'arrangement final de ceux-ci permet plusieurs possibilités et sera décidé à la fin. Cela se produit assez fréquemment pour de nombreux textes académiques (manuels, introductions, etc.) où l'ordre des chapitres peut varier.

\item Si nous rédigeons un certain nombre de documents connexes qui partagent certaines caractéristiques de style ou macro, il est alors intéressant de rassembler ces éléments dans des fichiers séparés, et ainsi utilisables directement dans d'autres projets. Ces fichiers constituent comme des modèles de composition de document.

\item Si, tout simplement, le document sur lequel nous travaillons est volumineux, de sorte que l'ordinateur ralentit soit lors de l'édition, soit lors de la compilation~; dans ce cas, le fait de répartir le matériel sur plusieurs fichiers sources accélérera considérablement la compilation de chacun.

\stopitemize

\stopsection

% * 4.2 - Structure de fichier source simple

\startsection
  [title=Structure du fichier source d'un projet simple,
  reference=sec:structure]

In simple projects developed in a single source file, the structure is very simple and revolves around the \MyKey{text} environment that must essentially appear in the same file. We differentiate between the following parts of this file:
% TRADUCTION ?
% TODO Garulfo j'ai l'impression d'une redite 

\startitemize

\item {\bf Le préambule du document}~: tout ce qui va de la première ligne du fichier jusqu'au début de l'environnement \MyKey{text} indiqué par la commande (\PlaceMacro{starttext}\PlaceMacro{stoptext}\tex{starttext}).
 
\item {\bf Le corps du document}~: il s'agit du contenu de l'environnement \MyKey{text}~; ou en d'autres termes, tout ce qui se trouve entre \tex{starttext} et \tex{stoptext}.

\stopitemize

\placefigure [here]  
             [img:ProyectoSimple]
             {\tfx Fichier source contenant un projet simple}
{\startDemoI
% Première ligne du document

% Zone Préambule:
% Contenant la configuration globales des commandes
% pour l'ensemble du document

\starttext   % Début du contenu à proprement parler

...
...          % Contenu du document
...

\stoptext    % Fin du contenu le reste sera ignoré
\stopDemoI}

Dans \in{figure}[img:ProyectoSimple] nous voyons un fichier source très simple. Absolument tout ce qui précède la commande \tex{starttext} (qui, dans l'image, se trouve à la ligne 5, en ne comptant que celles contenant du texte), constitue le préambule~; tout ce qui se trouve entre \tex{starttext} et \tex{stoptext} constitue le corps du document. Tout ce qui suit le stoptext sera ignoré.

Le préambule est utilisé pour inclure les commandes qui affectent le document dans son ensemble, celles qui déterminent sa configuration générale. Il n'est pas indispensable d'écrire une commande dans le préambule. S'il n'y en a pas, \ConTeXt\ adoptera une configuration par défaut qui n'est pas très développée mais qui pourrait faire l'affaire pour de nombreux documents. Dans un document bien planifié, le préambule contiendra toutes les commandes affectant le document dans son ensemble, comme les macros et les commandes personnalisées à utiliser dans le fichier source. Dans un préambule typique, cela pourrait inclure les éléments suivants~:



\startitemize[packed]


\item la langue principale du document (Voir \in{section}[sec:langdoc]).

\item le format du papier (\in{section}[sec:paperize]) et de la mise en page (\in{section}[sec:pagelayout]).

\item la police principale des documents (\in{section}[sec:mainfont]).

\item la personnalisation des commandes de section à utiliser (\in{section}[sec:setuphead]) et, le cas échéant, la définition de nouvelles commandes de section (\in{section}[sec:definehead]).

\item les en-têtes et les pieds de page (\in{section}[sec:headerfooter]).

\item les paramètres pour nos propres macros (\in{section}[sec:definingcommands]).              

\item etc.

\stopitemize


Le préambule est destiné à la configuration générale du document~; par conséquent, rien de ce qui concerne le contenu du document ou le texte à traiter ne doit y figurer. En théorie, tout texte traitable inclus dans le préambule sera ignoré, bien que parfois, s'il est présent, il provoquera une erreur de compilation. 

{\bf Le corps du document}, encadré entre les commandes \tex{starttext} et \tex{stoptext} comprend le contenu réel, c'est-à-dire le texte réel, ainsi que les commandes \ConTeXt\ qui s'applique à des parties spécifiques du texte contenu et non à l'ensemble du document.

\stopsection

% * 4.3 - Gestion multi-fichiers

% TODO Garulfo Pourquoi apprendre ici le TeX style?!
% Il n'est pas toujours nécessaire d'avoir une structure plus complexe ; par ailleurs avec TeXworks il est facile de définir un document maître et de travailler dans un document lié !
\startsection
  [title=Gestion multi-fichiers {\em à la \TeX}]

Afin de pouvoir travailler avec plus d'un fichier source, \TeX\ a inclus la primitive appelée \tex{input}, qui fonctionne également dans \ConTeXt, bien que ce dernier comprenne deux commandes spécifiques qui comme nous allons le voir, perfectionnent dans une certaine mesure le fonctionnement de \tex{input}.

% *** 4.5.1 input

\startsubsection
  [reference=input, title=La commande \tex{input}]
\PlaceMacro{input}

La commande \tex{input} insère le contenu du fichier qu'elle indique. Son format est le suivant~:

\placefigure [force,here,none] [] {}{
\startDemoI
\input NomFichier
\stopDemoI}

où {\em NomFichier} est le nom du fichier à insérer. Notez qu'il n'est pas nécessaire de placer le nom du fichier entre des accolades, bien que la commande ne génère pas d'erreur si cela est fait. En revanche, il ne doit jamais être placé entre crochets. Si l'extension du fichier est \quotation{\type{.tex}}, elle peut être omise.

Lorsque \ConTeXt\ compile un document et trouve une commande \tex{input}, il recherche le fichier indiqué et poursuit la compilation comme si ce fichier faisait partie du fichier qui l'a appelé. Lorsqu'il a terminé la compilation, il retourne au fichier d'origine et reprend là où il s'est arrêté~; le résultat pratique est donc que le contenu du fichier appelé au moyen de \tex{input} est inséré à l'endroit où il est appelé. Le fichier appelé par \tex{input} doit avoir un nom valide dans notre système d'exploitation et ne doit pas comporter d'espaces vides. \ConTeXt\ le cherchera dans le répertoire de travail, et s'il ne le trouve pas là, il le cherchera dans les répertoires inclus dans la variable de l'environnement TEXROOT. Si le fichier n'est finalement pas trouvé, il produira une erreur de compilation.

L'utilisation la plus courante de la commande \tex{input} est la suivante~: un fichier est écrit, appelons-le \MyKey{principal.tex}, et il sera utilisé comme conteneur pour appeler, par le biais de la commande \tex{input}, les différents fichiers qui composent notre projet. Ceci est illustré dans l'exemple suivant~:

\placefigure [force,here,none] [] {}{
\startDemoI
\input MaConfiguration   % Commandes de configuration générale 

\starttext

  \input PageDeTitre
  \input Preface

  \input Chap1
  \input Chap2
  ...
  \input Chap6

\stoptext
\stopDemoI}

Notez comment, pour la configuration générale du document, nous avons appelé le fichier \quotation{MaConfiguration.tex} qui, nous le supposons, contient les commandes globales que nous voulons appliquer. Ensuite, entre les commandes \tex{starttext} et \tex{stoptext} nous appelons les différents fichiers qui contiennent le contenu des différentes parties de notre document. Si, à un moment donné, pour accélérer le processus de compilation, nous souhaitons ne pas compiler certains fichiers, il suffit de mettre une marque de commentaire au début de la ligne appelant ce ou ces fichiers. Par exemple, si nous sommes en train d'écrire le deuxième chapitre et que nous voulons le compiler simplement pour vérifier qu'il ne contient pas d'erreurs, nous n'avons pas besoin de compiler le reste et pouvons donc écrire~:

\placefigure [force,here,none] [] {}{
\startDemoI
\input MaConfiguration   % Commandes de configuration générale 

\starttext

% \input PageDeTitre
% \input Preface

% \input Chap1
  \input Chap2
  ...
% \input Chap6

\stoptext
\stopDemoI}

et seul le chapitre 2 sera compilé. Notez comment, d'autre part, changer l'ordre des chapitres est aussi simple que de changer l'ordre des lignes qui les appellent.

\startSmallPrint

Lorsque nous excluons un fichier de la compilation d'un projet multi-fichier, nous gagnons en vitesse de traitement, mais parmi les conséquences, toutes les références que la partie en cours de compilation fait à d'autres parties non encore compilées ne fonctionneront plus. Voir \in{section}[sec:references].

\stopSmallPrint

Il est important de préciser que lorsque nous travaillons avec \tex{input}, seul le fichier principal, celui qui appelle tous les autres, doit inclure les commandes \tex{starttext} et \tex{stoptext}, car si les autres fichiers les incluent, il y aura une erreur. Cela signifie, d'autre part, que nous ne pouvons pas compiler directement les différents fichiers qui composent le projet, mais que nous devons nécessairement les compiler à partir du fichier principal, qui est celui qui abrite la structure de base du document.

\stopsubsection


% *** 4.5.2 readfile 

\startsubsection
  [title=\tex{ReadFile} et \tex{readfile}]
\PlaceMacro{ReadFile}\PlaceMacro{readfile}


Comme nous venons de le voir, si \ConTeXt\ ne trouve pas le fichier appelé avec \tex{input}, il génère une erreur. Dans le cas où nous voulons insérer un fichier uniquement s'il existe, mais en tenant compte de la possibilité qu'il n'existe pas, \ConTeXt\ offre une variante de la commande \tex{input}~: 

\placefigure [force,here,none] [] {}{
\startDemoI
\ReadFile{MonFichier}
\stopDemoI}

Cette commande est en tous points similaire à \tex{input}, à la seule exception que si le fichier à insérer est introuvable, elle poursuivra la compilation sans générer d'erreur. Elle diffère également de \tex{input} par sa syntaxe, puisque nous savons qu'avec \tex{input} il n'est pas nécessaire de mettre le nom du fichier à insérer entre accolades. Mais avec \tex{ReadFile}, c'est nécessaire. 

Si nous n'utilisons pas d'accolades, \ConTeXt\ pensera que le nom du fichier à rechercher est le même que le premier caractère qui suit la commande \tex{ReadFile}, suivi de l'extension \type{.tex}. Ainsi, par exemple, si nous écrivons 

\placefigure [force,here,none] [] {}{
\startDemoI
\ReadFile MonFichier
\stopDemoI}

\ConTeXt\ comprendra que le fichier à lire s'appelle \quotation{\type{M.tex}}, puisque le caractère qui suit immédiatement la commande \tex{ReadFile} (à l'exception des espaces vides qui sont, comme nous le savons, ignorés à la fin du nom d'une commande) est une \quotation{M}. Étant donné que \ConTeXt\ ne trouvera normalement pas un fichier appelé \quotation{\type{M.tex}}, et que \tex{ReadFile} ne génère pas d'erreur s'il ne trouve pas le fichier, \ConTeXt\ continuera la compilation après le \quote{M} dans \quotation{\type{MonFichier}}, et insérera le texte \quotation{\type{onFichier}}.

Une version plus raffinée de \tex{ReadFile} est \tex{readfile} dont le format est le suivant~:

\placefigure [force,here,none] [] {}{
\startDemoI
\readfile{MonFichier}{àInsérerSiExistant}{àInsérerSiNonExistant}
\stopDemoI}

Le premier argument est similaire à \tex{Readfile}~: le nom d'un fichier entre accolades. Le deuxième argument comprend le texte à écrire si le fichier existe, avant d'insérer le contenu du fichier. Le troisième argument comprend le texte à écrire si le fichier en question n'est pas trouvé. Cela signifie que, selon que le fichier saisi comme premier argument est trouvé ou non, le deuxième argument (si le fichier existe) ou le troisième (si le fichier n'existe pas) sera exécuté.

\stopsubsection

\stopsection

% * 4.4 - Projet context 


\startsection
  [title=Gestion multi-fichiers {\em à la \ConTeXt},
  reference=sec:srcprojects]

Le troisième mécanisme que propose \ConTeXt\ pour les projets multi-fichiers est plus complexe et plus complet~: il commence par faire une distinction entre différents fichiers, les fichiers de projet, les fichiers de produit, les fichiers de composants et les fichiers d'environnement. 

\placefigure
  [force,here,none]
  []
  {}
{\externalfigure[project–structure.pdf][width=\textwidth]}
 

Pour comprendre les relations et le fonctionnement de chacun de ces types de fichiers, je pense qu'il est préférable de les expliquer individuellement~:

% *** 4.6.1 environnements

\startsubsection
  [reference=environnements, title={Fichiers d'environnement}]
\PlaceMacro{startenvironment}\PlaceMacro{environnement}
\index{fichier+environnement}

Un fichier d'environnement est un fichier qui stocke les macros et les configurations d'un style spécifique destinées à être appliquées à plusieurs documents, qu'il s'agisse de documents totalement indépendants ou de parties d'un document complexe. Le fichier d'environnement peut donc inclure tout ce que nous écririons normalement avant \tex{starttext}, c'est-à-dire la configuration générale du document.

\startSmallPrint

J'ai conservé le terme \quotation{fichiers d'environnement} pour ces types de fichiers, afin de ne pas m'écarter de la terminologie officielle de \ConTeXt\, même si je pense qu'un meilleur terme serait probablement \quotation{fichiers de mise en forme} ou \quotation{fichiers de configuration générale}.

\stopSmallPrint

Comme tous les fichiers source de \ConTeXt, les fichiers d'environnement sont des fichiers texte, et supposent que l'extension sera \quotation{\type{.tex}}, bien que si nous le voulons, nous pouvons la changer, peut-être en \quotation{\type{.env}}. Cependant, cela n'est généralement pas fait dans \ConTeXt. Le plus souvent, les fichiers d'environnement sont identifiés en commençant ou en terminant leur nom par \quotation{env}. Par exemple~: \quotation{\type{env_livreA.tex}} ou \quotation{\type{mon-livreA.tex}}. L'intérieur d'un tel fichier d'environnement ressemblerait à ce qui suit~:

\placefigure [force,here,none] [] {}
{\startDemoI
\startenvironment env_livreA

  \mainlanguage[fr]

  \setupbodyfont
    [palatino,14pt]

  \setupwhitespace
    [big]

  ...

\stopenvironment
\stopDemoI}

ou pourrait également faire appel à d'autres fichiers environnements de façon a décomposer les différents éléments~:

\placefigure [force,here,none] [] {}
{\startDemoI
\startenvironment env_livreA

\environment    env_livreA-polices
\environment    env_livreA-couleurs
\environment    env_livreA-abbreviations
\environment    env_livreA-urls
\environment    env_livreA-macros

\stopenvironment
\stopDemoI}

En d'autres termes, les définitions et les commandes de configuration se trouvent dans \tex{startenvironment} et \tex{stopenvironment}. Immédiatement après \tex{startenvironment}, nous écrivons le nom par lequel nous voulons identifier l'environnement en question, puis nous incluons toutes les commandes dont nous aimerions que notre environnement soit composé.

\startSmallPrint

En ce qui concerne le nom de l'environnement, d'après mes tests, le nom que nous ajoutons immédiatement après \tex{startenvironment} est simplement indicatif, et si nous ne lui donnions pas de nom, alors rien de préjudiciable ne se passe.

\stopSmallPrint

Les fichiers d'environnement ont été conçus pour fonctionner avec des composants et des produits (expliqués dans la section suivante). C'est pourquoi un ou plusieurs environnements peuvent être appelés depuis un composant ou un produit à l'aide de la commande \tex{environment}. Mais cette commande fonctionne également si elle est utilisée dans la zone de configuration (préambule) de tout fichier source \ConTeXt, même s'il ne s'agit pas d'un fichier source destiné à être compilé en parties.

La commande \tex{environnement} peut être appelée en utilisant l'un des deux formats suivants~:

\placefigure [force,here,none] [] {}{
\startDemoI
\environment env_livreA
\environment[env_livreA]
\stopDemoI}

Dans les deux cas, l'effet de cette commande sera de charger le contenu du fichier pris en argument. Si ce fichier n'est pas trouvé, la compilation se poursuivra de manière normale sans générer d'erreur. Si l'extension du fichier est \quotation{\type{.tex}}, elle peut être omise.

\stopsubsection

% *** 4.6.2. component and products

\startsubsection
  [reference=components and products, title=Composants et produits]
\PlaceMacro{startcomponent}\PlaceMacro{startproduct}\PlaceMacro{product}
\index{structure+composant}
\index{structure+produit}
\index{fichier+composant}
\index{fichier+produit}

Dans le cas d'un livre dont chaque chapitre se trouve dans un fichier source différent, on dira que les chapitres sont des  {\em composants} et que le livre est le {\em produit}. Cela signifie que le composant est une partie autonome d'un produit, capable d'avoir son propre style et d'être compilé indépendamment. Chaque composant aura un fichier différent et, en outre, il y aura un fichier produit qui rassemblera tous les composants. Les commandes utilisées \tex{startcomponent} et \tex{startproduct} sont suivies d'un nom qui sert à se repérer mais qui n'a pas d'impact sur le fonctionnement de \ConTeXt. L'habitude consiste à indiquer le nom du fichier lui même.

Un fichier de composant typique \quotation{\tt cmp_chapitre-1.tex} serait le suivant

\placefigure [force,here,none] [] {}
{\startDemoI
\startcomponent cmp_chapitre-1

  \startchapter[title={Titre du chapitre 1}]
  ...

\stopcomponent
\stopDemoI}

Et un fichier produit \quotation{\tt prd_livre-A.tex} rassemblant les composants ressemblerait à ce qui suit~:

\placefigure [force,here,none] [] {}
{\startDemoI
\startproduct   prd_livre-A

  \environment  env_livreA

  \component    cpm_chapitre-1
  \component    cpm_chapitre-2
  \component    cpm_chapitre-3
  ...

\stopproduct
\stopDemoI}

Le nom du composant qui est appelé à partir d'un produit doit être le nom du fichier qui contient le composant en question. Toutefois, si l'extension de ce fichier est \quotation{\type{.tex}}, il peut être omis.

% TODO Garulfo added

 Pour les questions concernant les chemins de fichiers et répertoire voyez le paragraphe dédié  \at{page}[refchemins].

Notez que le contenu réel de notre document sera réparti entre les différents fichiers \quote{component} et que le fichier produit se limite à établir l'ordre des composants. D'autre part, les composants (individuels) et les produits peuvent être compilés directement. La compilation d'un produit génère un fichier PDF contenant tous les composants de ce produit. Si, par contre, l'un des composants est compilé individuellement, cela générera un fichier PDF contenant uniquement le composant compilé.

% TODO Garulfo
% avant la commande \tex{startcomponent} <== pas nécessaire, hans met derrière
% avant la commande \tex{startproduct}

Dans un fichier de composant, nous pouvons appeler un ou plusieurs fichiers d'environnement avec \tex{environment FichierEnvironment}. Nous pouvons faire de même dans le fichier produit. Plusieurs fichiers d'environnement peuvent être chargés simultanément. Nous pouvons, par exemple, avoir notre collection préférée de macros et les différents styles que nous appliquons à nos documents dans différents fichiers. Notez toutefois que lorsque nous utilisons deux ou plusieurs environnements, ceux-ci sont chargés dans l'ordre dans lequel ils sont appelés, de sorte que si la même commande de configuration a été incluse dans plusieurs environnements et qu'elle a des valeurs différentes, ce sont les valeurs du dernier environnement chargé qui s'appliquent. D'autre part, les fichiers d'environnement ne sont chargés qu'une seule fois, donc dans les exemples précédents où l'environnement est appelé à partir du fichier produit et de fichiers composants spécifiques, si nous compilons le produit, c'est le moment où les environnements sont chargés, et dans l'ordre indiqué~; quand un environnement est appelé à partir de l'un des composants, \ConTeXt\ vérifiera si cet environnement est déjà chargé, auquel cas il ne fera rien.

% TODO ADDED BY GARULFO Garulfo
Si le fichier composant ne fait référence à aucun fichier environnement (ou fichier projet comme nous le verrons juste après), sa compilation directe n'appliquera pas les environnements appelés par le fichier produit.

Au contraire, si le fichier composant fait référence à un fichier environnement spécifique, la compilation du produit les appliquera.

\stopsubsection

% *** 4.6.3. projets

\startsubsection
  [title=Projets \ConTeXt]
\PlaceMacro{startproject}\PlaceMacro{project}
\index{structure+projet}
\index{fichier+projet}

La distinction entre produits et composants est suffisante dans la plupart des cas. Néanmoins, \ConTeXt\ possède un niveau encore plus élevé où l'on peut regrouper un certain nombre de produits~: il s'agit du {\em projet}.

Un fichier projet typique se présenterait plus ou moins comme suit

\placefigure [force,here,none] [] {}
{\startDemoI
\startproject   prj_macollection

\environment    env_general

\product        prd_livre-A  % version française
\product        prd_livre-B  % version espagnole
\product        prd_livre-C  % version anglaise
...

\stopproject
\stopDemoI}

Un scénario dans lequel nous aurions besoin d'un projet serait, par exemple, lorsque nous devons éditer une collection de livres, tous avec les mêmes spécifications de format~; ou bien différentes traduction d'un même livre~; ou si nous éditons une revue~: la collection de livres, ou la revue en tant que telle, serait le projet~; chaque livre ou chaque numéro de revue serait un produit~; et chaque chapitre d'un livre ou chaque article d'un numéro de revue serait un composant.

Les projets, en revanche, ne sont pas destinés à être compilés directement. Considérons que, par définition, chaque produit appartenant au projet (chaque livre de la collection, ou chaque numéro de revue) doit être compilé séparément et générer son propre PDF. Par conséquent, la commande \tex{product} incluse dans le projet pour indiquer quels produits appartiennent au projet, ne fait en fait rien~: il s'agit simplement d'un rappel pour l'auteur.

Évidemment, certains pourraient demander pourquoi nous avons des projets s'ils ne peuvent pas être compilés~: la réponse est que le fichier de projet lie certains environnements au projet. C'est pourquoi, si nous incluons la commande \PlaceMacro{projet}\tex{projet NomProjet} dans un fichier de composant ou de produit, \ConTeXt\ lira le fichier de projet et chargera automatiquement les environnements qui lui sont liés. C'est pourquoi la commande \tex{environnement} dans les projets doit venir après \tex{startproject} (comme pour les composants et produits)

% TODO Garulfo : semble faux
% ; cependant, dans les produits et les composants, \tex{environnement} doit venir {\em avant}. \tex{startproduct} ou \tex{startcomponent}. 

Tout comme pour les commandes \tex{environnement} et \tex{composant}, la commande \tex{projet} nous permet d'indiquer le nom du projet entre crochets ou de ne pas utiliser de crochets du tout. Cela signifie que \tex{projet NomdeFichier} et \tex{Projet[NomdeFichier]} sont des commandes équivalentes.  {\bf Résumé des différentes manières de charger un environnement}  Il ressort de ce qui précède qu'un environnement peut être chargé par n'importe laquelle des
procédures suivantes~:


\startitemize[a, broad]

\item Pour un fichier composant, en insérant entre \tex{startcomponent} et \tex{starttext} soit la commande \tex{environnement FichierEnvironnement} soit la commande \tex{projet FichierProjet}.

\item Pour un fichier produit, en insérant entre \tex{startproduct} et le premier \tex{component} soit la commande \tex{environnement FichierEnvironnement} soit la commande \tex{projet FichierProjet}.

\stopitemize

\stopsubsection

% *** 4.6.4 aspects communs

\startsubsection
  [title={Aspects communs des environnements, composants, produits et projets},
  reference=sec:chemins]


\startdescription{Noms des environnements, des composants, des produits et des projets~:}

% TODO Garulfo : imporession d'une redite avec un peu plus haut, faire une ref

Nous avons déjà vu que, pour tous ces éléments, après la commande \tex{start} qui initie un environnement, un composant, un produit ou un projet particulier, son nom est saisi directement. Ce nom, en règle générale, doit coïncider avec le nom du fichier contenant l'environnement, le composant ou le produit car, par exemple, lorsque \ConTeXt\ est en train de compiler un produit et que, selon le fichier du produit, il doit charger un environnement ou un composant, nous n'avons aucun moyen de savoir quel est le fichier de cet environnement ou de ce composant, à moins que le fichier ait le même nom que l'élément à charger.

Dans le cas contraire, d'après mes tests, le nom écrit après \tex{startproduct}, \tex{startcomponent}, \tex{startproject} ou \tex{startenvironment} dans les fichiers produit et environnement est simplement indicatif. S'il est omis une erreur bloque la compilation, mais s'il ne correspond pas au nom du fichier, rien de grave ne se produit.

% TODO Garulfo : faux selon mes tests 
% Cependant, dans le cas des composants, il est important que le nom du
% composant corresponde au nom du fichier qui le contient.

\stopdescription

\startdescription[reference=refchemins]{Structure des répertoires et chemins des fichiers~:}

\index{fichiers+chemin}
\index{repertoire+chemin}
\index{structure+chemin}

Par défaut, \ConTeXt\ cherche les fichiers dans le répertoire de travail et dans le chemin indiqué par la variable TEXROOT. Cependant, lorsque nous utilisons les commandes \tex{project}, \tex{product}, \tex{component} ou \tex{environment}, il est supposé que le projet possède une structure de répertoires dans laquelle les éléments communs se trouvent dans le répertoire parent, et les éléments spécifiques dans un répertoire enfant. Ainsi, si le fichier indiqué dans le répertoire de travail est introuvable, il sera recherché dans son répertoire parent, et s'il n'est pas trouvé là non plus, dans le répertoire parent de ce répertoire, et ainsi de suite.

% ADDED TODO Garulfo

Il est parfois utile d'indiquer le chemin d'un fichier particuler, cela se fait naturellement par exemple~: 

\placefigure [force,here,none] [] {}
{\startDemoI
\component chapitres/cmp_chapitre-6
\stopDemoI}

\PlaceMacro{usepath}\PlaceMacro{setupexternalfigures}

Mais bien souvent cela se défini directement dans l'un des fichiers environnements par la commande \tex{usepath} qui permet d'indiquer la liste des répertoires dans lesquels \ConTeXt\ doit chercher les fichiers {\tt .tex}.

\placefigure [force,here,none] [] {}
{\startDemoI
\usepath[{chapitres,annexes}]  
\stopDemoI}

Une autre commande \tex{setupexternalfigures} permet d'indiquer le chemin des images (dont l'utilisation sera détaillée à la \in{section}[sec:extimage]), avec une syntaxe similaire indiquée à l'option {\tt directory}~:

\placefigure [force,here,none] [] {}
{\startDemoI
\setupexternalfigures[directory={../general_images,images}]}
\stopDemoI}


\stopdescription

\stopsubsection

\stopsection



%Pour rappel tous les éléments de composition dont nous allons parler maintenant
%ont vocation à être indiqués dans zone préambule du fichier source
%(voir \in{section}[sec:srcpreambule]) et mieux encore dans un fichier
%environnement (voir \in{section}[sec:srcenvfile]) qui sera lui-même
%appelé en préambule. Il est biensûr possible de les indiquer localement
%dans le corps de texte, pour un élément très spécifique, 
%mais ce n'est pas une bonne pratique.


% * END

\stopchapter

\stopcomponent

%%% Local Variables:
%%% mode: ConTeXt
%%% mode: auto-fill
%%% coding: utf-8-unix
%%% TeX-master: "../introCTX_fra.tex"
%%% End:
%%% vim:set filetype=context tw=75 : %%%
