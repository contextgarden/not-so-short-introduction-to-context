\startcomponent 02-04-02-01_Typeset_ToC

\environment introCTX_env_00

%==============================================================================

\startsection
  [title={Table des matières},
  reference=sec:typset:content,]


\TocChap

  [
    
    title=Table des matières,
  ]

% ** Subsection full auto sans titre

\startsubsection
  [
    reference=sec:placecontent,
    title=Table des matières automatique sans titre,
  ]



En principe, la table des matières contiendra absolument toutes les sections numérotées, bien que nous puissions limiter son niveau de profondeur avec la commande \PlaceMacro{setupcombinedlist} \tex{setupcombinedlist} (dont nous parlerons plus loin). Ainsi, par exemple :


\placefigure [force,here,none] [] {}{
\startDemoI
\setupcombinedlist[content][list={chapter,section}]
\stopDemoI}

limitera le contenu de la table des matières aux deux niveaux indiqués~: chapitre et section.

Une particularité de cette commande est qu'elle est sensible à son emplacement dans le fichier source. C'est très facile à expliquer avec quelques exemples, mais beaucoup plus difficile si nous voulons spécifier exactement comment la commande fonctionne et quelles rubriques sont incluses dans la Table des matières dans chaque cas. Commençons donc par les exemples~:

\startitemize


\item \tex{placecontent} placé au début du document, avant la première commande de section (partie, chapitre ou section, selon la situation) générera une table des matières complète.

\startSmallPrint

Je ne suis pas vraiment sûr que la table des matières générée par défaut soit {\em complète}, je crois qu'elle comprend suffisamment de niveaux de section pour être complète dans la plupart des cas ; mais je soupçonne qu'elle ne dépassera pas le huitième niveau de section. Quoi qu'il en soit, comme mentionné ci-dessus, nous pouvons ajuster le niveau de sectionnement que la TdM atteint avec

\placefigure [force,here,none] [] {}{
\startDemoI
\setupcombinedlist[content][list={chapitre, section, sous-section, ...}]
\stopDemoI}


\stopSmallPrint

\item En revanche, cette même commande située à l'intérieur d'une partie, d'un chapitre ou d'une section générera une TdM strictement limité au contenu de l'élément de section concerné, ou en d'autres termes des chapitres, des sections et d'autres niveaux inférieurs de découpage d'une partie spécifique, ou des sections (et d'autres niveaux) d'un chapitre spécifique, ou des sous-sections d'une section spécifique.

\stopitemize

% TODO Garulfo : il me semble que l'on doit pouvoir simplifier la formulation et éviter une certaine redondance.

En ce qui concerne l'explication technique et détaillée, afin de bien comprendre le fonctionnement par défaut de \tex{placecontent}, il est essentiel de se rappeler que les différentes sections sont, en fait, des {\em environnements} pour \ConTeXt\ Mark~IV qui commencent par \tex{start{\em TypedeSection}} et se terminent par \tex{stop{\em TypedeSection}} et peuvent être contenues dans d'autres commandes de section de niveau inférieur. Donc, en tenant compte de cela, nous pouvons dire que \tex{placecontent} génère par défaut une table des matières qui comprendra uniquement~:

\startitemize

\item les éléments qui appartiennent à l'{\em environnement} (niveau de section) où la commande est placée. Cela signifie que la commande, lorsqu'elle est placée dans un chapitre, ne fera pas apparaître les sections et sous-sections des autres chapitres.

\item les éléments qui ont un niveau de section inférieur au niveau correspondant au point où se trouve la commande. Cela signifie que si la commande se trouve dans un chapitre, seules les sections, les sous-sections et les autres niveaux inférieurs sont inclus ; Si la commande se trouve au niveau d'une section, seules les sous-sections, les sous-sous-sections et les autres niveaux inférieurs sont inclus.

\stopitemize

En outre, pour que la table des matières soit générée, il faut que \tex{placecontent} se trouve {\em avant} la première section du chapitre dans lequel il se trouve, ou avant la première sous-section de la section dans laquelle il se trouve, etc.

\startSmallPrint

Je ne suis pas sûr d'avoir été clair dans l'explication ci-dessus. Peut-être qu'avec un exemple un peu plus détaillé que les précédents, nous pourrons mieux comprendre ce que je veux dire : imaginons la structure suivante d'un document :

\vbox{ \startitemize[packed]

  \item Chapter 1

    \startitemize[packed]

    \item Section 1.1

    \item Section 1.2

      \startitemize[packed]

      \item Subsection 1.2.1

      \item Subsection 1.2.2

      \item Subsection 1.2.3

      \stopitemize

    \item Section 1.3

    \item Section 1.4

    \stopitemize

  \item Chapter 2

  \stopitemize}

Ainsi : \tex{placecontent} placée avant le chapitre 1 générera une table des matières complète, similaire à celle générée par \tex{completecontent} mais sans titre. Mais si la commande est placée à l'intérieur du chapitre 1 et avant la section 1.1, la table des matières sera uniquement celle du chapitre ; et si elle est placée au début de la section 1.2, la table des matières sera uniquement le contenu de cette section. Mais si la commande est placée, par exemple, entre les sections 1.1 et 1.2, elle sera ignorée. Elle sera également ignorée si elle est placée à la fin d'une section, ou à la fin du document.

\stopSmallPrint

\placefigure [force,here,none] [] {}{
\startDemoVW
\setuphead[chapter][page=no]

\startchapter[title=chapter 1]

 \startcolor[darkgreen]
 \placecontent                    %  <=====
 \stopcolor

 \startsection[title=Section 1.1]
 \stopsection
 \startsection[title=Section 1.2]

  \startcolor[darkred]
  \placecontent                  %  <=====
  \stopcolor

  \startsubsection[title=Sous-section 1.2.1]
  \stopsubsection
  \startsubsection[title=Sous-section 1.2.2]
  \stopsubsection
  \startsubsection[title=Sous-section 1.2.3]
  \stopsubsection

 \stopsection
 \startsection[title=Section 1.3]
 \stopsection
 \startsection[title=Section 1.4]
 \stopsection

\stopchapter

\startchapter[title=chapter 2]
\stopchapter
\stopDemoVW}

Tout ceci, bien sûr, ne concerne que le cas où la commande ne comporte pas d'options. En particulier, l'option {\tt criterium} modifiera ce comportement par défaut.

Parmi les options autorisées par \tex{placecontent} je n'en expliquerai que deux, les plus importantes pour la mise en place de la table des matières, et, de plus, les seules qui sont (partiellement) documentées dans le manuel de référence \ConTeXt. L'option {\tt criterium}, qui affecte le contenu de la table des matières par rapport à l'endroit du fichier source où se trouve la commande, et l'option {\tt alternative}, qui affecte la présentation générale de la table des matières à générer.

\stopsubsection

% ** Subsection option de fonctionnement (quels éléments intégrer)

\startsubsection
  [
    reference=sec:criteriumlist,
    title={Option de sélection des éléments intégrés à la TdM~: l'option {\tt criterium}},
  ]

Le fonctionnement par défaut de \tex{placecontent} vis-à-vis de la position de la commande dans le fichier source a été expliqué ci-dessus. En plus de la commande \tex{setupcombinedlist} vue à la \in{section}[sec:toc with unnumbered secs] qui permet de sélectionner les éléments à intégrer dans la TdM, l'option {\tt criterium} permet d'adapter le fonctionnement. Entre autres, elle peut prendre les valeurs suivantes~:

\startitemize

item {\tt all} : la TdM sera complète, quel que soit l'endroit du fichier source où se trouve la commande.

\item {\tt previous} : la table des matières n'inclura que les commandes de section (du niveau auquel nous nous trouvons) {\em précédent} la commande \tex{placecontent}. Cette option est destinée aux TdMs qui sont positionnées à en fin de document ou en fin de la section courante.

\item {\tt part, chapter, section, subsection...} : implique que la TdM doit être limitée au niveau de section indiqué.

\item {\tt component} : dans les projets multifichiers (voir \in{section} [sec-projects]), cette option génère uniquement la TdM correspondant au {\em composant} (component) où se trouve la commande \tex{placecontent} ou \tex{completecontent}.

\stopitemize

\stopsubsection

% ** Subsection layout

\startsubsection
  [
    reference=sec:alternativelist,
    title={Mise en page de la TdM~: l'option {\tt alternative}},
  ]

L'option {\tt alternative} contrôle la présentation générale de la table des matières. Ses principales valeurs peuvent être consultées dans le  \in{tableau}[tbl:contentalternatives].



\placetable
  [here]
  [tbl:contentalternatives]
  {\tfx Ways of formatting the table of contents}
{\switchtobodyfont[small]
\starttabulate[|c|l|l|]
\HL
\NC {\bf alternative}
\NC {\bf Entrées sélectionnnées}
\NC {\bf Remarques}
\NR
\HL
\NC a
\NC Numéro -- Titre -- Page
\NC Une ligne par entrée
\NR
\NC b
\NC Numéro -- Titre -- Espaces -- Page
\NC Une ligne par entrée
\NR
\NC c
\NC Numéro -- Titre -- Ligne de points -- Page
\NC Une ligne par entrée
\NR
\NC d
\NC Numéro -- Titre -- Page
\NC TdM continue, entrées bout-à-bout
\NR
\NC e
\NC Titre
\NC Encadré
\NR
\NC f
\NC Titre
\NC Justification à gauche
\NR
\NC\NC\NC Justification à droite ou centrée
\NR
\NC g
\NC Titre
\NC Justification centrée
\NR
\HL
\stoptabulate}

Les quatre premières alternatives (a, b, c, d) fournissent toutes les informations de chaque section (son numéro, son titre et le numéro de page où elle commence), et conviennent donc aussi bien aux documents papier qu'aux documents électroniques. Les trois dernières alternatives (e, f, g)  ne nous informent que sur le titre, elles ne conviennent donc qu'aux documents électroniques où il n'est pas nécessaire de connaître le numéro de page où commence une section, à condition que la table des matières contienne un hyperlien vers celle-ci, ce qui est le cas par défaut avec \ConTeXt.

Par ailleurs, je pense que pour apprécier réellement les différences entre les différentes alternatives, il est préférable que le lecteur génère un document test où il pourra les analyser en détail. Voici quelques illustrations~:

\placefigure [force,here,none] [] {}{
\startDemoVW
\setuphead[chapter][page=no]

\startcolor[darkgray]
\placecontent[alternative=a]       %  <=====
\stopcolor

\startcolor[darkred]
\placecontent[alternative=b]       %  <=====
\stopcolor

\startcolor[darkgreen]
\placecontent[alternative=c]       %  <=====
\stopcolor

\startcolor[darkblue]
\placecontent[alternative=d]       %  <=====
\stopcolor

\startcolor[darkcyan]
\placecontent[alternative=e]       %  <=====
\stopcolor

\startcolor[darkmagenta]
\placecontent[alternative=f]       %  <=====
\stopcolor

\startcolor[darkyellow]
\placecontent[alternative=g]       %  <=====
\stopcolor

\startchapter[title=Chapitre 1]
 \startsection[title=Section 1.1]
  \startsubsection[title=Sous-section 1.1.1]
  \stopsubsection
 \stopsection
\stopchapter

\stopDemoVW}


\stopsubsection

% ** Subsection format 


\startsubsection
  [
    reference=sec:setuplist,
    title=Format des entrées de la TdM,
  ]
  \PlaceMacro{setuplist}

Nous avons vu que l'option {\tt alternative} de \tex{placecontent} ou \tex{completecontent} nous permet de contrôler la {\em mise en page} générale de la table des matières, c'est-à-dire les informations qui seront affichées pour chaque rubrique, et la présence ou non de sauts de ligne séparant les différentes rubriques. Les ajustements finaux de chaque entrée de la table des matières sont effectués avec la commande \tex{setuplist} dont la syntaxe est la suivante :

\placefigure [force,here,none] [] {}{
\startDemoI
\setuplist[Élement][Configuration]
\stopDemoI}

où {\em Élement} fait référence à un type particulier de section. Il peut s'agir de {\tt part}, {\tt chapter}, {\tt section}, etc. On peut également configurer plusieurs éléments en même temps, en les séparant par des virgules. {\em Configuration} offre jusqu'à 54 possibilités, dont beaucoup, comme d'habitude, ne sont pas expressément documentées ; mais cela n'empêche pas celles qui sont documentées, ou celles qui ne sont pas suffisamment claires, de permettre un ajustement très complet de la TdM.  Je vais maintenant expliquer les options les plus importantes, en les regroupant selon leur utilité, mais avant de les aborder, rappelons qu'une entrée de la TdM, selon la valeur de l'option {\tt alternative}, peut comporter jusqu'à trois éléments différents : Le numéro de section, le titre de la section et le numéro de page. Les options de configuration nous permettent de configurer les différents composants de manière globale ou séparée :


\startitemize

\item {\bf Inclusion ou exclusion des différents composants} : Si nous avons choisi une alternative qui inclut, en plus du titre, le numéro de section et le numéro de page (alternatives \quote{a} \quote{b} \quote{c} ou \quote{d}), les options {\tt headnumber=no} ou {\tt pagenumber=no} indiquent pour le niveau spécifique que nous configurons,  que le numéro de section ({\tt headnumber}) ou le numéro de page ({\tt pagenumber}) ne doivent pas être affichés.

\item {\bf Couleur et style} : Nous savons déjà que l'entrée qui génère une section spécifique dans la TdM peut avoir (selon l'alternative) jusqu'à trois composants différents : le numéro de section, le titre et le numéro de page. Nous pouvons indiquer conjointement le style et la couleur des trois composants à l'aide des options {\tt style} et {\tt couleur}, ou le faire individuellement pour chaque composant au moyen des options
{\tt numberstyle}, {\tt textstyle} et {\tt pagestyle}  (pour le style) et
{\tt numbercolor}, {\tt textcolor} ou {\tt pagecolor} pour la couleur.


Pour contrôler l'apparence de chaque entrée, en plus du style lui-même, nous pouvons appliquer une commande à l'entrée entière ou à l'un de ses différents éléments. Pour cela, il existe les options {\tt command}, {\tt numbercommand}, {\tt pagecommand} et {\tt textcommand}. La commande indiquée ici peut être une commande standard \ConTeXt\ ou une commande de notre propre création. Le numéro de section, le texte du titre et le numéro de page seront passés comme arguments à l'option {\tt command}, tandis que le titre de la section sera passé comme argument à {\tt textcommand} et le numéro de page à {\tt pagecommand}. Ainsi, par exemple, la phrase suivante fera en sorte que les titres de section soient écrits en (fausses) petites majuscules :

\placefigure [force,here,none] [] {}{
\startDemoVW
\setuphead[chapter][page=no]

\setuplist
  [section] 
  [textcommand=\cap,
   pagecolor=darkgreen,
   numberstyle=bold]

\placecontent[alternative=c]       %  <=====

\startchapter[title=Chapitre 1]
 \startsection[title=Section 1.1]
  \startsubsection[title=Sous-section 1.1.1]
  \stopsubsection
 \stopsection
\stopchapter

\stopDemoVW}

\item {\bf Séparation des autres éléments de la TdM} : Les options {\tt before} et {\tt after} nous permettent d'indiquer les commandes qui seront exécutées avant ({\tt before}) et après ({\tt after}) la composition de l'entrée de la table des matières. Normalement, ces commandes sont utilisées pour définir l'espacement ou un élément de séparation (tel un trait horizontal) entre les entrées précédentes et suivantes.

\item {\bf Indentation d'un élément} : défini avec l'option {\tt margin} qui nous permet de définir l'espace d'indentation gauche des entrées du niveau que nous configurons.

\item {\bf Hyperliens intégrés à la TdM} : Par défaut, les entrées de l'index comprennent un hyperlien vers la page du document où commence la section en question. L'option {\tt interaction} permet de désactiver cette fonction ({\tt interaction=no}) ou de limiter la partie de l'entrée d'index où se trouvera l'hyperlien, qui peut être le numéro de section ({\tt interaction=number} ou {\tt interaction=sectionnumber}), le titre de la section ({\tt interaction=text} ou {\tt interaction=title}) ou le numéro de page ({\tt interaction=page} ou {\tt interaction=pagenumber}).

\item {\em Autres aspects}:

\startitemize

\item {\tt width et distance} : spécifient respectivement la largeur accordé à l'affichage du numéro et la distance de séparation entre l'espace d'affichage du numéro et le titre de la section. Il peut s'agir d'une dimension (ou du mot clé {\tt fit} pour {\tt width} qui définit la largeur exacte du numéro de section). 

\item {\tt numberalign} : indique l'alignement des éléments de numérotation ; il peut être {\tt left}, {\tt right}, {\tt middle}, {\tt flushright}, {\tt flushleft}, {\tt inner}, {\tt outer}

Voici un exemple qui combine ces trois dernières options~:


\placefigure [force,here,none] [] {}{
\startDemoVW
\setuphead[chapter][page=no]

\setuplist[chapter]   [width=15mm, margin=0mm,
                       distance=3mm,
                       numberalign=flushright]  
\setuplist[section]   [width=15mm, margin=0mm,
                       distance=6mm,
                       numberalign=flushright]  
\setuplist[subsection][width=15mm,  margin=0mm,
                       distance=9mm,
                       numberalign=flushright] 
\placecontent

\startchapter[title=Chapitre 1]
 \startsection[title=Section 1.1]
  \startsubsection[title=Sous-section 1.1.1]
  \stopsubsection
 \stopsection
\stopchapter
\stopDemoVW}

\item {\tt symbol} : permet de remplacer le numéro de section par un {\em symbole}. Trois valeurs possibles sont prises en charge : {\tt one}, {\tt two} et {\tt three}. La valeur {\tt none} de cette option supprime le numéro de section de la table des matières.

\stopitemize

\stopitemize

Parmi les multiples options de configuration de la TdM, aucune ne nous permet de contrôler directement l'espacement entre les lignes. Celui-ci sera, par défaut, celui qui s'applique à l'ensemble du document. Souvent, cependant, il est préférable que les lignes de la table des matières soient légèrement plus serrées que le reste du document. Pour ce faire, nous devons inclure la commande qui génère la table des matières (\tex{placecontent} ou \tex{completecontent}) dans un groupe où l'espacement interligne est défini différemment. Par exemple~:

\placefigure [force,here,none] [] {}{
\startDemoVW
\setuphead[chapter][page=no]
\start
  \startcolor[darkred]
  \setupinterlinespace[6mm]
  \placecontent
  \stopcolor
\stop

\start
  \startcolor[darkyellow]
  \setupinterlinespace[small]
  \placecontent
  \stopcolor
\stop

\startchapter[title=Chapitre 1]
 \startsection[title=Section 1.1]
  \startsubsection[title=Sous-section 1.1.1]
  \stopsubsection
 \stopsection
\stopchapter
\stopDemoVW}

\stopsubsection

% ** Subsection ajustements


\startsubsection
  [
    reference=sec:manual adjustments,
    title=Ajustements manuels,
  ]

Nous avons déjà expliqué les deux commandes fondamentales pour générer des tables des matières (\tex{placecontent} et \tex{completecontent}), ainsi que leurs options. Ces deux commandes permettent de générer automatiquement des tables des matières, construites à partir des sections numérotées existantes dans le document, ou dans le bloc ou le segment du document auquel la table des matières fait référence. Je vais maintenant expliquer certaines configurations que nous pouvons effectuer pour que le contenu de la table des matières ne soit pas aussi {\em automatique} et plus {\em personnalisé}. Cela implique~:


\startitemize

\item la possibilité d'inclure également certains titres de sections non numérotées dans la table des matières.

\item la possibilité d'intégrer dans la table des matières une entrée particulière qui ne correspond pas à une section numérotée.

\item la possibilité d'exclure une section numérotée particulière de la table des matières.

\item la possibilité que le titre d'une section particulière figurant dans la table des matières ne coïncide pas exactement avec le titre figurant dans le corps du document.

\stopitemize

% *** Subsubsection include not numbered

\startsubsubsection
  [
    reference=sec:toc with unnumbered secs,
    title=Inclure les sections non numérotées dans la TdM,
  ]

Le mécanisme par lequel \ConTeXt\ construit la TdM implique que toutes les sections numérotées sont automatiquement incluses, ce qui, comme je l'ai déjà dit (voir \in{section}[sec:title parts]) dépend des deux options ({\tt number} et {\tt incrementnumber}) que nous pouvons modifier avec \tex{setuphead} pour chaque type de section. Il a également été expliqué qu'un type de section où {\tt incrementnumber=yes} et {\tt number=no} serait une section numérotée {\em en interne} mais pas {\em en externe} (c'est à dire pas de façon visible).

Par conséquent, si nous voulons qu'un type de section non numéroté particulier -- par exemple, {\tt subject} -- soit inclus dans la table des matières, nous devons modifier la valeur de l'option {\tt incrementnumber} pour ce type de section, en lui attribuant la valeur {\tt yes}, puis inclure ce type de section parmi ceux qui doivent être affichés dans la table des matières, ce qui se fait, comme expliqué précédemment, avec \tex{setupcombinedlist}.

Nous pouvons ensuite, si nous le souhaitons, formater cette entrée en utilisant \tex{setuplist} de la même manière que les autres ; par exemple~:

\placefigure [force,here,none] [] {}{
\startDemoVW
\setuphead [chapter] [page=no]

\setuphead [subject]
           [incrementnumber=yes]
\setuplist [subject] [color=darkred]

\setupcombinedlist
   [content]
   [list={chapter, subject, section, subsection}]

\placecontent

\startchapter[title=Chapitre 1]
 \startsection[title=Section 1.1]
 \stopsection
 \startsubject[title=Sujet 1.2]
 \stopsubject
\stopchapter
\stopDemoVW}

{\bf Note:} La procédure qui vient d'être expliquée inclura toutes les occurrences dans notre document du type de section non numérotée concerné (dans notre exemple, les sections de type {\tt subject}). Si l'on ne souhaite inclure qu'une occurrence particulière de ce type de section dans la table des matières, il est préférable de le faire par la procédure expliquée ci-dessous.

\stopsubsubsection

% *** Subsubsection rajouts manuels

\startsubsubsection
  [
    reference=sec:manualtoc,
    title=Ajouter des entrées manuellement à la TdM,
  ]

On peut envoyer soit une entrée (simulant l'existence d'une section qui n'existe pas réellement), soit une commande à la table des matières, à partir de n'importe quel point du fichier source.
 
Pour envoyer une entrée qui simule l'existence d'une section qui n'existe pas réellement, utilisez la fonction \PlaceMacro{writetolist} \tex{writetolist} dont la syntaxe est :

\placefigure [force,here,none] [] {}{
\startDemoI
\writetolist[TypedeSection][Options]{Numéro}{Texte}
\stopDemoI}

dans laquelle~:

\startitemize

\item Le premier argument indique le niveau que doit avoir cette entrée de section dans la table des matières : {\tt chapter}, {\tt section}, {\tt sous-section}, etc.

\item Le deuxième argument, qui est facultatif, permet de configurer cette entrée d'une manière particulière. Si l'entrée envoyée manuellement est omise, elle sera formatée comme le sont toutes les entrées du niveau indiqué par le premier argument ; je dois cependant signaler que dans mes tests, je n'ai pas réussi à le faire fonctionner.

\startSmallPrint

Tant dans la liste officielle des commandes de \ConTeXt\ (voir \in{section}[sec:qrc-setup-fr]) que dans le wiki, on nous dit que cet argument permet les \Doubt mêmes valeurs que \tex{setuplist} qui est la commande qui nous permet de formater les différentes entrées de la COT. Mais, j'insiste, dans mes tests je n'ai pas réussi à modifier de quelque manière que ce soit l'apparence de l'entrée de la TdM envoyée manuellement.

\stopSmallPrint

\item Le troisième argument est censé refléter la numérotation que possède l'élément envoyé à la COT \Doubt, mais je n'ai pas non plus réussi à le faire fonctionner dans mes tests.

\item Le dernier argument comprend le texte à envoyer à la table des matières.

\stopitemize

Ceci est utile, par exemple, si nous voulons envoyer une section non numérotée particulière, mais seulement celle-ci à la table des matières. Dans \in{section}[sec:toc with unnumbered secs], il est expliqué comment obtenir qu'une catégorie entière de sections non numérotées soit envoyée à la table des matières ; mais si nous voulons seulement y envoyer une occurrence particulière d'un type de section, il est plus pratique d'utiliser la commande \tex{writetolist} localement. Ainsi, par exemple, si nous voulons que la section de notre document contenant la bibliographie ne soit pas une section numérotée, mais qu'elle soit tout de même incluse dans la table des matières, nous devons écrire~:

\placefigure [force,here,none] [] {}{
\startDemoVW
\setuphead [chapter] [page=no]

\placecontent

\startchapter[title=Chapitre 1]
 \startsection[title=Section 1.1]
 \stopsection
 \startsubject[title=Sujet]
 \writetolist[section]{53}{Sujet important}
 \stopsubject
\stopchapter
\stopDemoVW}

Voyez comment nous utilisons la version non numérotée de {\tt section}, qui est {\tt subject}, pour la section mais nous l'envoyons à l'index, manuellement, comme s'il s'agissait d'une section numérotée ({\tt section}).

Une autre commande destinée à influencer manuellement la table des matières est \PlaceMacro{writebetweenlist} \tex{writebetweenlist} qui est utilisée pour envoyer non pas une entrée elle-même, mais une {\em commande} à la table des matières, depuis un point particulier du document. Par exemple, si nous voulons inclure une ligne entre deux éléments de la TdM, nous pouvons écrire ce qui suit à n'importe quel endroit du document situé entre les deux sections concernées :

\placefigure [force,here,none] [] {}{
\startDemoVW
\setuphead [chapter] [page=no]

\placecontent

\startchapter[title=Chapitre 1]
 \startsection[title=Section 1.1]
 \stopsection
 \writebetweenlist
   [section]
   [location=here]
   {\hrule}
 \startsection[title=Section 1.2]
 \stopsection
\stopchapter
\stopDemoVW}



\stopsubsubsection

% *** Subsubsection exclude particulier

\startsubsubsection
  [title=Exclure de la TdM une section particulière appartenant à un type de section qui est inclus dans le TdM]

La table des matières est construite à partir de {\em types de section} établis, comme nous le savons déjà, par l'option {\tt list} de \tex{setupcombinedlist}. Par conséquent, si un certain {\em type de section} doit apparaître dans la table des matières, il n'y a aucun moyen d'en exclure une section particulière que nous ne voulons pas voir figurer dans la table des matières, pour quelque raison que ce soit.

Normalement, si l'on ne veut pas qu'une section apparaisse dans la table des matières, il faut utiliser son équivalent non numéroté, c'est-à-dire, par exemple, {\tt title} au lieu de {\tt chapter}, {\tt subject} au lieu de {\tt section}, etc. Ces sections ne sont pas envoyées à la TdM, et ne sont pas non plus numérotées. 

Cependant, si pour une raison quelconque, nous voulons qu'une certaine section soit numérotée mais n'apparaisse pas dans la table des matières, même si d'autres types de ce genre le font, nous pouvons utiliser une {\em astuce} qui consiste à créer un nouveau type de section qui est un clone de la section en question. Par exemple~:

\placefigure [force,here,none] [] {}{
\startDemoVW
\definehead[MySubsection][subsection]
\placecontent
\startsection [title={Première section}]
\stopsection
\startsubsection [title={Première subsection}]
\stopsubsection
\startMySubsection [title={Seconde subsection}]
\stopMySubsection
\startsubsection [title={Troisième subsection}]
\stopsubsection
\stopDemoVW}


Ainsi, lors de l'insertion d'une section de type {\tt MySubsection}, le compteur de sous-sections augmentera, puisque cette section est un {\em clone} des sous-sections, mais la TdM ne sera pas modifiée, puisque par défaut elle n'inclut pas les section de types {\tt MySubsection}.

\stopsubsubsection

% *** Subsubsection textes différents

\startsubsubsection
  [title=Textes du titre différents entre TdM et corps du document]

Si nous voulons un texte de titre d'une section particulière différent entre celui inclu dans la table des matières et celui affiché dans le corps du document, nous devons créer la section non pas avec la syntaxe traditionnelle (\type{SectionType{Titre}}) mais avec \tex{SectionType [Options]}, ou avec \tex{startSectionType [Options]}, et attribuer le texte que l'on souhaite voir écrit dans la Table des matières à l'option {\tt list} (voir \in{section}[sec:sectionyntax]).

\placefigure [force,here,none] [] {}{
\startDemoVW
\placecontent
\startsection
  [title={Une introduction approximative et légèrement répétitive à la réalité de l'évidence},
   list={Une introduction à la réalité de l'évidence}]
\stopsection
\stopDemoVW}

% TODO Garulfo \nolist is deprecated since 2015
% https://www.mail-archive.com/ntg-context@ntg.nl/msg80259.html

\stopsubsubsection

\stopsubsection


\stopsection

%==============================================================================

\stopcomponent

%%% TeX-master: "../introCTX_fra.tex"
