\startcomponent 02-02-02-01_Markup_ToC

\environment introCTX_env_00

%==============================================================================

\startsection
  [title={Table des matières},
  reference=sec:content,]

\TocChap

  
\tex{placecontent}
\tex{completecontent}

Une table des matières et un index sont des élements globaux d'un document. Presque tous les documents auront une table des matières, tandis que seuls certains documents auront un index. Dans de nombreuses langues (mais pas en anglais), la table des matières et l'index sont regroupés sous le terme général \quote{index}. Pour les lecteurs anglais, la table des matières se trouve normalement au début (d'un document, voire dans certains cas au début des chapitres), et l'index à la fin.

L'un et l'autre impliquent une application particulière du mécanisme des références internes dont l'explication est incluse dans \in{section}[sec:references].


% ** Subsection TOC gébé

\startsubsection
  [title=Vue d'ensemble de la table des matières]

Dans le chapitre précédent, nous avons examiné les commandes qui permettent d'établir la structure d'un document au fur et à mesure de sa rédaction. Cette section se concentre sur la table des matières et l'index, qui reflètent en quelque sorte la structure du document et offrent des outils efficaces pour y naviguer. La table des matières est très utile pour se faire une idée du document dans son ensemble (elle permet de contextualiser l'information) et pour rechercher l'endroit exact où se trouve un passage particulier. Les livres dont la structure est très complexe, avec de nombreuses sections et sous-sections de différents niveaux de profondeur, semblent nécessiter un autre type de table des matières, car une table des matières moins détaillée (peut-être avec seulement les deux ou trois premiers niveaux de sectionnement) est essentielle pour se faire une idée générale du contenu du document. Par contre, elle ne permet pas de localiser un passage particulier ce qui peut être l'objet d'une table des matières très détaillée. Cette dernière ayant pour faiblesse de facilement faire perdre la vue d'ensemble à son lecteur. C'est pourquoi, parfois, les livres dont la structure est particulièrement complexe comportent plusieurs tables des matières~: une table des matières peu détaillée en début d'ouvrage, indiquant les parties principales, une table des matières plus détaillée au début de chaque chapitre ainsi que, éventuellement, un index en fin de document.

Tous ces éléments peuvent être générés automatiquement par \ConTeXt\ avec une relative facilité. On peut~:

\startitemize

\item Générer une table des matières complète ou partielle à n'importe quel endroit du document.

\item Décider du contenu de l'une ou l'autre.

\item Configurer leur apparence jusqu'au moindre détail.

\item Inclure des hyperliens qui nous permettent de passer directement à la section en question.

\stopitemize

En fait, cette dernière fonctionnalité est incluse par défaut dans toutes les tables des matières, à condition que la fonction d'interactivité ait été activée dans le document. Voir, à ce sujet, \in{section}[sec:interactivity].


L'explication de ce mécanisme dans le manuel de référence de \ConTeXt\ est, à mon avis, quelque peu confuse, ce qui, je pense, est dû au fait que trop d'informations sont introduites en même temps. Le mécanisme de construction des tables des matières de \ConTeXt\ comporte de nombreux éléments, et il est difficile de les expliquer tous tout en restant clair. En revanche, l'explication dans \goto{wiki} [url(https://wiki.contextgarden.net/Table_of_Contents#Page_numbering_in_ToC)], 
se limite pratiquement à des exemples~: cela est très utile pour apprendre les {\em trucs} mais inadéquat -- je pense -- pour comprendre le mécanisme et comment il fonctionne. C'est pourquoi la stratégie que j'ai décidé d'utiliser pour expliquer les choses dans cette introduction commence par supposer quelque chose qui n'est pas strictement vrai (ou pas complètement vrai)~: qu'il existe quelque chose dans \ConTeXt\ appelé la {\em table des matières}. À partir de là, les commandes {\em normales} pour générer la table des matières sont expliquées, et quand ces commandes et leur configuration sont bien connues, je pense que c'est le moment d'introduire -- bien qu'à un niveau théorique plutôt que plus pratique -- les informations sur les pièces du mécanisme qui ont été omises jusqu'alors. La connaissance de ces {\em éléments supplémentaires} nous permet de créer des tables des matières beaucoup plus personnalisées que celles que nous pouvons appeler les {\em les éléments de base} crées avec les commandes expliquées jusqu'à ce point~; cependant, dans la plupart des cas, nous n'aurons pas besoin de le faire.

\stopsubsection

% ** Subsection full auto

\startsubsection
  [
    reference=sec:completecontent,
    title=Table des matières entièrement automatique avec un titre,
  ]

Les commandes de base pour générer automatiquement une table des matières (TdM ou ToC en anglais) à partir des sections numérotées d'un document ({\tt part}, {\tt chapter}, {\tt section}, etc.) sont \PlaceMacro{completecontent} \tex{completecontent} et \PlaceMacro{placecontent} \tex{placecontent}. La principale différence entre ces deux commandes est que la première ajoute un {\em titre} à la Table des matières~; pour ce faire, elle insère immédiatement avant la Table des matières un {\em chapitre non numéroté} dont le titre par défaut est \quotation{Table des matières}.

Par conséquent, \tex{completecontent}~:


\startitemize

\item Insère, à l'endroit où il se trouve, un nouveau chapitre non numéroté intitulé \quotation{Table des matières}.

\startSmallPrint

Nous rappelons que dans \ConTeXt\ la commande utilisée pour générer une section non numérotée au même niveau que les chapitres, est \tex{title}. (voir \in{section}[sec:sectiontypes]). Par conséquent, en réalité, \tex{completcontent} n'insère pas un {\em Chapitre} (\tex{chapter}) mais un {\em Title} (\tex{title}). Je ne l'ai pas dit parce que je pense qu'il peut être déroutant pour le lecteur d'utiliser les noms des commandes de section non numérotées ici, puisque le terme {\em Title} a également un sens plus large, et il est facile pour le lecteur de ne pas l'identifier avec le niveau concret de section auquel nous faisons référence.

\stopSmallPrint

\item Ce {\em chapitre} (en fait, \tex{title}) est formaté exactement de la même manière que le reste des chapitres non numérotés du document, ce qui inclut par défaut un saut de page.

\item La table des matières est affichée immédiatement après le titre.

\stopitemize

Initialement, la TdM générée est {\em complète}, comme nous pouvons le déduire du nom de la commande qui la génère (\tex{completecontent}). Mais d'une part, nous pouvons limiter le niveau de profondeur de la TdM comme expliqué dans \in{section}[sec:placecontent] et, d'autre part, comme cette commande est {\em sensible} à l'endroit où elle se trouve dans le fichier source (voir ce qui est dit plus loin à propos de \tex{placecontent}), si \tex{completecontent} ne se trouve pas au début du document, il est possible que la TdM générée ne soit pas complète~; et à certains endroits du fichier source, il est même possible que la commande soit apparemment ignorée. Si cela se produit, la solution consiste à invoquer la commande avec l'option \MyKey{criterium=all}. À propos de cette option, voir également \in{section}[sec:placecontent].

Pour modifier le titre par défaut attribué aux TdM, nous utilisons la commande \PlaceMacro{setupheadtext} \tex{setupheadtext} dont la syntaxe est~:

\placefigure [force,here,none] [] {}{
\startDemoI
\setupheadtext[Langage][Élément=Texte]
\stopDemoI}

où {\em Langage} est facultatif et fait référence à l'identificateur de langue utilisé par \ConTeXt\ (voir \in{section}[sec:langdoc]), et {\em Élément} fait référence à l'élément dont nous voulons modifier le nom (\MyKey{content} dans le cas de la table des matières) et {\em Texte} est le texte du titre que nous voulons donner à notre TdM. Par exemple

\placefigure [force,here,none] [] {}{
\startDemoI
\setupheadtext[fr][content={Carte de navigation}]
\stopDemoI}

fera en sorte que la TdM générée par \tex{completecontent} soit intitulée \quotation{Carte de navigation} au lieu de \quotation{Table des matières}.

De plus, \tex{completecontent} permet les mêmes options de configuration que \tex{placecontent}, pour l'explication desquelles je renvoie à la section suivante.

La commande générale d'insertion d'une TdM sans titre, générée automatiquement à partir des commandes de sectionnement du document, est \tex{placecontent}, dont la syntaxe est~:

\placefigure [force,here,none] [] {}{
\startDemoI
\placecontent [Options]
\stopDemoI}

\stopsubsection
 
\stopsection

%==============================================================================

\stopcomponent

%%% TeX-master: "../introCTX_fra.tex"
