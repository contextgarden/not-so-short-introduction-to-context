\environment env_mini_pages

\starttext

\startbuffer
Une {\bf grande idée} consisterait à utiliser un {\em logiciel} particulier permettant d'indiquer explicitement les spécifications de la {\WORD mise en page} que l'auteur ou le typographe souhaite voir appliquer.
\stopbuffer
\showframe
\startstandardmakeup
{\tfb
  \midaligned{Les Aventures trépidantes}
  \midaligned{de Pif}
  \blank[big]
  \midaligned{par}
  \midaligned{P. Paf et P. Pouf}}
\stopstandardmakeup

\startfrontmatter

  \title{Table des matières}
  \placecombinedlist[content]

  \startchapter[title=Avant-propos]
    \getbuffer
  \stopchapter
  
\stopfrontmatter

\startbodymatter

  \startchapter[title=Chapitre 1]
    \getbuffer
  \stopchapter

  \startchapter[title=Chapitre 2]
    \getbuffer

    \startsection[title=Première Section]
      \getbuffer
    \stopsection

    \startsection[title=Deuxième Section]
      \getbuffer
    \stopsection

  \stopchapter

\stopbodymatter

\startappendices

  \startchapter[title=Annexe 1]
    \getbuffer
  \stopchapter

  \startchapter[title=Annexe 2]
    \getbuffer
  \stopchapter
  
\stopappendices

\startbackmatter

  \startchapter[title=Synthèse]
    \getbuffer
  \stopchapter
  
\stopbackmatter

\stoptext
