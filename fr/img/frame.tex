\showframe

\startsetups [mylayout]

\def\nbdiv{15}                                         % découpage de la grille
                                                % le premier élément est laissé
                                  % pour la marge interne et le blanc supérieur      
\def\nbdiva{10}                         % nb d'élements pour la zone principale
\def\nbdivb{3}                                    % nb d'élements pour la marge

\def\nbdivc{12}                                         % texte vertical
\def\nbdivd{2}                                          % blanc inf vertical

\newdimen\marge   
\marge=6mm
\def\myfooterdistance{\marge}
\def\myheaderdistance{\marge}

\def\myheader{\dimexpr(((1\paperheight)/\nbdiv)-\marge)\relax}
\def\myfooter{\dimexpr(((\nbdivd\paperheight)/\nbdiv)-\marge)\relax}

\def\myleftmargin{\dimexpr(((1\paperwidth)/\nbdiv)/2-\marge)\relax}

\def\myleftmargin{\dimexpr((1\paperwidth)/\nbdiv)-\marge\relax}
\def\myrightmargin{\dimexpr((\nbdivb\paperwidth)/\nbdiv)-\marge\relax}

\setuplayout[
  %
  width={\dimexpr((\nbdiva\paperwidth)/\nbdiv)-\marge\relax},
  backspace={\dimexpr(1\paperwidth/\nbdiv)+(\marge/2)\relax},
  % topspace + header + headerdistance = 24,75mm
  %
  header=\myheader,
  headerdistance=\myheaderdistance,
  topspace={\dimexpr(1\paperheight/\nbdiv)+(\marge/2)-\myheader-\myheaderdistance\relax},
  % height - ( header + headerdistance + footer + footerdistance ) = 222.75mm
  %
  footer=\myfooter,
  footerdistance=\myfooterdistance,
  height={\dimexpr((\nbdivc\paperheight)/\nbdiv)-\marge+\myfooter+\myfooterdistance+\myheader+\myheaderdistance\relax},
  %
  topdistance=\marge,
  top=0pt,
  bottomdistance=\marge,
  bottom=0pt,
  %
  % edge and margin distance
  %
  margindistance=\marge,
  edgedistance=\marge,
  %
  % left
  leftmargin=\myleftmargin,
  leftedge=0pt,
  % rigth
  rightmargin=\myrightmargin,
  rightedge=0pt]

\stopsetups

%------------------------------------------------------------------------------

\setups{mylayout}                     % application du layout défini "mylayout"

\setuppagenumbering
  [
   alternative=doublesided,                             % document recto-verso
   location=]    % suppression affichage par défaut numéro page centré en-tête

\setupbackgrounds [page]
[background=color, backgroundcolor=white]


\starttext
~
\page
~
\stoptext
