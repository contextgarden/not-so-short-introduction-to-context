%%% File:     Installation.mkiv
%%% Author:       Joaquín Ataz-López
%%% Begun:      May 2020
%%% concluded: May 2020
%%% Contents:  I thought about whether or not to include this appendix. I was clear from the outset that if it were to be included it had to be as an appendix, because if I put it at the beginning, it would be more difficult to engage the reader from the start. I wasn't so keen on it either because I didn't want to explain aspects related to Windows or Apple that I don't really know, since  I haven't used the former for more than 15 years and the latter I have never used. I was also tempted to refer to Pablo Rodriguez's text (much more complete), but finally, the fact that now the "official" version has become lmtx, in a way, Pablo's information is out of date, so I decided to include the appendix, but in very summary form. As I am unable to compile this document with the current version of LMTX (it compiled well up to a version in early August) and I have been reading the ConTeXt distribution list for a couple of months, I see that it is relatively common for a new version of LMTX to include a bug that needs to be fixed... I decided to recommend Standalone. 
%%%
%%% Edited with: Emacs + AuTeX - And at times vim + context-plugin
%%%

\environment introCTX_env_00

\setuphead
  [section]
  [sectionsegments=3:3]

\setuphead
  [subsection]
  [sectionsegments=3:4]

\startcomponent c01_Installation.mkiv

\startchapter
  [
    reference=installation_suite,
    title={Installing, configuring and updating \ConTeXt},
  ]

\TeX's main distributions (TeX Live, teTeX, MikTeX, MacTeX, etc.) include a version of \ConTeXt. However, this is not the most updated version. In this appendix I will explain two procedures to install two different versions of \ConTeXt; the first includes both \ConTeXt\ Mark~II and Mark~IV and the second includes only \ConTeXt\ Mark~IV.

The installation procedure follows the same steps on any operating system; but the details change from one system to another. However, we can simplify things in such a way that in the following lines I will distinguish between two big groups of systems:

\startitemize

\item {\bf Unix-type systems:} This includes Unix itself, as well as GNU Linux, Mac OS, FreeBSD, OpenBSD or Solaris. The procedure is basically the same in all these systems; there are some very small differences that I will highlight in the appropriate place.

\item {\bf Windows systems}, that includes the different versions of that operating system: Windows 10 (the latest version, I think), Windows 8, Windows 7, Windows Vista, Windows XP, Windows NT, etc.
  
\stopitemize

\startSmallPrint

  {\bf Important note on the installation process on Microsoft Windows systems:}

  \dontleavehmode\ConTeXt, like all \TeX\ systems, is designed to work from a terminal; the programs and procedures for installation, too. In Windows this is also perfectly possible and should not create any major difficulty. The problem is that, on the one hand Windows users are not always used to doing this, and on the other, since Windows came into being in the {\em  illusion} (false) that everything in a computer system could be done  graphically, in general the versions of that operating system do not {\em advertise} too much about how to use the terminal. And then, it is common for each version of this system to change the name of the program that runs the terminal and how to open it. As far as I know, the Windows terminal emulation program has been given many names: \quotation{DOS window}, \quotation{Command Prompt}, \quotation{cmd}, etc. The location of this program in the Windows application menu also changes depending on the version of Windows in question.

  I stopped using Windows-based systems in 2004, so there is little I can do here to help the reader. He or she will have to figure out, on their own, how to open a terminal in their particular version of the operating system; which shouldn't be too difficult.

\stopSmallPrint

\startsection[title=Installing and configuring\\ \suite-]

The \ConTeXt\ distribution known as \quotation{Standalone}, also known as \quotation{\ConTeXt\ Suite}, is a complete and updated distribution of \ConTeXt, which downloads the necessary files from the Internet, does not take up too much disk space, is easy to update, and above all --- hence the name {\em  Standalone} --- is contained in a single directory which can be located anywhere we want on the hard disk. It would even be possible for a single computer to have several versions of \ConTeXt\, each in its own directory. This distribution includes the fonts, binary files and documentation needed to run \ConTeXt\ Mark~II (which implies the \TeX\ PdfLatex and XeTeX engines), and \ConTeXt\ Mark~IV (which implies the LuaTeX engine).

\startSmallPrint

For information about \TeX\ {\em engines}, see \in{section}[sec:engines]; and on \TeX\ engines in relation to \ConTeXt, as well as the versions known as Mark~II and Mark~IV, \in{section}[sec:historyctx].

\stopSmallPrint

The following explains how to install, run, update and restore \suite- on our system. The data and procedures provided here are a summary of the much more extensive information included in the \goto{\ConTeXt\ wiki}[url(https://wiki.contextgarden.net/ConTeXt_Standalone#Unix-like_platforms_.28Linux.2FMacOS_X.2FFreeBSD.2FSolaris.29)], to which I have added some additional detail drawn from a wikibook on \ConTeXt\ hosted on \goto{wikibooks}[url(https://fr.wikibooks.org/wiki/ConTeXt/Installation)]. If there is any problem with the installation, or if you want to extend any detail, you should directly consult any of these (though the latter is in French)

\startsubsection[title=Installation]

Installing \suite- means having an Internet connection, and implies the following steps:

\startitemize[n, packed]

\item Creating the directory in which \ConTeXt\ will be installed.

\item Downloading the installation {\em script} into this directory.

\item Running this {\em script} with the desired options.

\item Making some final adjustments.

\stopitemize

\subsubsubject{Step 1: creating the installation directory}

This, in fact, has nothing to do with \ConTeXt\ and we have to assume that every user will know how to do it. In Windows systems the normal way is to do it from the file manager. On Unix-type systems, it can be done from a file manager or from a terminal. It is important, however, to keep in mind that it is not recommended that the installation directory contains any blank space in your path. I personally also tend to shy away from using non-English directory names with things like accented vowels in them.

From now on I will assume that the installation directory is, in Unix-like systems, \MyKey{\lettertilde/context/} and in Windows, \MyKey{C:\backslash Programs\backslash context}.

\subsubsubject{Step 2: Download the installation {\em script} into the installation directory}

The installation {\em script} will differ according to the operating system you are installing on:

\startitemize

\item On Unix-like systems it can be downloaded, with a web browser, or, from a terminal with \MyKey{wget} or \MyKey{rsync}:

  \starttyping

    wget http://minimals.contextgarden.net/setup/first-setup.sh
    rsync rsync://minimals.contextgarden.net/setup/first-setup.sh
    
  \stoptyping

\item On Windows-type systems, as far as I know, there are no standard tools for downloading from the console. It has to be done with a web browser. The download address can be any of the following:

    \color [blue] {\goto {\ttx http://minimals.contextgarden.net/setup/context-setup-mswin.zip}[url(http://minimals.contextgarden.net/setup/context-setup-mswin.zip)]\\    \goto{\ttx http://minimals.contextgarden.net/setup/context-setup-win64.zip}[url(http://minimals.contextgarden.net/setup/context-setup-win64.zip)]}

  Once downloaded, in Windows you have to unzip the file,

\stopitemize

\subsubsubject{Step 3: Run the installation {\em script}}

The installation {\em script} must be run from the terminal. In Unix-type systems the name of the {\em script} is \MyKey{first-setup.sh} and can be run with {\tt bash} or {\tt sh}. In Windows-type systems the {\em script} is called \MyKey{first-setup.bat} and is run by simply typing its name in the system console or MS-DOS window from the installation directory.

The installation {\em script} allows for the following options:

%By default the installation {\em script} installs the latest beta version,
%not including the extension modules. But this depends on the installation
%options that we indicate when running the {\em script}. These are the
%following:\footnote{In the French Wikibook I mentioned at the beginning of
%the appendix, the \MyKey{--fonts=all} and \MyKey{--goodies=all} are also
%discussed. Contextgarden does not mention them, but including them in the
%command as well does not hurt.}

\startitemize

\item {\tt\bf --context}: this option determines which version of ConTeXt will be installed, whether the most recent development version (\quotation{--context=latest}) or the latest stable version (\quotation{--context=beta}). The default value is \quotation{beta}.

\item {\tt\bf --engine}: allows us to indicate whether we want to install Mark IV (\quotation{--engine=luatex}, the default value) or Mark II.

\item {\tt\bf --modules}: also install the ConTeXt extension modules that do not belong to the distribution as such, but that offer interesting additional utilities. To do this we need to indicate \quotation{--modules=all}.

\stopitemize

\startSmallPrint
With regard to the installation options, I believe that the information in the wiki is now obsolete. There it says that to install only Mark IV you need to explicitly indicate the "--engine=luatex" option and that the "--context=latest" option installs the latest stable version, not the development version. However, from halfway through 2020 the content of first-setup.sh changed, and taking a look inside it I found that to install the very latest version you need to expressly indicate "--context=latest", and that "--engine=luatex" is enabled by default.
\stopSmallPrint

The French Wikibook I mentioned at the beginning adds two other possible
options to the options I just mentioned (documented on the ConTEXt wiki):
\quotation{--fonts=all} and \quotation{goodies=all}. ConTeXtgarden doesn't
mention them, but including them in the installation command as well
doesn't hurt. Therefore I would advise you to run the installation script
with the following options (depending on whether we are on a Unix- or
Windows-type system):

\startitemize[packed]\tfx

\item Unix: \type{bash first-setup.sh --context=latest --modules=all --fonts=all --goodies=all}

\item Windows:  \type{first-setup.bat --context=latest --modules=all --fonts=all --goodies=all}

\stopitemize
This, depending on the speed of our Internet connection, may take some time, but not too much.

\startSmallPrint

\subsubsubject{Configuring a proxy}

The installation script uses rsync to obtain the necessary files. So, if you are behind a proxy server, you need to specify its details to rsync. The easiest way to set this is to set the variable RSYNC_PROXY in the terminal or in your startup {\em script} (.bashrc or the corresponding file for each shell). Replace the username, password, proxyhost and proxyport with the correct information. This is done, on Unix-type systems, with the \MyKey{export} command, and in Windows-type systems with the \MyKey{set} command. For example:

\type{export RSYNC_PROXY=username:password@proxyhost:proxyport}

Sometimes, when we are behind a firewall, port 873 may be closed for outgoing TCP connections. If port 22 is open for ssh connections, one trick that can be used is to connect to a computer somewhere outside the firewall and tunnel into port 873 (using the nc program).

\type{export RSYNC_CONNECT_PROG='ssh tunelhost nc %H 873'}

where the \quote{tunnelhost} is the machine outside the firewall we have access to. Of course, this machine must have nc and port 873 open for the outgoing TCP connection

\stopSmallPrint

After running \MyKey{first-setup} in the installation directory two new directories will appear called, \MyKey{bin} and \MyKey{tex} respectively.

\subsubsubject{Step 4: Final adjustments (Only on GNU Linux)}

In GNU Linux systems there are many directories where fonts can be installed. If we want \ConTeXt\ to use these fonts we must tell it where to find them. To do this we must add the following line to the \MyKey{tex/setuptex} file created after the installation:

{\tfx\type{export OSFONTDIR="~/.fonts:/usr/share/fonts:/usr/share/texmf/fonts/opentype/”}}

with which the environment variable OSFONTDIR is loaded with the three directories in which the fonts installed in the system are normally located

\startSmallPrint

  The /usr/share/texmf/fonts/ will only be there if there is some other installation of \TEX\ or other systems based on it in our operating system; in this case it should be included in the OSFONTDIR path so we can use the opentype fonts that such an installation may have included. If you have any commercial fonts that you want \ConTeXt\ to use, you have to make sure that the path to these is one of those included in OSFONTDIR, or otherwise, add the path to this variable. I have seen, for example, that some fonts are installed in  /usr/local/fonts instead of /usr/share/fonts.


\stopSmallPrint

Finally, it may be a good idea to have \ConTeXt\ generate a database with the necessary files for execution. This will be done by running the following three commands from a terminal:


\starttyping
  . ~/context/tex/setuptex
  context --generate
  context --make
\stoptyping

The first instruction is a point (dot). That's an abbreviation for bash's internal {\tt source} command. We can also, of course, run {\em source} if it's more convenient for us.


\stopsubsection

\startsubsection[title=Running \suite-]

\suite- has been designed to be able to coexist with other installations of \TeX\ systems, which is an advantage because it allows us to have several different versions installed on the same operating system; but in order to exploit this advantage it is essential that the environment variables needed to run \ConTeXt\ are not set permanently, because every time we start a terminal to run \MyKey{context} from it, we'll have to start by loading these environment variables into memory. They are contained in the \MyKey{tex/setuptex} (Unix) or \MyKey{tex/setuptex.bat} (Windows) file. This is done:


\startitemize

\item In Unix-type systems, after opening the terminal in which we want to use \MyKey{context},  by running either of the following two commands:

  \starttyping
    source ~/context/tex/setuptex
    . ~/context/tex/setuptex
  \stoptyping

  (assuming that the directory where the version of \quotation{{\tt context}} we want to use is \MyKey{\lettertilde/context}).

\item In Windows-type systems, by running the \type{tex\setuptex.bat} command from the installation directory in the terminal from which we will use \ConTeXt.


\stopitemize

If there is no other installation of \TeX\ or any of its derivatives in our system, we can avoid this by automating the execution of this order every time a terminal is opened:

\startitemize

\item On Unix-like systems this is done by including it in the file containing the general terminal startup {\em script} (usually \MyKey{.bashrc}).


  \startSmallPrint

The configuration file of a terminal depends on the {\em shell} program that the terminal uses by default. If this is {\tt bash} (which is the most used in GNU Linux systems), the file read at the beginning is {\tt .bashrc}. The {\tt sh} and {\tt ksh} {\em shells} use a file called {\tt .profile}, {\tt zsh} uses {\tt .zshenv}, and {\tt tcsh} or {\tt csh} read the {\tt .cshrc} file. Some specific implementation may change the   names of these files and so, for example, {\tt .bashrc} is sometimes called {\tt .bash_profile}.


    
  \stopSmallPrint

\item In Windows-type systems we can create a shortcut on the desktop that runs cmd.exe and then edit it, putting as a command to run when we double click on it:


  \starttyping
    C:\WINDOWS\System32\cmd.exe /k C:\Programs\context\tex\setuptex.bat
  \stoptyping

\stopitemize

Another possibility, if we do not wish to run this script each time we want to use ConTeXt, nor want to permanently set the environment variables necessary for it to be run, is to do it from the text editor itself, instead of running ConTeXt from a terminal. How you do this depends on the particular text editor you are using. The ConTeXt wiki provides information on how to set up various common editors: LEd, Notepad++, Scite, TeXnicCenter, TeXworks, vim and some others.

\stopsubsection

\startsubsection
  [title=Updating the version of \suite- or returning to an earlier version]

Mark~IV is still under development, so \suite- is often updated. To update our installation just repeat the process: we download a new version of \MyKey{first-setup.sh} and run it.

If, for whatever reason, we want to go back to a previous version of \suite-, just run \MyKey{first-setup} with the \MyKey{--context=date} option, where {\em date} is the date corresponding to the version we want to recover. Note that the date has to be introduced in the US  months-days-years format.

The complete list of \ConTeXt\ versions and associated dates can be found at \goto{this link}[url(https://minimals.contextgarden.net/current/context/)].

Finally, keep in mind that after reinstalling the system, whether it is to upgrade or to return to a previous version, on GNU Linux systems you will have to run step 4 of the installation again, which I have called \quotation{Final Adjustments}.

\stopsubsection

\stopsection

\startsection[title=Installing \cap{lmtx}]

If we only plan to use \ConTeXt\ Mark~IV, and we want to compile our projects not directly with LuaTeX but with LuaMetaTex, a simplified LuaTeX that uses less system resources and that can work on {\em less powerful} systems, instead of \suite-, we need to install LMTX which is the latest version of \ConTeXt. The name is an acronym of the name of the \TeX\ engine being used: LuaMetaTeX. This version was launched in 2019, and since approximately May 2020 it is the recommended default \ConTeXt\ distribution as suggested in \goto{\ConTeXt\ wiki}[url(wiki)].


\startSmallPrint

The current development of LMTX is intense, and the beta version can change several times a week. Some of its developments, moreover, temporarily pose certain incompatibilities with Mark~IV, and so, for example, while I am writing these lines, the latest version of LMTX (August 4, 2020) produces an error with the \tex{Caps} command. Therefore I would advise newcomers, for the moment, to work with \suite- instead.


\stopSmallPrint

\startsubsection[title=The installation itself]

The installation is as simple as:

\startitemize

\item {\bf Step 1:} Decide on the directory you want to install LMTX in, and, if necessary, create it. I will assume that the installation is done in a directory called \quotation{context} located in our user directory.


\item {\bf Step 2:} Download (to the installation directory) the zip file from the \goto{\ConTeXt\ wiki}[url(https://wiki.contextgarden.net/Installation)] that corresponds to your operating system and processor. It can be any of the following:


\page[preference]

  \startitemize[packed]

  \item GNU/Linux
    \startitemize
    \item X86 Processor
      \startitemize
      \item \goto{32 bit version}[url(http://lmtx.pragma-ade.nl/install-lmtx/context-linux.zip)].
      \item \goto{64 bit version}[url(http://lmtx.pragma-ade.nl/install-lmtx/context-linux-64.zip)].
      \stopitemize
    \item ARM Processor
      \startitemize
      \item \goto{32 bit version}[url(http://lmtx.pragma-ade.nl/install-lmtx/context-linux-armhf.zip)].
      \item \goto{64 bit version}[url(http://lmtx.pragma-ade.nl/install-lmtx/context-linux-aarch64.zip)].
      \stopitemize
    \stopitemize
  \item Microsoft Windows
    \startitemize
    \item \goto{32 bit version}[url(http://lmtx.pragma-ade.nl/install-lmtx/context-mswin.zip)]
    \item \goto{64 bit version}[url(http://lmtx.pragma-ade.nl/install-lmtx/context-win64.zip)]
    \stopitemize
  \item Mac OS, \goto{versión de 64 bits}[url(http://lmtx.pragma-ade.nl/install-lmtx/context-osx-64.zip)]
  \item FreeBSD
    \startitemize
    \item \goto{32 bit version}[url(http://lmtx.pragma-ade.nl/install-lmtx/context-freebsd.zip)].
    \item \goto{64 bit version}[url(http://lmtx.pragma-ade.nl/install-lmtx/context-freebsd-amd64.zip)].
    \stopitemize
  \item OpenBSD6.6
    \startitemize
    \item \goto{32 bit version}[url(http://lmtx.pragma-ade.nl/install-lmtx/context-openbsd6.6.zip)].
    \item \goto{64 bit version}[url(http://lmtx.pragma-ade.nl/install-lmtx/context-openbsd6.6-amd64.zip)].
    \stopitemize
  \item OpenBSD6.7
    \startitemize
    \item \goto{32 bit version}[url(http://lmtx.pragma-ade.nl/install-lmtx/context-openbsd6.7.zip)].
    \item \goto{64 bit version}[url(http://lmtx.pragma-ade.nl/install-lmtx/context-openbsd6.7-amd64.zip)].
    \stopitemize

  \stopitemize

  If you don't know whether your system is 32-bit or 64-bit, chances are -- unless your computer is very old -- it's 64-bit. If you don't know whether your processor is X86 or ARM, it's most likely X86.


\item {\bf Step 3:} Unzip,  the file downloaded in the previous step into the installation directory. A folder will be created called \MyKey{bin} and two files, one called \MyKey{installation.pdf}, that contains more detailed information about the installation, and a second file which is the actual installation program called \MyKey{install.sh} (in Unix-type systems) or \MyKey{install.bat} (in Windows systems).

\item {\bf Step 4:} Run the installation program (\MyKey{install.sh} or \MyKey{install.bat}). It needs an Internet connection as the installation program searches the web for the files it needs.

  \startitemize

  \item On Unix-type systems the installation program, located in the installation directory, is run from a terminal, either with {\tt bash}, or with {\tt sh}. It is not necessary to have administrator privileges, unless the installation directory is outside the user's \MyKey{home}  directory.

  \item In Windows-type systems, you must open a terminal, move to the installation directory, and from the terminal, run {\tt install.bat}. It is not necessary here either that the installation program is run as a system administrator, but it is recommended that this be done so that symbolic links of the files can be used, thus saving disk space.
    
  \stopitemize

\item {\bf Step 5} inform the system of the path to LMTX:

  In Windows systems, the installation program generates a file,  called \MyKey{setpath.bat} which updates all the configuration files necessary to let Windows know that you have installed LMTX in the system and where you has done so.  In GNU Linux systems, FreeBSD or Mac OS no {\em script} that automates the  task is generated, so we must incorporate the address for the \ConTeXt\ binaries in the system's PATH variable, which we would get by running in the terminal, from the installation:

{\ttx export PATH="{\em InstallationDir}/tex/texmf-{\em Platform}/bin:"\$PATH}

where {\tt {\em InstallationDir}} is the installation directory (for example, “/home/user/context”) and {\tt texmf-{\em Platform}} will vary according to the version of LMTX we have installed. For example, an installation on a 64 bit Linux system, {\tt texmf-{\em Platform}} will be “texmf-linux-64”.  Therefore we should run the following command from the terminal:

{\ttx export PATH="/home/user/context/texmf-linux-64/bin:"\$PATH}

This command will include LMTX in the system path, only as long as the terminal from which it has been run remains open. If we want this to be done automatically every time a terminal is opened, we must include this command in the configuration file of the {\em shell} program used by default in the system. The name of this file changes according to which {\em shell} program it is: bash, sh, zsh, ksh, tcsh, csh... On most Linux systems, which use bash, the file is called \MyKey{.bashrc} so we should run the following command from our home directory:

{\ttx echo 'export PATH="/home/user/context/texmf-linux-64/bin:"\$PATH >> .bashrc}

{\bf Important note:} By executing this step, we will disable the possibility of using other versions of \ConTeXt\ on our system, such as the one incorporated in TeX Live or \quotation{\ConTeXt\ Standalone}. If we want to make both versions compatible, it is preferable to use the procedure described in \in{section}[sec:alias].

\stopitemize

\stopsubsection

\startsubsection[title=Installing extension modules in LMTX]

\ConTeXt\ LMTX does not incorporate a procedure for installing or upgrading the \ConTeXt\ extension modules. However, in \ConTeXt\ wiki there is a {\em script} that allows you to install and update all the modules along with the rest of the installation.

To do this we need to copy the \goto{aforementioned {\em script}}[url(https://wiki.contextgarden.net/Modules#ConTeXt_LMTX)], paste it into a text file located in the main LMTX installation directory (the one containing {\tt install.sh} or {\tt install.bat}) and run it from a terminal. I have personally verified that this works on a GNU Linux system. I'm not sure if it will work on a Windows system, since I don't have any version of that operating system to check it with.

\stopsubsection

\startsubsection[title=Updating LMTX]

Updating LMTX is as simple as running the installation program again: it will check the installed files against those on the web server and update as necessary.

If the website from which the files are obtained has changed, we obviously need to also change this address in the installation {\em script}; although perhaps it is easier to download a new version of the installation files in the same directory and extract from it the new \MyKey{install.sh} or \MyKey{install.bat}; or, even easier, unzip the file with the installation program and reinstall without first needing to removing the old files.

\stopsubsection

\startsubsection
  [title={Creating a file that loads the variables into memory needed for LMTX (only GNU/Linux systems)}]

\suite- contains, as we already know, a (\MyKey{tex/setuptex}) file that loads into memory all the variables needed to run it, but LMTX does not include a similar file. We can, however, easily create it ourselves and store it, for example, as \MyKey{setuplmtx} in the \MyKey{tex} directory. The commands that this file could have would be:

\starttyping
export PATH="~/context/LMTX/tex/texmf-linux-64/bin:"\$PATH
echo "Adding ~/context/LMTX/tex/texmf-linux-64/bin to PATH"
export TEXROOT="~/context/LMTX/tex"
echo "Setting ~/context/LMTX/tex as TEXROOT"
export OSFONTDIR="~/.fonts:/usr/share/fonts:/usr/share/texmf/fonts/opentype/"
echo "Loading font directories into memory"
alias lmtx="~/context/LMTX/tex/texmf-linux-64/bin/context"
echo "Creating an alias to run lmtx"
\stoptyping%$

With this, besides loading into memory the paths and variables needed to run LMTX we would be enabling the \MyKey{lmtx} command as a synonym of \MyKey{context}. 

After creating this file, before being able to use LMTX,  where we intend to use it we should run the following in the terminal:

\type{source ~/context/LMTX/tex/setuplmtx}

all this assuming that LMTX is installed in \MyKey{~/context/LMTX} and that we have called this file
\MyKey{setuplmtx} and stored it in \MyKey{~/context/LMTX/tex}.

\startSmallPrint

  The above is what I do, to work with LMTX in the same way I used to work with \suite-. However, I do not exclude the possibility that in LMTX it is not necessary, for example, to load into memory the variable OSFONTDIR, since I am struck by the fact that \ConTeXt\ wiki says nothing about this.
  
\stopSmallPrint

\stopsubsection

\stopsection

\startsection
  [
    reference=sec:alias,
    title={Using several versions of \ConTeXt\ on the same system (only for Unix-type systems)}
  ]

The operating system utility called {\tt alias} allows us to associate different names with different versions of \ConTeXt. So we can use, for example, the version of \ConTeXt\ included in TeX Live and LMTX; or the {\em Standalone} version and LMTX.

For example, if we store the versions of LMTX downloaded in January and August 2020 in different directories,  we could write the following two instructions in \MyKey{.bashrc} (or equivalent file read by default when opening a terminal):

\starttyping
alias lmtx-01=”/home/user/context/202001/tex/texmf-linux-64/bin/context”
alias lmtx-08=”/home/user/context/202008/tex/texmf-linux-64/bin/context”
\stoptyping

These instructions will associate the names {\tt lmtx-01} with the version of LMTX installed in the \MyKey{context/202001} directory  and {\tt lmtx-08} with the version installed in \MyKey{context/202008}.

\stopsection

\stopchapter

\stopcomponent

%%% Local Variables:
%%% mode: ConTeXt
%%% mode: auto-fill
%%% coding: utf-8-unix
%%% TeX-master: "../introCTX_fra.tex"
%%% End:
