\environment ../introCTX_env.mkiv

\startcomponent b13_Floats.mkiv

\startchapter
  [title={Obrázky, tabulky a~další plovoucí objekty}]

\TocChap

Tato kapitola je především o~plovoucích předmětech (plovoucích). 
Ale v~návaznosti na tento koncept jej využívá k~vysvětlení dvou typů objektů, 
které nemusí být nutně plovoucí, ačkoli jsou často konfigurovány, jako by byly: 
externí obrázky a~tabulky. Při pohledu na obsah této kapitoly by si někdo mohl myslet, 
že je to všechno velmi neuspořádané: začíná to povídáním o~plovoucích objektech, 
pak pokračuje povídáním o~obrázcích a~tabulkách a~končí se znovu povídáním o~plovoucích objektech. 
Důvody této nepořádnosti jsou {\em pedagogické}: obrázky a~tabulky lze vysvětlit, 
aniž bychom příliš trvali na tom, že jsou normálně plovoucí; a~přesto, když je začneme zkoumat, 
hodně nám pomůže zjištění, že, překvapení, už víme o~dvou plovoucích objektech.

\startsection
  [
    reference=sec:floating objects,
    title={Co jsou plovoucí objekty a~co dělají?},
  ]

Pokud by dokument obsahoval pouze {\em normální} text, bylo by stránkování relativně snadné: 
znalost maximální výšky textové oblasti stránky stačí k~měření výšky různých odstavců, 
abyste věděli, kam vložit konce stránek. Problém je v~tom, že v~mnoha dokumentech jsou objekty, 
fragmenty nebo nedělitelné bloky textu, jako je obrázek, tabulka, vzorec, orámovaný odstavec atd.

Někdy mohou tyto objekty zabírat velkou část stránky, což zase představuje problém, 
že pokud je musíte vložit na určité místo v~dokumentu, nemusí se vejít na aktuální stránku 
a~musí být náhle přerušeny, takže velký prázdný prostor ve spodní části, takže dotyčný objekt a~text, 
který za ním následuje, se přesune na další stránku. Pravidla dobré sazby však naznačují, 
že kromě poslední stránky kapitoly by mělo být na každé stránce stejné množství textu.

Je proto vhodné vyhnout se výskytu velkých prázdných vertikálních mezer; 
a~{\em plovoucí} objekty jsou hlavním mechanismem, jak toho dosáhnout. 
\quotation{plovoucí objekt} je takový, který nemusí být umístěn v~přesném bodě dokumentu, 
ale může se kolem něj {\em pohybovat} nebo {\em vznášet}. Cílem je umožnit
\ConTeXt{}u\ rozhodnout 
o~nejlepším místě, z~hlediska stránkování, k~umístění takových objektů 
a~dokonce jim povoluji přesunout se na jinou stránku; 
ale vždy se snažte nevzdálit se příliš daleko od bodu zahrnutí do zdrojového souboru.

Proto neexistují žádné objekty, které by musely být plovoucí. 
Ale jsou předměty, které občas potřebují být plovoucí. Rozhodnutí je každopádně na autorovi 
nebo osobě odpovědné za sazbu, jde-li o~dvě různé osoby.

Možnost změny přesného umístění nedělitelného předmětu nepochybně velmi usnadňuje sazbu pěkně 
vyvážených stránek; ale problém, který s~tím souvisí, je ten, že jelikož v~době psaní originálu nevíme, 
kde přesně takový objekt skončí, je těžké na něj odkazovat. Pokud jsme tedy například 
do svého dokumentu právě vložil příkaz, který vkládá obrázek a~v~dalším odstavci jej chci popsat 
a~napsat o~něm něco jako: \quotation{Jak můžete vidět z~předchozího obrázku}, 
když postava {\em plave} mohl by být klidně umístěn {\em za} co jsem právě napsal 
a~výsledkem je nekonzistence: čtenář hledá obrázek {\em před} textem, 
který na něj odkazuje a~nemůže jej najít, protože plovoucí obrázek skončil za tímto odkazem.

Toto je opraveno {\em číslováním} plovoucích objektů (po jejich rozdělení do kategorií), 
takže namísto odkazování na obrázek jako \quotation{předchozí obrázek} nebo \quotation{další obrázek} 
budeme odkazovat na něj jako \quotation{obrázek 1.3}, 
protože můžeme použít vnitřní referenční mechanismus \ConTeXt{}u, abychom zajistili, 
že číslo obrázku bude vždy aktuální (viz \in{section}[sec:reference]). 
Číslování těchto druhů objektů na druhou stranu usnadňuje vytvoření jejich rejstříku (index tabulek, grafů, obrázků, rovnic atd.). 
Jak to udělat, viz (\in{section}[sec:lists]).

Mechanismus pro práci s~plovoucími objekty v~\ConTeXt{}u\ je poměrně sofistikovaný 
a~občas tak abstraktní, že nemusí být vhodný pro začátečníky. 
Proto v~této kapitole začnu vysvětlením pomocí dvou konkrétních případů: obrázků a~tabulek. 
Pak se to pokusím poněkud zobecnit, abychom pochopili, jak rozšířit mechanismus na další druhy objektů.

\stopsection

\startsection
  [title={Externí obrázky}]

Jak čtenář v~této fázi ví (protože to bylo vysvětleno v~\in{section}[sec:ctx]), 
\ConTeXt\ je dokonale integrován s~MetaPostem a~může generovat obrázky a~grafiku, 
které jsou {\em naprogramované} ve velké míře stejným způsobem, 
jako se programují transformace textu. Existuje také rozšiřující modul 
pro \ConTeXt\footnote{\ConTeXt\ rozšiřující moduly, které mu poskytují další nástroje, 
ale nejsou zahrnuty v~tomto úvodu.}, který mu umožňuje pracovat s~TiKZ.
\footnote{Toto je grafický programovací jazyk určený pro práci se systémy založenými na
\TeX{}u. 
Je to \quotation{rekurzivní zkratka} z~německé věty \quotation{TiKZ ist keinen Zeichenprogramm}, 
což v~překladu znamená: \quotation{TiKZ není kreslící program}. 
Rekurzivní zkratky jsou jakýmsi žertem programátorů. Když pomineme MetaPost (který neumím používat), 
domnívám se, že TiKZ je skvělý systém pro programování grafiky.} 
Ale takovými obrázky se v~tomto úvodu nezabýváme (to by si pravděpodobně vynutilo zdvojnásobení jeho délky). 
Mám zde na mysli použití externích obrázků, které jsou umístěny v~souboru na našem pevném disku 
nebo jsou staženy přímo z~internetu pomocí \ConTeXt{}u.

\startsubsection
  [title={Přímé vkládání obrázků}]
  \PlaceMacro{externalfigure}

K~přímému vložení obrázku (ne jako plovoucího objektu) použijeme příkaz \tex{externalfigure}, 
jehož syntaxe je

\type{\externalfigure [Název] [Konfigurace]}

kde

\startitemize

\item {\em Jméno} může být buď název souboru obsahujícího obrázek, 
nebo webová adresa obrázku nalezeného na internetu, nebo symbolický název, 
který jsme dříve přiřadili k~obrázku pomocí příkazu \PlaceMacro{useexternalfigure}\tex{useexternalfigure}, 
jehož formát je podobný formátu \tex{externalfigure}, i~když vyžaduje první argument se symbolickým názvem, 
který bude spojen s~daným obrázkem.

\item {\em Konfigurace} je volitelný argument, který nám umožňuje aplikovat určité transformace na obrázek předtím, 
než je vložen do našeho dokumentu. Tento argument blíže prozkoumáme v~\in{section}[sec:configimage].

\stopitemize

Povolené formáty obrázků jsou pdf, mps, jpg, png, jp2, jbig, jbig2, jb2, svg, eps, gif nebo tif. 
\ConTeXt\ umí přímo spravovat osm z~nich, zatímco zbytek (svg, eps, gif nebo tif) 
je třeba před otevřením převést pomocí externího nástroje, který se mění podle formátu, 
a~proto musí být nainstalován do systému tak, aby \ConTeXt\ mohl manipulovat s~těmito druhy souborů.

\startSmallPrint

Mezi formáty podporované \tex{externalfigure} jsou také některé video formáty. 
Konkrétně: QuickTime (přípona .mov), Flash Video (přípona .flv) a~MPeg~4 (přípona .mp4). 
Většina přehrávačů PDF však neví, jak zacházet se soubory PDF s~vloženým videem. 
Nemohu k~tomu moc říct, protože jsem žádné testy nedělal.

\stopSmallPrint

Příponu souboru není třeba uvádět: \ConTeXt\ vyhledá soubor se zadaným názvem a~jednou z~přípon 
známých formátů obrázků. Je-li více kandidátů, použije se nejprve formát PDF, pokud existuje 
a~v~případě jeho absence formát MPS (grafika generovaná MetaPostem). Pokud tyto dva neexistují, 
postupuje se v~následujícím pořadí: jpeg, png, jpeg~2000, jbig a~jbig2.

\startSmallPrint

Pokud skutečný formát obrázku neodpovídá příponě souboru, ve kterém je uložen, 
\ConTeXt\ jej nemůže otevřít, pokud neoznačíme aktuální formát obrázku pomocí volby {\tt method}.

\stopSmallPrint

Pokud obrázek není umístěn samostatně mimo odstavec, ale je integrován do textového odstavce 
a~jeho výška je větší než řádkování, řádek se upraví tak, aby se předešlo překrývání obrázku 
s~předchozími řádky, jako příklad, který doprovází tento řádek\externalfigure[kráva-hnědá][šířka=2.5em].

\ConTeXt\ standardně vyhledává obrázky v~pracovním adresáři, ve svém nadřazeném adresáři 
a~v~nadřazeném adresáři tohoto adresáře. Umístění adresáře obsahujícího obrázky, 
se kterými budeme pracovat, můžeme označit pomocí volby {\tt directory} příkazu \tex{setupexternalfigures}, 
která přidá tento adresář do vyhledávací cesty. Pokud chceme, aby vyhledávání probíhalo pouze 
v~adresáři s~obrázky, musíme nastavit i~volbu {\tt location}. Takže například, 
aby náš dokument hledal všechny obrázky, které potřebujeme v~adresáři \MyKey{img}, 
měli bychom napsat:

\starttyping
  \setupexternalfigures
    [directory=img, location=global]
\stoptyping

\startSmallPrint

Ve volbě {\tt directory} v~\tex{setupexternalfigures} můžeme zahrnout více než jeden adresář 
a~oddělit je čárkami; ale v~tomto případě musíme adresáře uzavřít do složených závorek. 
Například \MyKey{directory=\{img, \lettertilde/imágenes\}}.

V~{\tt adresáři} vždy používáme znak \quote{/} jako oddělovač mezi adresáři; 
včetně systému Microsoft Windows, jehož operační systém používá jako oddělovač adresářů
\quote{\textbackslash}.

\stopSmallPrint

\tex{externalfigure} je také schopen používat obrázky hostované na internetu. 
Takže například následující úryvek vloží logo CervanTeX přímo z~internetu do dokumentu. 
Toto je \TeX\ španělsky mluvící uživatelská skupina:
\footnote{Internetové adresy jsou velmi dlouhé a~pro zobrazení příkladu se dvěma sloupci není 
k~dispozici mnoho místa. Aby tedy pořadí v~levém sloupci správně seděla, 
vložil jsem do webové adresy zalomení řádku. Pokud někdo chce příklad zkopírovat a~vložit, 
nebude to fungovat, pokud se tento konec řádku neodstraní.}

\startDoubleExample
\starttyping
\externalfigure
[http://www.cervantex.es/files/
cervantex/cervanTeXcolor-small.jpg]
\stoptyping

\externalfigure
[http://www.cervantex.es/files/cervantex/cervanTeXcolor-small.jpg]

\stopDoubleExample

\startSmallPrint

Když je dokument obsahující vzdálený soubor poprvé zkompilován, je stažen ze serveru 
a~uložen do adresáře mezipaměti LuaTeXu. Tento soubor v~mezipaměti se používá při následných kompilacích. 
Normálně se vzdálený obrázek stáhne znovu, pokud je obrázek v~mezipaměti starší než 1 den. 
Chcete-li změnit tento práh, podívejte se na \goto{\ConTeXt\ wiki}[url(https://wiki.contextgarden.net/Using_Graphics)].

\stopSmallPrint

Pokud \ConTeXt\ nenajde obrázek, který by měl být vložen, nevygeneruje se žádná chyba, 
ale místo obrázku bude vložen textový blok s~informacemi o~obrázku, který tam má být. 
Velikost tohoto bloku bude velikost obrázku (pokud ji \ConTeXt\ zná) nebo jinak standardní velikost. 
Příklad toho je v~\in{section}[sec:startcombination].

\stopsubsection

\startsubsection
  [
    reference=sec:placefigure,
    title={ Vložení obrázku pomocí \tex{placefigure}},
  ]
  \PlaceMacro{placefigure}

Obrázky lze vkládat přímo. Ale je lepší to udělat pomocí \tex{placefigure}. Tento příkaz způsobí
v~\ConTeXt{}u: 

\startitemize

\item že ví, že se vkládá obrázek, který musí být začleněn do seznamu obrázků v~dokumentu, 
který pak lze použít, pokud si přejeme, k~vytvoření indexu obrázků.

\item přiřazení čísla obrázku, a~tím  usnadnění interní odkazy na něj.

\item přidání názvu k~obrázku a~vytvoření textového bloku mezi obrázkem a~jeho názvem, 
což znamená, že je nelze oddělit.

\item automatické nastavení bílého prostoru (horizontálního a~vertikálního) potřebného 
pro správné zobrazení obrázku.

\item že umístí obrázek na vyznačené místo a~text kolem něj v~případě potřeby obtéká.

\item že převede obrázek na plovoucí objekt, pokud je to možné, 
s~přihlédnutím k~jeho velikosti a~specifikacím umístění.
\footnote{Toto je můj závěr, vzhledem k~tomu, že mezi možnostmi umístění jsou takové, 
jako je {\tt force} nebo  \Conjecture {\tt split}, které odporují skutečné představě o~plovoucím objektu.}

\stopitemize
  
\stopfigure

Syntaxe tohoto příkazu je následující:

\type{\placefigure[Možnosti] [Štítek] {Název} {Obrázek}}

Různé argumenty mají následující význam: 

\startitemize

\item {\em Options} je sada indikací, které obecně odkazují na umístění obrázku. 
Protože jsou tyto možnosti v~tomto a~dalších příkazech stejné, 
vysvětlím je společně později (v~\in{section}[sec:placeingobjects]). 
Prozatím budu pro příklady používat volbu {\tt here}. Říká \ConTeXt{}u\, že pokud je to možné, 
má umístit obrázek přesně do bodu v~dokumentu, kde se nachází příkaz, který jej vkládá.

\item {\em Label} je textový řetězec, který interně odkazuje na tento objekt, 
abychom na něj mohli odkazovat (viz \in{section}[sec:reference]).

\item {\em Title} je text titulku, který má být přidán k~obrázku.

\item {\em Image} je příkaz, který vloží obrázek.

\stopitemize

Například

\starttyping
  \placefigure
    [here]
    [fig:texknuth]
    {\TeX\ logo and photo of {\sc Knuth}}
    {\externalfigure[https://i.ytimg.com/vi/8c5Rrfabr9w/maxresdefault.jpg]}
\stoptyping

  \placefigure
    [here]
    [fig:texknuth]
    {\TeX\ logo and photo of {\sc Knuth}}
    {
      \externalfigure
        [https://i.ytimg.com/vi/8c5Rrfabr9w/maxresdefault.jpg]
        [scale=600]
    }

Jak můžeme vidět na příkladu, vložením obrázku (které bylo mimochodem provedeno přímo z~obrázku 
hostovaného na internetu) došlo k~určitým změnám ohledně toho, co se stane při přímém použití příkazu 
\tex{externalfigure}. Přidá se svislý prostor, obrázek se vycentruje a~přidá se název. 
To jsou {\em externí} změny zřejmé na první pohled. Z~vnitřního hlediska má příkaz také další 
neméně důležité efekty:

\startitemize

\item Nejprve byl obrázek vložen do \quotation{seznamu obrázků}, který \ConTeXt\ interně udržuje 
pro objekty vložené do dokumentu. To zase znamená, že se obrázek objeví v~indexu obrázků, 
který lze vygenerovat pomocí \tex{placelist[figure]} (viz \in{section}[sec:lists]), 
ačkoli existují dva specifické příkazy pro generování indexu obrázků, které jsou 
  \PlaceMacro{placelistoffigures}\tex{placelistoffigures} nebo
  \PlaceMacro{completelistoffigures}\tex{completelistoffigures}.

\item Za druhé, obrázek byl propojen se štítkem, který byl přidán jako druhý argument do příkazu 
\tex{placefigure}, což znamená, že od této chvíle na něj můžeme pomocí tohoto štítku vytvářet 
interní odkazy (viz \in{section}[sec:reference]).

\item Konečně se obrázek stal plovoucím, což znamená, že pokud by se pro potřeby sazby (stránkování) 
potřeboval přesunout, \ConTeXt\ by změnil své umístění.

\stopitemize

Ve skutečnosti \tex{placefigure}, navzdory svému názvu, neslouží pouze k~vkládání obrázků. 
Můžeme s~ním vložit cokoliv, včetně textu. Avšak s~textem nebo jinými položkami vloženými do dokumentu 
pomocí \tex{placefigure} bude zacházeno {\em jako by to byl obrázek}, i~když tomu tak není; 
budou přidány do seznamu obrázků interně spravovaných pomocí \ConTeXt{}u a~můžeme použít transformace 
podobné těm, které používáme pro obrázky, jako je změna měřítka nebo rotace, orámování atd. 
Tedy následující příklad: 

\placefigure
  [here, force]
  [fig:testtext]
  {Using \backslash placefigure for text transformations}
  {\rotate[rotation=180]{\framed{\tfd Test text}}}

čehož je dosaženo následovně:

\starttyping
\placefigure
  [here, force]
  [fig:testtext]
  {Using \backslash placefigure for text transformations}
  {\rotate[rotation=180]{\framed{\tfd Test text}}}
\stoptyping

\stopsubsection

\startsubsection
  [title={Vkládání obrázků integrovaných do textového bloku}]

S~výjimkou velmi malých obrázků, které lze integrovat do řádku bez přílišného narušení mezer 
mezi odstavci, se obrázky obvykle vkládají do odstavce, který obsahuje pouze je (nebo jinými slovy, 
obrázek lze považovat za odstavec v~jeho vlastní právo). Pokud je obrázek vložen pomocí 
\tex{placefigure} a~jeho velikost to dovoluje, v~závislosti na tom, co jsme uvedli ohledně 
jeho umístění (viz \in{section}[sec:placingobjects]), \ConTeXt\ povolí text z~předchozího 
a~následujícího odstavce obtékají obraz. Pokud však chceme zajistit, že určitý obrázek nebude 
oddělen od určitého textu, můžeme použít prostředí {\tt \PlaceMacro{startfiguretext}figuretext}, 
jehož syntaxe je následující:

\starttyping
  \startfiguretext
    [Options]
    [Label]
    {Title}
    {Image}

    ... Text

  \stopfiguretext
\stoptyping

Argumenty prostředí jsou úplně stejné jako pro \tex{placefigure} a~mají stejný význam. 
Ale zde již nejsou možnosti pro umístění plovoucího objektu, ale indikace týkající se integrace 
obrázku do odstavce; takže například \MyKey{left} zde znamená, že obrázek bude umístěn vlevo,
z~atímco text poteče vpravo, zatímco \MyKey{left, bottom} bude znamenat, 
že obrázek musí být umístěn vlevo dole v~textu s~ním spojeného. 

Text napsaný v~prostředí je to, co bude obtékat obraz.

\stopsubsection

\startsubsection
  [
    reference=sec:configimage,
    title={Vložená konfigurace a~transformace obrázků},
  ]

\startsubsubsection
  [title={Vložení možnosti příkazu, které způsobí určitou transformaci obrázku}]
  \PlaceMacro{setupexternalfigures}

Poslední argument v~příkazu \tex{externalfigure} nám umožňuje provést určité úpravy 
vloženého obrázku. Můžeme provést tyto úpravy: 

\startitemize

\item Obecně pro všechny obrázky, které se mají vložit do dokumentu; nebo pro vložení všech obrázků 
z~určitého bodu. V~tomto případě provedeme úpravu příkazem \tex{setupexternalfigures}.

\item Pro konkrétní obrázek, který chceme do dokumentu vložit několikrát. V~tomto případě se úprava 
provádí v~posledním argumentu příkazu \tex{useexternalfigure}, který spojuje externí postavu se 
symbolickým názvem.

\item Přesně ve chvíli, kdy vkládáme konkrétní obrázek. V~tomto případě se úprava provádí 
v~samotném příkazu \tex{externalfigure}.

\stopitemize

Změny v~obrazu, kterých lze dosáhnout touto cestou, jsou následující:

\startdescription{Změna velikosti obrázku.}

Můžeme udělat toto:

\startitemize

\item {\em Přiřazením přesné šířky nebo výšky} se něco udělá s~možnostmi {\tt width} a~{\tt height}; 
pokud je upravena pouze jedna ze dvou hodnot, druhá se automaticky přizpůsobí tak, 
aby byl zachován poměr.

  Můžeme přiřadit přesnou výšku nebo šířku nebo je uvést jako procento výšky stránky nebo 
  šířky řádku. Například:

  \type{width=.4\textwidth}

  zajistí, že obrázek bude mít šířku rovnou 40\% šířky čáry.

\item {\em Změna měřítka obrázku}: Volba {\tt xscale} změní měřítko obrázku vodorovně; 
{\tt yscale} to udělá vertikálně a~{\tt scale} to udělá horizontálně a~vertikálně. 
Tyto tři možnosti očekávají číslo reprezentující faktor měřítka vynásobené 1000. 
To znamená: {\tt scale=1000} ponechá obrázek v~původní velikosti, zatímco {\tt scale=500} 
jej zmenší na polovinu, a~{\tt scale=2000} zdvojnásobí svou velikost.

  Podmíněné měřítko, které se použije pouze v~případě, že obrázek překročí určité rozměry, 
se získá pomocí voleb {\tt maxwidth} a~{\tt maxheight}, které převezmou rozměr. 
Například {\tt maxwidth=.2\backslash textwidth} změní měřítko obrázku pouze v~případě, že se ukáže, 
že je větší než 20\% šířky čáry. 

\stopitemize  

\stopdescription

\startdescription{Otáčení obrázku.}

K~otočení obrázku používáme volbu {\tt orientace}, která přebírá číslo reprezentující počet stupňů 
natočení, které budou použity. Otáčení se provádí proti směru hodinových ručiček.

  \startSmallPrint

     Z~wiki vyplývá, že rotace, kterých lze dosáhnout touto volbou, musí být násobky 90 (90, 180 
     nebo 270), ale abychom dosáhli jiné rotace, museli bychom použít příkaz \tex{rotate}. 
     Neměl jsem však žádný problém použít otočení o~45 stupňů na obrázek pouze s~{\tt orientací=45}, 
     aniž bych musel použít příkaz \tex{rotate}. 

\stopSmallPrint
  
\stopdescription

\startdescription{Zarámování obrázku.}

Můžeme také obklopit obrázek rámečkem pomocí možnosti {\tt frame=on} a~nakonfigurovat jeho barvu 
({\tt framecolor}), vzdálenost mezi rámečkem a~obrázkem ({\tt frameoffset}), tloušťku čáry, 
která kreslí rám ({\tt rulethickness}) nebo tvaru jeho rohů ({\tt framecorner}), 
které mohou být zaoblené ({\tt round}) nebo obdélníkové.

\stopdescription

\startdescription{Další konfigurovatelné aspekty obrázků.} 

Kromě již viděných aspektů, které znamenají transformaci obrázku, který se má vložit, 
můžeme pomocí \tex{setupexternalfigures} nakonfigurovat další aspekty, například kde hledat 
obrázek (volba {\tt directory}), zda by se měl obrázek hledat pouze v~uvedeném adresáři ({\tt location=global}) 
nebo zda by měl obsahovat také pracovní adresář a~jeho nadřazené adresáře ({\tt location=local}) 
a~zda obrázek bude nebo nebude interaktivní ({\tt interakce}) atd.
  
\stopdescription

\stopsubsubsection

\startsubsubsection
   [title={Specifické příkazy pro transformaci obrázku}]

V~\ConTeXt{}u\ jsou tři příkazy, které vytvářejí určitou transformaci v~obrázku 
a~lze je použít v~kombinaci s~\tex{externalfigure}. Tady jsou:

\startitemize
  
\item {\em Zrcadlový obraz}: dosaženo pomocí příkazu \PlaceMacro{mirror}\tex{mirror}.

\item {\em Clipping}: toho lze dosáhnout pomocí příkazu \PlaceMacro{clip}\tex{clip}, 
když šířka ({\tt width}), výška ({\tt height}), horizontální odsazení 
({\tt jsou uvedeny rozměry hoffset}) a~vertikální odsazení ({\tt voffset}). Například:

\starttyping
\clip
  [width=2cm, height=1cm, hoffset=3mm, voffset=5mm]
  {\externalfigure[logo.pdf]}
\stoptyping

\item {\em Rotace.}
   Třetí příkaz schopný aplikovat transformace na obrázek je příkaz \PlaceMacro{rotate}\tex{rotate}. 
Může být použit ve spojení s~\tex{externalfigure}, ale normálně by to nebylo nutné vzhledem k~tomu, 
že druhý má, jak jsme viděli, volbu {\tt orientace}, která poskytuje stejný výsledek.

\startSmallPrint

\stopSmallPrint

\stopitemize 

Typické použití těchto příkazů je s~obrázky, ale ve skutečnosti působí na {\em boxy}. 
To je důvod, proč je můžeme použít na jakýkoli text jednoduše tak, 
že jej uzavřeme do rámečku (což příkaz provede automaticky), což způsobí zvláštní efekty, 
jako jsou následující: 

\startDoubleExample
\vbox{\starttyping
  \mirror{Test text}\\
  \rotate[rotation=20] {Test text}
\stoptyping}

\vbox{  \mirror{Test text}\\
  \rotate[rotation=20]
     {Test text}}

\stopDoubleExample

\stopsubsubsection

\stopsubsection

\stopsection

\startsection
   [
     reference=sec:tables,
     title={Tabulky},
   ]

\startsubsection
   [title={Obecné představy o~tabulkách a~jejich umístění v~dokumentu}] 

Tabulky jsou strukturované objekty, které obsahují text, vzorce nebo dokonce obrázky uspořádané 
do série {\em buněk}, které nám umožňují graficky vidět vztah mezi obsahem každé buňky. 
K~tomu jsou informace uspořádány do řádků a~sloupců: všechna data (nebo položky) ve stejném řádku 
mají mezi sebou určitý vztah, stejně jako všechna data (nebo položky) ve stejném sloupci. 
Buňka je průsečík řádku se sloupcem, jak je znázorněno na \in{figure}[fig:tabulce].

\placefigure
  [right]
  [fig:table]
  {Image of a~simple table}
  {\externalfigure[tablas_eng][width=.6\textwidth]}

Tabulky jsou ideální pro zobrazení textu nebo dat, které spolu souvisejí, 
protože každá je uzamčena ve své vlastní buňce, i~když se její obsah nebo obsah zbývajících buněk 
mění, relativní pozice jedné vůči ostatním se nezmění.

Ze všech úkonů spojených se sazbou textu je tvorba tabulek jedinou, podle mého názoru, 
snadněji proveditelnou v~grafickém programu (typ textového procesoru) než
v~\ConTeXt{}u. 
Protože je jednodušší {\em nakreslit} tabulku (což je to, co děláte v~programu pro zpracování textu), 
než {\em to popsat}, což je to, co děláme, když pracujeme s~\ConTeXt{}em. Každá změna řádku 
a~sloupce vyžaduje přítomnost příkazu, což znamená, že implementace tabulky trvá mnohem déle, 
místo toho, abychom jednoduše řekli, kolik řádků a~sloupců chceme.

\ConTeXt\ má tři různé mechanismy pro vytváření tabulek; prostředí {\tt tabulate}, 
které se doporučuje pro jednoduché tabulky a~které je nejvíce přímo inspirováno tabulkami
\TeX{}u; tzv. {\em natural tables}, inspirované HTML tabulkami, vhodné pro tabulky se speciálními 
požadavky na design, kde například nemají všechny řádky stejný počet sloupců; 
a~takzvané {\em extrémní tabulky}, jasně založené na XML a~doporučené pro střední nebo 
dlouhé tabulky, které zabírají více než jednu stránku. Ze tří vysvětlím pouze první. 
Přirozené tabulky jsou poměrně dobře vysvětleny v~\quotation{\ConTeXt\ Mark IV exkurze} 
a~pro {\em extrémní tabulky} je o~nich dokument v~dokumentaci \suite-.

Něco podobného,se děje s~obrázky, které se vyskytují v~tabulkách: můžeme jednoduše napsat 
potřebné příkazy v~určitém bodě dokumentu pro vygenerování tabulky a~ta bude vložena přesně 
v~tomto bodě, nebo můžeme použít \PlaceMacro{placetable}\tex příkaz {placetable} pro vložení tabulky. 
To má některé výhody:

\startitemize

\item \ConTeXt\ očísluje tabulku a~přidá ji do seznamu tabulek umožňujících vnitřní odkazy 
na tabulku (prostřednictvím jejího číslování) a~zahrnout ji do případného rejstříku tabulek.

\item Získáme flexibilitu v~umístění tabulek v~dokumentu, čímž si usnadníme úlohu stránkování.

\stopitemize

Formát \tex{placetable} je podobný tomu, co jsme viděli
\tex{placefigure} (viz \in{section}[sec:placefigure]): 

\type{\placetable[Options] [Label] {Title} {table}}

Odkazuji na sekce \in{}[sec:umístění objektů] a~\in{}[sec:confcaptions] ohledně možností 
týkajících se umístění tabulky a~konfigurace nadpisu. Mezi možnostmi je však jedna, 
která se zdá být určena výhradně pro tabulky. Toto je volba \MyKey{split}, která, když je nastavena, 
opravňuje \ConTeXt\ rozšířit tabulku na dvě nebo více stránek, v~takovém případě tabulka již 
nemůže být plovoucím objektem.

Obecně můžeme konfiguraci pro tabulky nastavit příkazem \PlaceMacro{setuptables}\tex{setuptables}. 
Stejně jako u~obrázků je také možné generovat index tabulek pomocí
\PlaceMacro{placelistoftables}\tex{placelistoftables} nebo \PlaceMacro{completelistoftables}\tex{completelistoftables}.
V~tomto ohledu viz \in{section}[sec:variouslists]. 

\stopsubsection

\startsubsection
   [title={Jednoduché tabulky s~prostředím {\tt tabulate}}]
\PlaceMacro{starttabulate}

Nejjednodušší tabulky jsou ty, kterých se dosáhne pomocí {\em
   tabulate} prostředí, jehož formát je:

\vbox{\starttyping
   \starttabulate[Rozvržení sloupců tabulky]
     ... % Obsah tabulky
     ...
     ...
   \stoptabulate
\stoptyping} 

Kde argument v~hranatých závorkách popisuje (v~kódu) počet sloupců, které tabulka bude mít 
a~(někdy nepřímo) udává jejich šířku. Říkám, že argument popisuje design {\em v~kódu}, 
protože na první pohled působí velmi tajemně: skládá se z~posloupnosti znaků, 
z~nichž každý má zvláštní význam. Vysvětlím to postupně a~po krocích, protože si myslím, 
že takto je to srozumitelnější.

\startSmallPrint

   Toto je typický případ, kdy obrovské množství aspektů, které můžeme konfigurovat, znamená, 
že k~jejich popisu potřebujeme hodně textu. Zdá se, že je to ďábelsky obtížné. 
Ve skutečnosti pro většinu tabulek, které jsou postaveny v~praxi, stačí body 1 a~2. 
Zbytek jsou další možnosti, o~kterých je užitečné vědět, že existují, ale nejsou nezbytné 
pro sazbu tabulky.

\stopSmallPrint


\startitemize[n] 

\item {\bf Oddělovač sloupců}: znak \MyKey{\|} se používá k~oddělování sloupců tabulky. 
Takže například \MyKey{[\|lT\|rB\|]} bude popisovat tabulku se dvěma sloupci, 
z~nichž jeden by měl vlastnosti spojené s~indikátory \MyKey{l} a~\MyKey{T} 
(který uvidíme hned za ním) a~druhý sloupec bude mít vlastnosti spojené s~\MyKey{r} a~\MyKey{B}. 
Jednoduchá třísloupcová tabulka zarovnaná doleva by například byla popsána jako \MyKey{[\|l\|l\|l\|]}.

\item {\bf Určení základní povahy buněk ve sloupci:} První věc, kterou je třeba určit, 
když sestavujeme naši tabulku, je, zda chceme, aby byl obsah každé buňky zapsán na jeden řádek, 
nebo zda naopak , pokud je text libovolného sloupce příliš dlouhý, chceme, 
aby jej naše tabulka rozložila na dva nebo více řádků. V~prostředí {\tt tabulate} se 
o~této otázce nerozhoduje buňka po buňce, ale je považována za charakteristiku sloupců.

   \startitemize[a]

\item {\em Jednořádkové buňky:} Pokud má být obsah buněk ve sloupci, bez ohledu na jejich délku, 
zapsán na jeden řádek, musíme určit zarovnání textu ve sloupci, který lze ponechat (\MyKey{l}, 
od {\em left}), vpravo (\MyKey{r}, od {\em right}) nebo na střed
     (\MyKey{c}, z~{\em center}).


     \startSmallPrint

V~zásadě budou tyto sloupce tak široké, aby se vešly do nejširší buňky. 
Ale můžeme omezit šířku sloupce pomocí specifikátoru \MyKey{w(Width)}. 
Například \MyKey{[\|rw(2cm)\|c\|c\|]} bude popisovat tabulku se dvěma sloupci, 
z~nichž první je zarovnán doprava a~má přesnou šířku 2 centimetry, a~další dva uprostřed 
a~bez omezení šířky.

      Je třeba poznamenat, že omezení šířky v~jednořádkových sloupcích může způsobit, 
že text v~jednom sloupci překryje text v~dalším sloupci. Takže moje rada je, 
že když potřebujeme sloupce s~pevnou šířkou, vždy používejte víceřádkové sloupce buněk.

    \stopSmallPrint
    
  \item {\em Buňky, které mohou v~případě potřeby zabírat více než jeden řádek}: 
specifikátor \MyKey{p} generuje sloupce, ve kterých text v~každé buňce zabere tolik řádků, 
kolik je potřeba. Pokud jednoduše zadáme \MyKey{p}, šířka sloupce bude plná dostupná šířka. 
Ale je také možné uvést \MyKey{p(Width)}, v~takovém případě bude šířka přesně specifikovaná. 
Tedy následující příklady:

\starttyping
\starttabulate[|l|r|p|]
\starttabulate[|l|p(4cm)|]
\starttabulate[|r|p(.6\textwidth)|]
\starttabulate[|p|p|p|]
\stoptyping

První příklad vytvoří tabulku se třemi sloupci, prvním a~druhým z~jednoho řádku, 
zarovnanými doleva a~doprava, a~třetím, který zabere zbývající šířku a~výšku potřebnou 
k~uložení veškerého jejího obsahu. Ve druhém příkladu bude druhý sloupec měřit přesně čtyři centimetry 
na šířku bez ohledu na jeho obsah; ale pokud se do toho prostoru nevejde, zabere více než jeden řádek. 
Třetí příklad vypočítá šířku druhého sloupce v~poměru k~maximální šířce čáry a~v~posledním příkladu 
budou tři sloupce, jejichž šířka bude mít stejnou šířku.

  \stopitemize


  \startSmallPrint

    Všimněte si, že ve skutečnosti, pokud je buňka čtyřúhelník, specifikátor \MyKey{p} autorizuje 
proměnnou výšku buněk ve sloupci v~závislosti na délce textu.
    
  \stopSmallPrint

\item {\bf Přidání označení k~popisu sloupce, o~stylu a~variantě písma, které se má použít}: 
jakmile se rozhodne o~základní povaze sloupce (šířka a~výška buněk, automatická nebo pevná), 
může ještě přidat do popisu obsahu sloupce znak reprezentující {\em formát}, 
ve kterém musí být zapsán. Tyto znaky mohou být \MyKey{B} pro tučné písmo, \MyKey{I} pro kurzívu, 
\MyKey{S} pro šikmé písmo, \MyKey{R} pro písmo v~římském stylu nebo \MyKey{T} pro styl {\em typewriter} 
nápis.


\starthead {\bf Další aspekty, které lze specifikovat v~popisu sloupců tabulky}:\stophead

   \startitemize[1]

   \item {\em Sloupce s~matematickými vzorci}: specifikátory \MyKey{m} a~\MyKey{M} umožňují matematický 
režim ve sloupci, aniž by bylo nutné jej uvádět v~každé z~jeho buněk. Buňky v~tomto sloupci nebudou 
schopny pojmout normální text.

%     \startSmallPrint

\startSmallPrint

      Přestože \TeX, předchůdce \ConTeXt{}u, vznikl pro sazbu jakéhokoli druhu matematiky, 
až dosud jsem o~psaní matematiky téměř nic neřekl. V~matematickém režimu (který nebudu vysvětlovat) 
\ConTeXt\ mění naše normální pravidla a~dokonce používá jiné fonty. 
Matematický režim má dvě varianty: jeden bychom mohli nazvat {\em lineární} v~tom, 
že vzorec je umístěn v~řádku obsahujícím normální text (indikátor \MyKey{m}) 
a~{\em kompletní matematický režim}, který zobrazuje vzorce v~prostředí, kde není normální text. 
Hlavním rozdílem mezi těmito dvěma režimy v~tabulce je v~podstatě velikost, 
ve které bude vzorec zapsán, a~horizontální a~vertikální prostor, který jej obklopuje.

    \stopSmallPrint

  \item {\em Přidat další horizontální bílé místo kolem obsahu buněk ve sloupci}: 
pomocí indikátorů \MyKey{in}, \MyKey{jn} a~\MyKey{kn} můžeme přidat další bílé místo vlevo obsahu 
sloupce (\MyKey{in}), vpravo (\MyKey{jn}) nebo na obě strany (\MyKey{kn}). 
Ve všech třech případech představuje \MyKey{n} číslo, kterým se vynásobí prázdné místo, 
které by normálně zůstalo bez jednoho z~těchto specifikátorů (standardně je průměr {\em em}). 
Takže například \MyKey{\|j2r\|} bude indikovat, že stojíme před sloupcem, který bude zarovnán doprava 
a~ve kterém chceme prázdné místo o~šířce 1 {\em em}.

  \item {\em Přidání textu před nebo za obsah každé buňky ve sloupci}. Specifikátory {\tt b\{Text\}} 
a~{\tt a\{Text\}} způsobí, že text mezi složenými závorkami bude zapsán před (\MyKey{b}, 
z~{\em before}) nebo za ( \MyKey{a}, od {\em after}) obsah buňky.

  \item {\em Použití příkazu format na celý sloupec}. Indikátory \MyKey{B}, \MyKey{I}, \MyKey{S}, 
\MyKey{R} \MyKey{T}, které jsme zmínili dříve, nepokrývají všechny možnosti formátu: 
např. není tam žádný indikátor pro malá písmena nebo pro {\em sans serif}, 
nebo který ovlivňuje velikost písma. Pomocí indikátoru \MyKey{h\backslash command} 
můžeme zadat příkaz formátu, který se automaticky použije na všechny buňky ve sloupci. 
Například \MyKey{\|lf\backslash sc\|} vysází obsah sloupce velkými písmeny.

  \item {\em Použití libovolného příkazu na všechny buňky ve sloupci}. Nakonec indikátor
\MyKey{h\backslash příkaz}

použije zadaný příkaz na všechny buňky ve sloupci.


\stopitemize

\stopitemize

V~\in{table}[tbl:examplestabulate] jsou uvedeny některé příklady řetězců specifikace formátu tabulky.

\placetable
   [tady]
   [tbl:examplestabulate]
   {Některé příklady, jak určit formát sloupců v~{\tt tabulate}}
{\starttabulate[|lT|p(.6\textwidth)|]
\HL
\NC{\bf\rm Specifikátor formátu}
\NC{\bf Význam}
\NR
\HL
\NC \|l\|
\NC Vygeneruje sloupec, jehož šířka je automaticky zarovnána doleva.
\NR
\NC \|rB\|
\NC Vygeneruje sloupec, jehož šířka je automaticky zarovnána doprava a~je uvedena tučně.
\NR
\NC \|cIm\|
\NC Vygeneruje sloupec povolený pro matematický obsah. Na střed a~kurzívou.
\NR
\NC \|j4cb\{---\}\|
\NC Tento sloupec bude mít obsah vystředěný, bude začínat em pomlčkou (---) a~přidá 2 {\em ems} mezery vpravo.
\NR
\NC \|l\|p(.7\tex{šířka textu})\|
\NC generuje dva sloupce: první je zarovnán doleva a~má automatickou šířku. Druhá zabírá 70\% celkové šířky čáry.
\NR
\HL
\stoptabulate}

Jakmile je tabulka navržena, je třeba zadat její obsah. Abych vysvětlil, jak to udělat, 
začnu popisem, jak by měla být vyplněna tabulka, kde máme linky oddělující řádky a~sloupce:

\startitemize

\item Vždy začínáme nakreslením vodorovné čáry. V~tabulce se to provede příkazem \PlaceMacro{HL}\tex{HL} 
(z~{\em Horizontal Line}).

\item Poté napíšeme první řádek: na začátku každé buňky musíme označit, že začíná nová buňka 
a~že je třeba nakreslit svislou čáru. To se provádí příkazem \PlaceMacro{VL}\tex{VL} 
(z~{\em Vertical Line}). Začneme tedy tímto příkazem a~zapíšeme obsah každé buňky. Pokaždé, 
když měníme buňky, opakujeme příkaz \tex{VL}.

\item Na konci řádku výslovně označíme, že bude spuštěn nový řádek příkazem \PlaceMacro{NR}\tex{NR} 
(z~{\em Next Row}). Poté zopakujeme \tex{HL} a~nakreslíme novou vodorovnou čáru.

\item A~tak jeden po druhém zapíšeme všechny řádky tabulky. Když skončíme, 
přidáme jako další příkaz \tex{NR} a~další \tex{HL} pro uzavření mřížky spodní vodorovnou čarou.

\stopitemize

Pokud nechceme kreslit mřížku tabulky, odstraníme příkazy \tex{HL} a~nahradíme příkazy \tex{VL} 
příkazy \PlaceMacro{NC}\tex{NC} (z~{\em Nový sloupec}).

Není to nijak zvlášť obtížné, když to pochopíme, i~když se podíváme na zdrojový kód tabulky, 
je těžké získat představu, jak bude vypadat. V~\in{table}[tbl:tablecommands] vidíme příkazy, 
které mohou (a~musí) být použity v~tabulce. Jsou některé, které jsem nevysvětlil, ale myslím, 
že popis, který jsem uvedl, stačí.

\placetable
   [tady, síla]
   [tbl:tablecommands]
   {Příkazy k~použití v~tabulce}
{\starttabulate[|l|p(.6\textwidth)|]
\HL
\NC {\bf Příkaz}
\NC {\bf Význam}
\NR
\HL
\NC \tex{HL}
\NC Vloží vodorovnou čáru
\NR
\NC \tex{NC}
\NC Začne nový sloupec
\NR
\NC \tex{NR}
\NC Začne nový řádek
\NR
\NC \tex{VL}
\NC Vloží svislou čáru ohraničující sloupec (používá se místo \tex{NC})
\NR
\NC \PlaceMacro{NN}\tex{NN}
\NC Začne sloupec v~matematickém režimu (používá se místo \tex{NC})
\NR
\NC \PlaceMacro{TB}\tex{TB}
\NC Přidá další vertikální mezeru mezi dva řádky
\NR
\NC \PlaceMacro{NB}\tex{NB}
\NC Označuje, že další řádek začíná nedělitelný blok, ve kterém nemůže být konec stránky
\NR
\HL
\stoptabulate}

A~nyní jako příklad přepíšu kód, kterým byl napsán \in{table}[tbl:tablecommands].

\starttyping
\placetable
   [tady]
   [tbl:tablecommands]
   {Příkazy k~použití v~tabulce}
{\starttabulate[|l|p(.6\textwidth)|]
\HL
\NC {\bf Příkaz}
\NC {\bf Význam}
\NR
\HL
\NC \tex{HL}
\NC Vloží vodorovnou čáru
\NR
\NC \tex{NC}
\NC Začne nový sloupec
\NR
\NC \tex{NR}
\NC Začne nový řádek
\NR
\NC \tex{VL}
\NC Vloží svislou čáru ohraničující sloupec (používá se místo \tex{NC})
\NR
\NC \tex{NN}
\NC Začne sloupec v~matematickém režimu (používá se místo \tex{NC})
\NR
\NC \tex{TB}
\NC Přidá další vertikální mezeru mezi dva řádky
\NR
\NC \tex{NB}
\NC Označuje, že další řádek začíná nedělitelný blok, ve kterém nemůže být konec stránky
\NR
\HL
\stoptabulate}
\stoptyping

Čtenář si všimne, že jsem obecně použil jeden (nebo dva) řádky textu pro každou buňku. 
Ve skutečném zdrojovém souboru bych pro každou buňku použil pouze řádek textu; 
v~příkladu jsem rozdělil řádky, které jsou příliš dlouhé. Použití jednoho řádku na buňku mi 
usnadňuje psaní tabulky, protože to, co dělám, je psát obsah každé buňky bez příkazů
k~oddělení řádků nebo sloupců. Když je vše napsáno, vyberu text z~tabulky a~požádám svůj textový editor, 
aby na začátek každého řádku vložil \quotation{\tex{NC }}. Poté každé dva řádky 
(protože tabulka má dva sloupce) vložím řádek, který přidá příkaz \tex{NR}, 
protože každé dva sloupce začíná nový řádek. Nakonec ručně vložím příkazy \tex{HL} do bodů, kde chci, 
aby se objevila vodorovná čára. Skoro déle mi trvá to popsat, než to udělat!

Ale také se podívejte, jak v~rámci tabulky můžeme používat běžné příkazy
\ConTeXt{}u. 
Zejména v~této tabulce neustále používáme \tex{tex}, což je vysvětleno v~\in{section}[sec:verbatim].

\stopsubsection

\stopsection

\startsection
   [title={Aspekty společné pro obrázky, tabulky a~další plovoucí objekty}]

Již víme, že obrázky a~tabulky nemusí být plovoucí objekty, ale jsou dobrými kandidáty, 
aby tomu tak bylo, i~když je třeba je vložit do dokumentu pomocí příkazů \tex{placefigure} nebo 
\tex{placetable}. Kromě těchto dvou příkazů se stejnou strukturou máme
v~\ConTeXt{}u\ příkaz \PlaceMacro{placechemical}\tex{placechemical} (pro vložení chemických vzorců), 
příkaz \PlaceMacro{placegraphic}\tex{placegraphic} (pro vložení grafiky) 
a~příkaz \PlaceMacro{placeintermezzo}\tex{placeintermezzo} pro vložení struktury, 
kterou \ConTeXt\ nazývá {\em Intermezzo} a~která tuším odkazuje na fragmenty orámovaného textu. 
Všechny tyto příkazy jsou zase konkrétními aplikacemi obecnějšího příkazu, 
kterým je \PlaceMacro{placefloat}\tex{placefloat}, jehož syntaxe je následující:

\type{\placefloat [Název] [Možnosti] [Štítek] {Název} {Obsah}}

Všimněte si, že \tex{placefloat} je identický s~\tex{placefigure} a~\tex{placetable} 
kromě prvního argumentu, který v~\tex{placefloat} přebírá jméno plovoucího objektu. Je to proto, 
že {\em každý typ plovoucího objektu lze do dokumentu vložit dvěma různými příkazy}: 
\tex{placefloat[TypeName]} nebo \tex{placeTypeName}. Jinými slovy: \tex{placefloat[figure]} 
a~\tex{placefigure} jsou přesně stejné příkazy, stejně jako \tex{placefloat[table]} 
je stejný příkaz jako \tex{placetable}.

Proto budu od nynějška mluvit o~\tex{placefloat}, ale mějte na paměti, že vše, co řeknu, 
bude platit také pro \tex{placefigure} nebo \tex{placetable}, což jsou specifické aplikace uvedeného příkazu.

Argumenty \tex{placefloat} jsou:

\startitemize

\item {\em Jméno} odkazuje na dotyčný plovoucí objekt. Může to být nějaký předem určený plovoucí 
objekt ({\tt obrázek, tabulka, chemikálie, intermezzo}) nebo plovoucí objekt vytvořený námi 
pomocí \tex{definefloat} (viz \in{section}[sec:definefloat]).

\item {\em Možnosti} Série symbolických slov, která \ConTeXt{}u\ říkají, jak má vložit objekt. 
Velká většina z~nich odkazuje na to {\em kam} vložit. To uvidíme v~další části.

\item {\em Štítek} Označení pro budoucí interní odkazy na tento objekt.

\item {\em Název} Text nadpisu, který se má přidat k~objektu. Ohledně jeho konfigurace 
viz \in{section}[sec:confcaptions].

\item {\em Obsah} To samozřejmě závisí na typu objektu. Pro obrázky je to obvykle příkaz 
\tex{externalimage}; pro tabulky, příkazy, které vytvoří tabulku; pro {\em intermezzi}, 
fragment textu v~rámečku; atd.

\stopitemize

První tři argumenty, které jsou uvedeny v~hranatých závorkách, jsou volitelné. Poslední dva 
(které jsou uvedeny mezi složenými závorkami) jsou povinné, i~když mohou být prázdné. 
Takže například:
\cmd{placefloat\{\}\{\}} vloží:

\placefloat{}{}

v~dokumentu.

\startitemize

{\bf Poznámka:} Vidíme, že \ConTeXt\ uvažoval, že objekt, který má být vložen, byl obrázek, 
protože byl očíslován jako obrázek a~zahrnut v~seznamu obrázků \Conjecture. To mě nutí předpokládat, 
že obrázky jsou výchozí plovoucí objekty.

\stopitemize 

\startsubsection
   [
     reference=sec:placeingobjects,
     title={Možnosti vložení plovoucího objektu},
   ]

Argument {\em Options} v~\tex{placefigure}, \tex{placetable} a~\tex{placefloat} řídí různé aspekty 
týkající se vkládání těchto typů objektů. Hlavně místo na stránce, kam bude objekt vložen. 
Zde je podporováno několik hodnot, každá jiné povahy:

\startitemize

\item Některá místa vložení jsou stanovena ve vztahu k~prvkům stránky ({\tt top, bottom inleft, 
inright, inmargin, margin, leftmargin, rightmargin, leftedge, rightedge, innermargin, inneredge, 
vnější hrana, vnitřní, vnější}). Musí se samozřejmě jednat o~objekt, který se vejde do oblasti, 
kde má být umístěn, a~pro tento prvek musí být v~rozvržení stránky vyhrazeno místo. 
Ohledně toho viz sekce \in{}[sec:page-elements] a~\in{}[sec:pagelayout].

\item Další možná místa vložení se více vztahují k~textu obklopujícímu objekt a~jsou indikací, 
kam by měl být objekt umístěn, aby kolem něj text obtékal. V~zásadě hodnoty {\tt left} a~{\tt right}.

\item Volba {\tt here} je interpretována jako doporučení ponechat objekt na místě ve zdrojovém souboru, 
kde se nachází. Toto {\em doporučení} nebude respektováno, pokud to požadavky na stránkování neumožňují. 
Tato indikace je posílena, pokud přidáme volbu {\tt force}, což znamená přesně
toto: vynutit vložení objektu do tohoto bodu. Všimněte si, že po vynucení vložení v~určitém bodě 
již objekt nebude plovoucí.

\item Další možné volby se týkají stránky, na kterou má být objekt vložen: \MyKey{stránka} jej vloží 
na novou stránku; \MyKey{opposite} jej vloží na stránku naproti aktuální; \MyKey{leftpage} 
na sudé stránce; \MyKey{rightpage} na liché stránce.
  
\stopitemize

Existují některé možnosti, které nesouvisejí s~umístěním objektu.
Mezi nimi:

\startitemize

\item {\tt none}: Tato volba potlačí název.

\item {\tt split}: Tato volba umožňuje objektu rozšířit se na více než jednu stránku. 
Musí to být samozřejmě předmět, který je od přírody dělitelný, například
tabulka. Když je tato možnost použita a~objekt je rozdělen, nelze již říci, že je plovoucí.

\stopitemize

\stopsubsection

\startsubsection
   [
     reference=sec:confcaptions,
     title={Konfigurace názvů plovoucích objektů},
   ]

Pokud nepoužijeme volbu \MyKey{none} v~\tex{placefloat}, ve výchozím nastavení jsou plovoucí objekty 
spojeny s~názvem, který se skládá ze tří prvků:

\startitemize

\item Jméno daného typu objektu. Toto jméno je přesně stejné jako jméno typu objektu; 
takže pokud například definujeme nový plovoucí objekt s~názvem \quotation{sekvence} 
a~vložíme \quotation{sekvenci} jako plovoucí objekt, bude nadpis \quotation{sekvence 1}. 
Jednoduše zadejte název objektu velkým písmenem.

  \startSmallPrint

    Navzdory tomu, co bylo právě řečeno, pokud hlavním jazykem dokumentu není angličtina, 
bude přeložen anglický název pro předdefinované objekty, jako jsou například objekty \MyKey{figure} 
nebo \MyKey{table}; Takže například objekt \MyKey{figure} v~dokumentech ve španělštině se nazývá 
\MyKey{Figura}, zatímco objekt \MyKey{table} se nazývá \MyKey{Tabla}. Tyto španělské názvy 
pro předdefinované objekty lze změnit pomocí \tex{setuplabeltext}, jak je vysvětleno v~\in{section}[sec:labels].
    
  \stopSmallPrint

\item Jeho číslo. Ve výchozím nastavení jsou objekty číslovány podle kapitol, 
takže první tabulka v~kapitole 3 bude tabulka \quote{3.1}.

\item Jeho obsah. Zavedeno jako argument \tex{placefloat}.

\stopitemize

Pomocí \PlaceMacro{setupcaptions}\tex{setupcaptions} nebo \PlaceMacro{setupcaption}\tex{setupcaption[Object]} 
můžeme změnit systém číslování a~vzhled samotného titulku. První příkaz ovlivní všechny názvy 
všech objektů a~druhý ovlivní pouze název určitého typu objektu:

\startitemize

\item Pokud jde o~systém číslování, je řízen volbami {\tt number}, {\tt way}, {\tt prefixsegments} 
a~{\tt numberconversion}:

  \startitemize

  \item {\tt číslo} může přijmout {\tt ano}, {\tt ne} nebo
    {\tt none} hodnoty a~řídí, zda bude číslo nebo ne.

  \item {\tt way} udává, zda bude číslování v~celém dokumentu sekvenční ({\tt way=bytext}), 
nebo zda bude znovu začínat na začátku každé kapitoly ({\tt way=bychapter}) 
nebo oddílu ( {\tt way=bysection}). V~případě restartu je vhodné koordinovat hodnotu této volby 
s~volbou {\tt prefixsegments}.

  \item {\tt prefixsegments} udává, zda číslo bude mít {\em prefix} a~co to bude. 
{\tt prefixsegments=chapter} tedy způsobí, že počet objektů vždy začíná číslem kapitoly, 
zatímco {\tt prefixsegments=section} bude před číslem objektu s~číslem sekce.

\item {\tt numberconversion} řídí druh číslování. Hodnoty pro tuto možnost mohou být: Arabská čísla
     (\MyKey{čísla}), malá písmena (\MyKey{a},
     \MyKey{znaky}), velká písmena (\MyKey{A},
     \MyKey{Characters}), malá písmena \MyKey{KA}), velká římská čísla (\MyKey{I}, \MyKey{R},
     \MyKey{Romannumerals}), malá písmena (\MyKey{i},
     \MyKey{r}, \MyKey{romannumerals} nebo malá písmena
     (\MyKey{KR})).

   \stopitemize

\item Vzhled samotného titulku je řízen mnoha možnostmi. Uvedu je, ale pro podrobné vysvětlení 
významu každého z~nich odkazuji na \in{section}[sec:titlestyle], kde je vysvětleno ovládání vzhledu 
příkazů dělení, protože možnosti jsou z~velké části stejný. Jedná se o~tyto možnosti:

  \startitemize

  \item Chcete-li ovládat formát všech prvků titulku,
    {\tt styl, barva, příkaz}.

  \item Chcete-li ovládat pouze formát názvu pro daný druh objektu:
    {\tt headstyle, headcolor, headcommand, headseparator}.

  \item Chcete-li ovládat pouze formát číslování: {\tt
    číselný příkaz}.

  \item Chcete-li ovládat pouze formát samotného titulku:
    {\tt textový příkaz}.

  \stopitemize

\item Můžeme také ovládat další aspekty, jako je vzdálenost mezi různými prvky, které tvoří nadpis, 
šířka nadpisu, jeho umístění vzhledem k~objektu atd. Zde odkazuji na informace
v~\goto{\ConTeXt\ wiki}[url(wiki)] týkající se možností, které lze konfigurovat pomocí tohoto příkazu.

\stopitemize

\stopsubsection

\startsubsection
   [
     reference=sec:startcombination,
     title={Kombinované vložení dvou nebo více objektů},
   ]

Pro vložení dvou nebo více různých objektů do dokumentu tak, aby je \ConTeXt\ držel pohromadě 
a~zacházel s~nimi jako s~jedním objektem, máme prostředí \PlaceMacro{startcombination}\tex{startcombination}, 
jehož syntaxe je:

\type{\startcombination[Ordering] ... \stopcombination}

kde {\em Ordering} udává, jak by měly být objekty seřazeny: pokud je všechny potřeba seřadit 
vodorovně, {\em Ordering} udává pouze počet objektů, které mají být kombinovány. Pokud ale chceme 
objekty spojit do dvou nebo více řádků, budeme muset uvést číslo objektu na řádek, 
za ním počet řádků a~obě čísla oddělit znakem *. Například:

\starttyping
\startcombination[3*2]
  {\externalfigure[test1]}
  {\externalfigure[test2]}
  {\externalfigure[test3]}
  {\externalfigure[test4]}
  {\externalfigure[test5]}
  {\externalfigure[test6]}
\stopcombination
\stoptyping

což vytvoří následující zarovnání obrázků.

\startcombination[3*2]
  {\externalfigure[test1]}
  {\externalfigure[test2]}
  {\externalfigure[test3]}
  {\externalfigure[test4]}
  {\externalfigure[test5]}
  {\externalfigure[test6]}
\stopcombination

V~předchozím příkladu obrázky, které jsem zkombinoval, ve skutečnosti neexistují, 
a~proto místo obrázků \ConTeXt\ vygeneroval textová pole s~informacemi o~nich.

Podívejte se na druhou stranu, jak je každý prvek, který má být kombinován v~\tex{startcombination}, 
uzavřen ve složených závorkách.

Ve skutečnosti nám \tex{startcombination} nejen umožňuje spojovat a~zarovnávat obrázky, 
ale jakýkoli druh {\em boxu}, jako jsou texty v~prostředí \tex{startframedtext}, tabulky atd. 
Ke konfiguraci kombinace můžeme použít příkaz \tex{setupcombination} a~můžeme také vytvořit 
předkonfigurované kombinace pomocí \PlaceMacro{definecombination}\tex{definecombination}.

\stopsubsection

\startsubsection
   [title={Obecná konfigurace plovoucích objektů}]

Již jsme viděli, že pomocí \tex{placefloat} můžeme ovládat umístění vkládaného plovoucího objektu 
a~některé další detaily. Je také možné nakonfigurovat:

\startitemize

\item Globální charakteristiky určitého typu plovoucího objektu. To se provádí pomocí 
\PlaceMacro{setupfloat}\cmd{setupfloat[Název typu plovoucího objektu]}.

\item Globální charakteristiky všech plovoucích objektů v~našem dokumentu. To se provádí pomocí 
\PlaceMacro{setupfloats}\tex{setupfloats}.

\stopitemize

Mějte na paměti, že stejně jako \tex{placefloat[figure]} je ekvivalentní \tex{placefigure}, 
\tex{setupfloat[figure]} je ekvivalentní \tex{setupfigures} a~\tex{setupfloat[table ]} 
je ekvivalentní \tex{setuptables}.

Co se týče konfigurovatelných možností, odkazuji na oficiální seznam příkazů \ConTeXt\ (\in{section}[sec:qrc-setup-cs]).

\stopsubsection

\startsection
   [
     reference=sec:definefloat,
     title={Definování dalších plovoucích objektů},
   ]
   \PlaceMacro{definefloat}

Příkaz \tex{definefloat} nám umožňuje definovat vlastní plovoucí objekty. Jeho syntaxe je:

\type{\definefloat [Jednotné jméno] [Množné číslo] [Konfigurace]}

Kde argument {\em Konfigurace} je volitelný argument, který nám umožňuje uvést konfiguraci 
tohoto nového objektu již v~době jeho vytvoření. Můžeme to udělat i~později pomocí 
\tex{setupfloat[Jména v~jednotném čísle]}.

Vzhledem k~tomu, že tímto oddílem končíme náš úvod, využiji toho, abych se dostal trochu hlouběji 
do zdánlivé {\em džungle} příkazů \ConTeXt{}u\, které, jakmile je pochopíme, není tolik {\em
džunglí}, ale ve skutečnosti je docela racionální.

Začněme tím, že se sami sebe zeptáme, k~čemu vlastně je plovoucí objekt
\ConTeXt{}u, přičemž odpověď zní, že je to objekt s~následujícími vlastnostmi:

\startitemize

\item má určitou volnost s~ohledem na své umístění na stránce;

\item což znamená, že je k~němu přidružen {\em seznam}, který mu umožňuje číslovat tyto druhy objektů 
a~případně generovat jejich index;

\item má název;

\item když objekt se skutečně chová jako plovoucí, musí se s~ním zacházet jako s~nedělitelnou jednotkou, 
což znamená (v~terminologii \TeX{}u) {\em uzavřen v~tzv. boxu}.

\stopitemize

Jinými slovy, plovoucí objekt se ve skutečnosti skládá ze tří prvků: samotného objektu, 
seznamu s~ním spojeného a~názvu. K~ovládání samotného objektu potřebujeme pouze jeden příkaz 
pro nastavení jeho umístění a~další pro vložení objektu do dokumentu; 
pro nastavení aspektů seznamu postačují obecné příkazy pro řízení seznamu a~pro nastavení 
aspektů názvu obecné příkazy pro řízení názvu.

A~zde přichází na řadu genialita \ConTeXt{}u\: jednoduchý příkaz pro ovládání plovoucích objektů 
(\tex{setupfloats}) a~jednoduchý příkaz pro vkládání plovoucích objektů: \tex{placefloat}, 
mohl být navržen: ale co \ConTeXt\ dělá:

\startitemize[n]

\item Navrhněte příkaz k~propojení názvu s~konkrétní konfigurací plovoucího objektu. 
Toto je příkaz \tex{definefloat}, který ve skutečnosti nepropojuje jedno jméno, ale dvě jména, 
jedno v~jednotném čísle a~jedno v~množném čísle.

\item Vytvořte společně s~globálním konfiguračním příkazem plovoucích objektů příkaz, 
který nám umožňuje konfigurovat pouze určitý typ objektu: \tex{setupfloat[Object]}.

\item Přidejte do příkazu umístění plovoucího objektu (\tex{placefloat}), argument, 
který nám umožňuje rozlišovat mezi jedním nebo druhým typem: (\tex{placefloat[Object]}).

\item Vytvoří příkazy, včetně názvu objektu, pro všechny akce plovoucího objektu. 
Některé z~těchto příkazů (které jsou ve skutečnosti klony jiných obecnějších příkazů) 
budou používat název objektu v~jednotném čísle a~jiné jej budou používat v~množném čísle.
  
\stopitemize

Když tedy vytvoříme nový plovoucí objekt a~řekneme \ConTeXt{}u\, jaké je jeho jméno v~jednotném 
a~množném čísle, \ConTeXt:

\startitemize

\item Vyhradí místo v~paměti pro uložení specifické konfigurace daného typu objektu.

\item Vytvoří nový seznam s~jednotným jménem tohoto typu objektu, protože plovoucí objekty jsou 
spojeny se seznamem.

\item Vytvoří nový druh \quotation{title} propojený s~tímto novým typem objektu, 
aby byla zachována přizpůsobená konfigurace těchto titulků.

\item A~nakonec vytvoří skupinu nových příkazů specifických pro tento nový typ objektu, 
jehož jméno je ve skutečnosti synonymem pro obecnější příkaz.
  
\stopitemize 

V~\in{table}[tbl:floatcommands] můžeme vidět příkazy, které se automaticky vytvoří,
když definujeme nový plovoucí objekt, a~také obecnější příkazy, jejichž synonyma jsou:

\placetable
   [tady]
   [tbl:floatcommands]
   {Příkazy, které se automaticky vytvářejí při vytvoření nového plovoucího objektu}
{\switchtobodyfont[malé]
\starttabulate[|lT|lT|lT|]
\HL
\NC{\bf\rm příkaz}
\NC{\bf\rm Synonymum}
\NC{\bf\rm příklad}
\NR
\HL
\NC\backslash úplný seznam<PluralName>
\NC\backslash completelist[PluralName]
\NC\backslash úplný seznam čísel
\NR
\NC\backslash <JednotnéJméno>
\NC\backslash placefloat[Jednotné jméno]
\NC\backslash obrázku
\NR
\NC\backslash seznam<Množné jméno>
\NC\backslash[PluralName]
\NC\backslash seznam obrázků
\NR
\NC\backslash nastavení???<JednotnéJméno>
\NC\backslash setupfloat[SingulárníNázev]
\NC\backslash nastavení
\NR
\HL
\stoptabulate
}

\startSmallPrint

   Ve skutečnosti jsou vytvořeny některé další příkazy, které jsou synonymem k~předchozím, 
a~protože jsem je nezahrnul do vysvětlení kapitoly, vynechal jsem je z~\in{table}[tbl:floatcommands]: \tex{start<JménoSingulární >}, 
\tex{start<JménoSingulární>text} a~\tex{startplace<JménoSingulární>}.

\stopSmallPrint

Příkaz používaný pro obrázky jsem použil jako příklad příkazů vytvořených při definování 
nového plovoucího objektu; a~udělal jsem to, protože obrázky, jako tabulky a~zbytek floatů 
předdefinovaných pomocí \ConTeXt{}u, jsou skutečnými případy \tex{definefloat}:

\starttyping
\definefloat[chemical][chemicals]
\definefloat[figure][figures]
\definefloat[table][tables]
\definefloat[intermezzo][intermezzi]
\definefloat[graphic][graphics]
\stoptyping

Konečně vidíme, že ve skutečnosti \ConTeXt\ žádným způsobem nekontroluje žádný druh materiálu 
obsaženého v~každém konkrétním plovoucím objektu; předpokládá, že jde o~práci autora. To je důvod, 
proč můžeme také vkládat text pomocí příkazů \tex{placefigure} nebo \tex{placetable}. Avšak text, 
který je vložen pomocí \tex{placefigure}, je zahrnut v~seznamu obrázků, a~pokud je vložen pomocí 
\tex{placetable} v~seznamu tabulek.

\stopsection

\stopchapter

\stopcomponent

\endinput

