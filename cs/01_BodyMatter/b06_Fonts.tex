%%% File:     b06_Fonts.mkiv
%%% Author:   Joaquín Ataz-López
%%% Begun:    March 2020
%%% Concluded: June 2020
%%% Title:  This chapter was the first chapter I began. I came back to
%%%         it often because I wanted it to include the use of
%%%         fonts installed on our system. But I didn't
%%%         succeed in getting the instructions in the wiki to
%%%         work. Recently (August) I discovered fonttest which
%%%         can be accessed after running mtxrun --script server --auto
%%%         I need to investigate further. This "sees" the fonts
%%%         installed on my system and tells me what code I need
%%%         to use them. But I stll don't understand how it
%%%         works. The key might possibly lie in the "Fonts
%%%         out of ConTeXt" manual, but i struggle to understand it.
%%%
%%% Edited with: Emacs + AuTeX - And at times vim + context-plugin
%%%

\environment ../introCTX_env.mkiv

\startcomponent b06_Fonts.mkiv

\startchapter
  [
    reference=sec:fontscol,
    title=Fonts and colours in\\ \ConTeXt,
    bookmark=Fonts and colours in ConTeXt,
  ]

\TocChap

\startsection
  [title=Typografická písma zahrnutá v\\ \suite-]

Fonty \ConTeXt{}u\ nabízí mnoho možností, ale také poměrně dost
složitých. Nebudu se zabývat všemi pokročilými možnostmi písma.
v~této příručce, ale omezím se na předpoklad, že budeme pracovat s~některými
z~21 písem, která jsou součástí instalace \ConTeXt\ Standalone,
těmi, které jsou uvedeny v~\in{tabulka}[tbl:ctx-fonts].


{\switchtobodyfont[small]
  \placetable
    [here]
    [tbl:ctx-fonts]
    {Fonts included in the \ConTeXt\ distribution}
    \starttabulate[|l|l|w{2cm}|]
      \HL
      \NC{\bf Oficialní jméno}\NC{\bf Jména v~\ConTeXt{}u}\NC{\bf Příklad}\NR
      \HL
      \NC Latin Modern \NC modern, modern-base\NC{\switchtobodyfont[modern] Emily Brontë's book}\NR
      \NC Antykwa Poltawskiego\NC antykwapoltawskiego\NC{\switchtobodyfont[antykwapoltawskiego] Emily Brontë's book}\NR
      \NC Antykwa Toruńska\NC antykwa\NC{\switchtobodyfont[antykwa] Emily Brontë's book}\NR
      \NC Cambria \NC cambria\NC{\switchtobodyfont[cambria] Emily Brontë's book}\NR
      \NC DejaVu\NC dejavu\NC{\switchtobodyfont[dejavu] Emily Brontë's book}\NR
      \NC DejaVu Condensed\NC dejavu-condensed\NC{\switchtobodyfont[dejavu-condensed] Emily Brontë's book}\NR
      \NC Gentium \NC gentium\NC{\switchtobodyfont[gentium] Emily Brontë's book}\NR
      \NC Iwona\NC iwona\NC{\switchtobodyfont[iwona] Emily Brontë's book}\NR
      \NC Latin Modern Variable\NC modernvariable, modern-variable\NC{\switchtobodyfont[modernvariable] Emily Brontë's book}\NR
      \NC PostScript\NC postscript\NC  {\switchtobodyfont[postscript] Emily Brontë's book}\NR
      \NC TeX Gyre Adventor\NC adventor, avantgarde\NC{\switchtobodyfont[adventor] Emily Brontë's book}\NR
      \NC TeX Gyre Bonum\NC bonum, bookman\NC{\switchtobodyfont[bonum] Emily Brontë's book}\NR
      \NC TeX Gyre Cursor\NC cursor, courier\NC{\switchtobodyfont[cursor] Emily Brontë's book}\NR
      \NC TeX Gyre Heros\NC heros, helvetica\NC{\switchtobodyfont[heros] Emily Brontë's book}\NR
      \NC TeX Gyre Schola\NC schola, schoolbook\NC{\switchtobodyfont[schola] Emily Brontë's book}\NR
      \NC Tex Gyre Chorus\NC chorus, chancery\NC{\switchtobodyfont[chorus] Emily Brontë's book}\NR
      \NC Tex Gyre Pagella\NC pagella, palatino\NC{\switchtobodyfont[pagella] Emily Brontë's book}\NR
      \NC Tex Gyre Termes\NC termes, times\NC{\switchtobodyfont[termes] Emily Brontë's book}\NR
      \NC Euler\NC eulernova\NC{\switchtobodyfont[eulernova] Emily Brontë's book}\NR
      \NC Stix2\NC stix\NC{\switchtobodyfont[stix] Emily Brontë's book}\NR
      \NC Xits\NC xits\NC{\switchtobodyfont[xits,8pt] Emily Brontë's book}\NR
      \HL
    \stoptabulate
}

Prostřední sloupec \in{tabulka}[tbl:ctx-fonts] označuje název nebo
jména, pod kterými \ConTeXt\ zná dané písmo. Pokud existují dvě
jména, jedná se o~synonyma. V~posledním sloupci je uveden příklad písma
který se používá. Co se týče pořadí, v~jakém jsou písma zobrazena, první z~nich je
písmo, které \ConTeXt\ používá ve výchozím nastavení, a~zbývající písma jsou v~pořadí
abecedním, zatímco poslední tři písma jsou speciálně navržena
pro matematiku. Všimněte si, že Eulerovo písmo nemůže přímo reprezentovat
písmena s~diakritikou, takže dostaneme Bront's, nikoli Brontë's.


Pro čtenáře, kteří přicházejí ze světa Windows a~jeho výchozích písem, 
uvedu, že {\em heros} je v~systému Windows ekvivalentem Arialu, zatímco {\em
termes} je stejné jako Times New Roman. Nejsou úplně stejná
ale jsou si natolik podobné, že by bylo třeba být velmi 
všímavý, aby je člověk rozeznal.

  \startSmallPrint


 Písma používaná systémem Windows nejsou {\em volný software} (ve skutečnosti téměř
    nic ve Windows není {\em volný software}), takže je nelze
    zahrnout do \ConTeXt{}u. Pokud je však \ConTeXt\ nainstalován na
    Windows, pak jsou tato písma již nainstalována a~mohou
    být použita stejně, jako jakákoli jiná písma nainstalovaná v~systému se spuštěným
   \ConTeXt{}em. V~tomto úvodu se však nebudu zabývat tím, jak 
    používat písma již nainstalovaná v~systému. Nápovědu lze nalézt
    na \goto{\ConTeXt\ wiki}[url(wiki)].

    
  \stopSmallPrint

\stopsection

\startsection
  [title=Vlastnosti písma]

\startsubsection
  [title={Písma, {\em styly} a~varianty stylů}]


Terminologie týkající se fontů je poněkud matoucí, protože někdy
to, co se nazývá písmo, je ve skutečnosti rodina písem, která zahrnuje
různé styly a~varianty, které mají společný základní design. Nebudu se zabývat 
otázkou, která terminologie je správnější; chci
pouze objasnit terminologii používanou v~\ConTeXt{}u. Tam,
se rozlišuje mezi písmy, styly a~variantami (resp.
alternativami) pro každý styl. Písma {\em} obsažená v~\ConTeXt{}u
(ve skutečnosti se jedná o~rodiny písem) jsou ty, které jsme viděli
v~předchozí části. Nyní se podíváme na {\em styly} a~{\em
alternativy}.

\subsubsubject{Styly písma}

\dontleavehmode{\sc Donald E. Knuth} navrhl {\em Computer Modern}
písmo pro \TeX a dal mu tři různé {\em styly} zvané {\em roman},
{\em sans serif} a~{\em teletype}. Styl {\em roman} je designem
kde jsou znaky ozdobně rozvolněné, známé v~typologickém designu,
proto je tento styl písma známý také jako {\em serif}.
Tento styl byl považován za {\em normalní} nebo
výchozí styl. Styl {\em sans serif}, jak naznačuje jeho název, postrádá ozdobnost
a~je tedy jednodušší, stylizovanější, někdy se používá i~jako font.
známý pod jinými názvy, např. ve španělštině {\em paloseco}; toto písmo může být
hlavním písmem v~dokumentu, ale je také vhodné pro použití
v~některých částech textu, kterých hlavním písmem je {\em roman}.
jako například nadpisy nebo záhlaví stránek. A~konečně styl {\em
teletype} byl zařazen do stylu {\em Computer Roman}, protože
byl navržen pro psaní knih týkajících se počítačového programování, které zahrnují
rozsáhlé části v~počítačovém {\em kódu}, který je konvenčně
v~tištěných materiálech zobrazován monospace stylem, který napodobuje
počítačových terminálů a~starých psacích strojů.

  \startSmallPrint



Čtvrtý styl určený pro matematické fragmenty, by mohl být přidán k~těmto tří 
 {\em stylům}. Ale protože \TeX\ automaticky používá tento
    styl, když přejde do matematického režimu, a~neobsahuje příkazy pro jeho povolení nebo zakázání, ani nemá {\em varianty}
    nebo alternativy ostatních stylů, není obvyklé o~něm přemýšlet jako o{\em stylu}.


\ConTeXt\ obsahuje příkazy pro dva další možné styly:
  ručně psaný a~kaligrafický. Nejsem si úplně jistý, rozdílem
  mezi nimi, protože na jedné straně žádný z~fontů
  obsažen \ConTeXt{}u tyto styly neobsahuje.
  a~na druhou stranu, jak vidím, kaligrafické písmo je v~tomto případě také spíše
  ručně psané písmo. Tyto příkazy, které \ConTeXt\ obsahuje, pro umožnění
  těchto stylů, pokud budou použity s~písmem, které je neimplementuje,
  nezpůsobí žádnou chybu při kompilaci: prostě se nic nestane.
  
  \stopSmallPrint

\subsubsubject{Alternativní formy písma}

Každý {\em styl} povoluje učitý počet alternativních forem, proto je 
\ConTeXt\ nazývá, ({\em alternativy}):

\startitemize[2,packed]

\item Regular or normal (\MyKey{tf}, z~{\em typeface}).
\item Bold (\MyKey{bf}, z~{\em boldface}).
\item Italic (\MyKey{it} z~{\em italic})
\item BoldItalic (\MyKey{bi} z~{\em bold italic})
\item Slanted (\MyKey{sl} z~{\em slanted})
\item BoldSlanted (\MyKey{bs} z~{\em bold slanted})
\item Small caps (\MyKey{sc} z~{\em small caps})
\item Medieval (\MyKey{os} z~{\em old style})
  
\stopitemize


Tyto {\em alternativy}, jak už jejich název napovídá, se vzájemně ovlivňují: 
když je jedna z~nich povolena, ostatní jsou zakázány. To je důvod, proč
\ConTeXt\ poskytuje příkazy pro jejich povolení, ale ne pro zakázání.
Protože když povolíme alternativu, zakážeme tu, kterou jsme předtím povolovali.
a~tak například, pokud píšeme kurzívou
a~povolíme tučné písmo, bude kurzíva zakázána. Pokud chceme používat tučné písmo
a~kurzívu současně, nemusíme povolit jedno a~pak druhé,
ale spíše povolit alternativu, která zahrnuje obojí (\MyKey{bi}).


Na druhou stranu je třeba mít na paměti, že ačkoli \ConTeXt\ 
předpokládá, že každé písmo bude mít tyto alternativy, a~tudíž
poskytuje příkazy, které umožňují, aby fungovaly a~produkovaly nějaké
vnímatelné efekty ve výsledném dokumentu, potřebují tyto příkazy, aby písmo.
mělo ve svém návrhu specifické formy pro každý styl a~alternativu.

\startSmallPrint

Zejména mnoho písem nerozlišuje ve svém designu mezi.
  šikmými a~kurzivními písmeny, nebo neobsahují speciální tvary pro malé kapitálky. 

\stopSmallPrint

\subsubsubject{Rozdíl mezi kurzívou a~šikmými písmeny}


Podobnost typografické funkce kurzívy a~šikmého písma
vede mnoho lidí k~záměně těchto dvou alternativ. 
Šikmé písmeno vzniká mírným pootočením pravidelného tvaru. Ale
kurzíva znamená - alespoň v~některých písmech - odlišný design ve tvaru písmene.
kde se {\em zdá}, že písmena jsou nakloněná, protože byla nakreslena
tak, aby to tak vypadalo, ale ve skutečnosti se nejedná o~skutečný náklon. To může být
v~následujícím příkladu, ve kterém jsme napsali stejné slovo
třikrát ve stejné velikosti, aby bylo snadno rozpoznatelné.
rozdíly. V~první verzi je použita základná podoba,
v~druhém šikmá a~ve třetím kurzíva:

\midaligned{\switchtobodyfont[20pt,rm] %
  \framed{{\rm italics} -- {\sl italics} -- {\it italics}} }%


Všimněte si, že design znaků je v~prvních dvou dílech stejný.
ale ve třetím příkladu jsou jemné rozdíly v~tazích
některých písmen, což je velmi zřejmé, zejména v~tom, jak vypadá \quote{a}.
ačkoli rozdíly se ve skutečnosti vyskytují téměř ve všech případech.
znaků.


Obvyklá použití kurzívy a~šikmých písmen jsou podobná a~každá osoba
se rozhodne, zda použije jedno nebo druhé. Zde je volnost, i~když
měli bychom zdůraznit, že dokument bude lépe napsán a~bude vypadat
lépe, pokud je použití kurzívního a~šikmého písma {\em konzistentní}.
V~mnoha písmech je navíc rozdíl v~designu mezi kurzivou a~šikmými písmy
zanedbatelný, takže je jedno, zda použijeme jedno nebo druhé.
nebo druhou.

Na druhou stranu, jak kurzíva, tak šikmé písmo jsou alternativami písma, které
znamená především dvě věci:

\startitemize[n]

\item Můžeme je použít pouze tehdy, jsou-li definovány v~písmu.

\item Povolením jedné z~nich deaktivujeme alternativu, která
  která byla do té doby používána.
  
\stopitemize

Spolu s~příkazy pro kurzívu a~šikmé písmo nabízí \ConTeXt\ 
další příkazy pro {\em zdůraznění} určitého textu. Jeho použití
znamená jemné rozdíly ve srovnání s~kurzivou nebo šikmým písmem. Viz
\in{sekce}[sek:zvýraznění].

\stopsubsection

\startsubsection
  [title=Velikost písma]

Všechna písma, se kterými \ConTeXt pracuje, jsou založena na vektorové grafice.
takže teoreticky mohou být zobrazena v~libovolné velikosti písma, ačkoli, jak
uvidíme, záleží to na skutečných instrukcích, které používáme k~určení
velikosti písma. Pokud není uvedeno jinak, předpokládá se, že velikost písma
bude 12 bodů.

\startSmallPrint

Všechna písma používaná v~\ConTeXt{}u jsou založena na vektorové grafice
a~Proto jsou to písma Opentype nebo Type 1, což znamená, že písma, jenž
 vznikly před touto technologií, byly implementovány znovu. Konkrétně se jedná o,
 výchozí písmo \TeX{}u, {\em Computer Modern}, které navrhl {\sc Knuth},
 existovalo pouze v~určitých velikostech, a~proto bylo znovu implementováno v~designu nazvaném
 \emph{Latin Modern}, který používá \ConTeXt, ačkoli v~mnoha dokumentech
 se nadále nazývá {\em Computer Modern} kvůli silné
 symbolice, kterou toto písmo stále má v~systému \TeX{}, odkdy
 začaly a~byly vyvinuty {\sc Knuthem} spolu s~dalším programem nazvaným
 \MetaFont{}, jehož cílem bylo navrhnout písma, která by mohla pracovat s~\TeX{}.

\stopSmallPrint

\stopsubsection

\stopsection

\startsection
  [
    reference=sec:mainfont,
    title=Nastavení hlavního písma dokumentu,
  ]
\PlaceMacro{setupbodyfont}

Ve výchozím nastavení, pokud není uvedeno jiné písmo, \ConTeXt{} použije
{\em Latin Modern Roman} o~velikosti 12 bodů jako hlavní písmo. Toto písmo bylo
původně navrhl {\sc Knuth} pro implementaci v~\TeX{}u. Je to
elegantní písmo římského stylu s~velkou proporční a~dekorativní 
\quotation{kresbou} - nazývanými {\em serifs} - v~některých tazích,
které je velmi vhodné jak pro tištěné texty, tak pro zobrazení na
obrazovce; ačkoli -- a~to je osobní názor -- není tak
vhodné pro malé obrazovky, jako je {\em smartphone}, protože {\em
serify} nebo ozdoby mají tendenci se hromadit, což ztěžuje čtení.

Pro nastavení jiného písma použijeme \tex{setupbodyfont}, který nám umožní ne
jen změnit skutečné písmo, ale také jeho velikost a~styl. Když
chceme, aby to platilo pro celý dokument, musíme to zahrnout do položky
preambuli zdrojového souboru. Pokud však chceme jednoduše změnit písmo v~místě
v~určitém bodě, musíme do něj zahrnout to, co následuje.

Formát \tex{setupbodyfont} je:

\type{\setupbodyfont[Options]}

kde různé možnosti příkazu umožňují uvést:

\startitemize

\item {\bf Jméno písma}, což může být kterýkoli ze symbolických názvů písem
nalezené v~\in{tabulce}[tbl:ctx-fonts].

\item {\bf Velikost}, která může být označena buď rozměry
(s~použitím bodu jako měrné jednotky) nebo určitými symbolickými
názvy. Všimněte si však, že i~když jsem dříve uvedl, že písma používaná
v~\ConTeXt{}u lze škálovat prakticky na libovolnou velikost, v~\tex{setupbodyfont} 
se používají pouze velikosti skládající se z~celých čísel mezi 4 a~12, stejně jako velikosti
14,4 a~17,3, jsou v~\ConTeXt{}u podporovány. Ve výchozím nastavení se předpokládá
velikost 12 bodů.


   \tex{setupbodyfont}, vytváří něco, co bychom mohli nazvat {\em základní
  velikost} dokumentu; jinými slovy {\em normální} velikost znaků.
  na jejímž základě se vypočítávají další velikosti, například nadpisy
  a~poznámky pod čarou. Když změníme hlavní velikost pomocí \tex{setupbodyfont}.
  všechny ostatní velikosti vypočtené na základě hlavního písma se také změní.
  změní.

Kromě přímo uvedené velikosti znaků (10pt, 11pt, 12pt atd.)
můžeme také použít některé symbolické názvy, které vypočítají velikost znaku pro
použítí na základě aktuální velikosti. Jedná se o~tyto symbolické názvy,
od největšího po nejmenší: big, small, script, x, scriptscript a~xx.
Pokud tedy například chceme nastavit text těla pomocí \tex{setupbodyfont}.
který je větší než 12 bodů, můžeme tak učinit pomocí \MyKey{big}.

\item {\bf Styl písma}, který, jak jsme již uvedli, může být roman
(se serify), nebo bez serifů (san serif), nebo styl psacího stroje
a~u~některých písem ručně psaný a~kaligrafický styl. \tex{setupbodyfont}
umožňuje různé symbolické názvy pro označení různých stylů. Jsou
v~tabulce \in[tab:ctx-stylesstbdf]:

  {\switchtobodyfont[script] %
    \placetable %
      [here, force] %
      [tab:ctx-stylesstbdf] %
      {Styles in setupbodyfont} %
      {\startxtable %
        \startxrow [topframe=on, bottomframe=on]%
          \startxcell \bf Styl \stopxcell %
          \startxcell \bf Symbolické jména povoleny\stopxcell %
        \stopxrow %
        \startxrow %
          \startxcell Roman \stopxcell %
          \startxcell \tt rm, roman, serif, regular \stopxcell %
        \stopxrow %
        \startxrow %
          \startxcell Sans Serif \stopxcell %
          \startxcell \tt ss, sans, support, sansserif\stopxcell %
        \stopxrow %
        \startxrow %
          \startxcell Styl psacího stroje \stopxcell %
          \startxcell \tt tt, modo, type, teletype \stopxcell %
        \stopxrow %
        \startxrow %
          \startxcell Ručne psaný \stopxcell %
          \startxcell \tt hw, handwritten \stopxcell %
        \stopxrow %
        \startxrow [bottomframe=on]%
          \startxcell Kaligrafický \stopxcell %
          \startxcell \tt cg, calligraphic\stopxcell %
        \stopxrow %
      \stopxtable} %
  }


Pokud mohu posoudit, různé názvy podporované pro jednotlivé
 styly jsou zcela synonymní.

\stopitemize

\startsubsubject
  [
    reference=sec:see-font,
    title=Náhled, jak písmo vypadá
  ]

Before deciding to use a~particular font in our document, we would
normally want to see what it looks like. This can almost always be done
from the operating system as there is usually some utility to examine
the appearance of the fonts installed on the system; but for
convenience, \ConTeXt\ itself offers a~utility that allows us to see
the appearance of any of the fonts enabled in \ConTeXt\. This is
\tex{showbodyfont}, that generates a~table with examples of the font we
indicate.

Než se rozhodneme použít v~dokumentu určité písmo, obvykle bychom se 
 chtěli podívat, jak vypadá. To lze téměř vždy provést
z~operačního systému, protože obvykle existuje nějaká utilita, která umožňuje prozkoumat
vzhled písem nainstalovaných v~systému; ale v~případě
\ConTeXt{}u se nabízí nástroj, který nám umožňuje vidět
vzhled kteréhokoli z~písem povolených v~\ConTeXt{}u. Jedná se
o~\tex{showbodyfont}, který vygeneruje tabulku s~příklady písma, které jsme
označili.

Formát \tex{showbodyfont} je nasledující:

\type{\showbodyfont [Options]}

where we can indicate as options precisely the same symbolic names as in
\tex{setupbodyfont}. So, for example, \tex{showbodyfont[schola, 8pt]}
will show us the table below, in which there are different examples of
the schola font at a~base size of 8 points:

\showbodyfont[schola,8pt]\bigskip


Všimněte si, že v~prvním řádku a~sloupci jsou určité příkazy.
tabulky. Dále, když byl význam těchto příkazů
vysvětlen, podíváme se ještě jednou do tabulky generované příkazem
\tex{showbodyfont}.

Pokud chceme zobrazit kompletní rozsah znaků v~určitém písmu,
můžeme použít příkaz \PlaceMacro{showfont}\tex{showfont[FontName]}.
Tento příkaz zobrazí hlavní konstrukci každého ze znaků
bez použití příkazů pro styly a~alternativy.


\stopsubsubject

\stopsection

\startsection
  [title=Změna písma nebo některých funkcí písma]

\startsubsection
  [title=Příkazy \tex{setupbodyfont} a~\tex{switchtobodyfont}]
\PlaceMacro{switchtobodyfont}\PlaceMacro{setupbodyfont}

Pro změnu písma, stylu nebo velikosti můžeme použít stejný příkaz, kterým jsme
stanovili písmo na začátku dokumentu, když 
nechceme použít výchozí písmo \ConTeXt{}u: \tex{setupbodyfont}. Vše, co potřebujeme,
je umístit tento příkaz na místo v~dokumentu, kde si přejeme
změnit písmo. Tím dojde k~{\em trvalé} změně písma,
což znamená, že přímo ovlivní hlavní písmo a~nepřímo všechny ostatní
písma, která s~ním souvisejí.

Velmi podobné \tex{setupbodyfont} je \tex{switchtobodyfont}. Oba
nám umožňují měnit stejné aspekty písma (písmo
samotný styl a~velikost), ale vnitřně fungují odlišně
a~jsou určeny pro různá použití. První z~nich (\tex{setupbodyfont}) je
určen pro {\em nastavení} hlavního (a~obvykle jediného) písma.
v~dokumentu; není to ani běžné, ani typograficky správné mít
v~dokumentu více než jedno hlavní písmo (proto se nazývá {\em
hlavní} písmo). Naproti tomu \tex{switchtobodyfont} je určen pro psaní.
některé části textu jiným písmem nebo k~přiřazení určitého písma k~jinému písmu.
speciálnímu druhu odstavce, který chceme v~dokumentu definovat.

Apart from the above -- which actually affects the internal functioning
of each of these two commands -- from the user's point of view there are
some differences between the use of one or the other command. In
particular:

Kromě výše uvedeného - který skutečně ovlivňuje vnitřní fungování
každého z~těchto dvou příkazů -- z~pohledu uživatele existují
některé rozdíly mezi použitím jednoho nebo druhého příkazu.
Zejména:

\startitemize[n]

\item Jak již víme, \tex{setupbodyfont} je omezen na konkrétní
rozsah velikostí, zatímco \tex{switchtobodyfont} nám umožňuje označit
prakticky libovolnou velikost, takže pokud písmo v~dané velikosti není k~dispozici,
bude na ni škálováno.

\item \tex{switchtobodyfont} nijak neovlivňuje textové prvky
kromě místa, kde je použito, na rozdíl od \tex{setupbodyfont}, které, jak
je uvedeno výše, nastavuje hlavní písmo a~jeho změnou také
změní písmo všech textových prvků, jejichž písmo je vypočteno 
na základě hlavního písma.


\stopitemize

Na druhou stranu oba příkazy mění nejen písmo, ale i~styl
a~také další aspekty spojené s~písmem, jako 
například mezery mezi řádky.

\startSmallPrint

\tex{setupbodyfont} generuje chybu při kompilaci, pokud je použito nepovolené písmo
  ale nevyvolá ji, pokud je požadována neexistující velikost písma.
  v~takovém případě je výchozí písmo ({\em Latin Modern Roman}).
  bude povoleno. \tex{switchtobodyfont} se chová stejně, pokud jde
o~velikost, jak jsem již řekl, snaží se toho
  dosáhnout zmenšením písma. Existují však písma, která nelze
  škálovat na určité velikosti, v~takovém případě by výchozí písmo bylo
  opět povoleno.

\stopSmallPrint

\stopsubsection

\startsubsection
  [
    reference=sec:quick-change,
    title={Rychlá změna stylu, alternativy a~velikosti},
  ]

\subsubsubject{Změna stylu a~alternativy}

Kromě příkazu \tex{switchtobodyfont} poskytuje \ConTeXt\ sadu příkazů.
které nám umožňují rychle změnit styl, alternativu nebo velikost. S~pomocí
\ConTeXt\ wiki nás upozorňuje, že někdy,
když se objeví na začátku odstavce, mohou způsobit některé
nežádoucí vedlejší účinky, takže doporučuje, aby v~takových případech příkaz
předcházel příkaz
\PlaceMacro{dontleavehmode}\tex{dontleavehmode}.

{\switchtobodyfont[small] %
\placetable[here][tab:ctx-styles]
  {Příkazy pro změnu mezi jednotlivými styly}
  {
    \starttabulate[|l|l|]
      \HL
      \NC {\bf Styl} \NC {\bf Příkazy, které to umožňují}\NR
      \HL
      \NC Roman\NC\type{\rm}, \type{\roman}, \type{\serif}, \type{\regular}\NR\PlaceMacro{rm}\PlaceMacro{roman}\PlaceMacro{serif}\PlaceMacro{regular}
      \NC Sans Serif\NC\type{\ss}, \type{\sans}, \type{\support}, \type{\sansserif}\NR\PlaceMacro{ss}\PlaceMacro{sans}\PlaceMacro{support}\PlaceMacro{sansserif}
      \NC Monospaced\NC\type{\tt}, \type{\mono}, \type{\teletype},\NR\PlaceMacro{tt}\PlaceMacro{mono}\PlaceMacro{teletype}
      \NC Handwritten\NC\type{\hw}, \type{\handwritten},\NR\PlaceMacro{hw}\PlaceMacro{handwritten}
      \NC Calligraphic\NC\type{\cf}, \type{\calligraphic}\PlaceMacro{cf}\PlaceMacro{calligraphic}\NR
      \HL
    \stoptabulate
  }}


\in{Tabulka}[tab:ctx-styles] obsahuje příkazy, které nám umožňují měnit
styl, aniž by se změnil jakýkoli jiný aspekt;
a~\in{tabulka}[tab:ctx-alternatives] obsahuje příkazy, které nám umožňují
měnit výhradně alternativu.

{\switchtobodyfont[small] %
  \placetable[here][tab:ctx-alternatives]
  {Příkazy pro povolení konkretní alternativy}
  {
    \starttabulate[|l|l|]
      \HL
      \NC {\bf Alternative} \NC {\bf Commands that enable it}\NR
      \HL
      \NC Normal           \NC\type{\tf}, \type{\normal}\NR\PlaceMacro{tf}\PlaceMacro{normal}
      \NC Italic          \NC\type{\it}, \type{\italic}\NR\PlaceMacro{it}\PlaceMacro{italic}
      \NC Bold          \NC\type{\bf}, \type{\bold}\NR\PlaceMacro{bf}\PlaceMacro{bold}
      \NC Bold-italic  \NC\type{\bi}, \type{\bolditalic}, \type{\italicbold}\NR\PlaceMacro{bi}\PlaceMacro{bolditalic}\PlaceMacro{italicbold}
      \NC Slanted        \NC\type{\sl}, \type{\slanted}\NR\PlaceMacro{sl}\PlaceMacro{slanted}
      \NC Bold-slanted\NC\type{\bs}, \type{\boldslanted}, \type{\slantedbold}\NR\PlaceMacro{bs}\PlaceMacro{boldslanted}\PlaceMacro{slantedbold}
      \NC Small caps       \NC\type{\sc}, \type{\smallcaps}\NR\PlaceMacro{sc}\PlaceMacro{smalcaps}
      \NC Medieval         \NC\type{\os}, \type{\mediaeval}\PlaceMacro{os}\PlaceMacro{mediaeval}\NR
    \stoptabulate
  }
}

Všechny tyto příkazy si zachovávají svou účinnost, dokud není použit jiný styl nebo
alternativa výslovně povolena, nebo {\em skupina}, v~jejímž rámci
deklarován příkaz, skončí. Pokud tedy chceme, aby příkaz ovlivnil
pouze část textu, musíme tuto část textu uzavřít do příkazu
skupiny, jako v~následujícím příkladu, kde pokaždé,když se objeví slovo {\em
myšlenka}, pokud se jedná o~podstatné jméno, nikoli o~sloveso, je uvedeno kurzívou,
se vytvoří pro něj skupina.

\startDoubleExample

\starttyping
Myslel jsem si {\it myšlenku} ale 
{\it myšlenka}, kterou jsem si myslel nebyla
{\it myšlenka}, kterou jsem si myslel, že jsem si myslel.
Kdyby {\it myšlenka} kterou jsem si myslel, že jsem si myslel
byla {\it myšlenka}, kterou jsem si myslel
Nemusel bych myslet tak moc!
\stoptyping

Myslel jsem si {\it myšlenku} ale 
{\it myšlenka}, kterou jsem si myslel nebyla
{\it myšlenka}, kterou jsem si myslel, že jsem si myslel.
Kdyby {\it myšlenka} kterou jsem si myslel, že jsem si myslel
byla {\it myšlenka}, kterou jsem si myslel
Nemusel bych myslet tak moc!

\stopDoubleExample

\subsubsubject[sec:sufijos de tamaño]{Přípony pro změnu alternativ a~velikosti naráz}
\PlaceMacro{rmx}\PlaceMacro{rmxx}\PlaceMacro{rma}\PlaceMacro{rmb}\PlaceMacro{rmc}\PlaceMacro{rmd}
\PlaceMacro{ssx}\PlaceMacro{ssxx}\PlaceMacro{ssa}\PlaceMacro{ssb}\PlaceMacro{ssc}\PlaceMacro{ssd}
\PlaceMacro{ttx}\PlaceMacro{ttxx}\PlaceMacro{tta}\PlaceMacro{ttb}\PlaceMacro{ttc}\PlaceMacro{ttd}
\PlaceMacro{tfx}\PlaceMacro{tfxx}\PlaceMacro{tfa}\PlaceMacro{tfb}\PlaceMacro{tfc}\PlaceMacro{tfd}
\PlaceMacro{itx}\PlaceMacro{itxx}\PlaceMacro{ita}\PlaceMacro{itb}\PlaceMacro{itc}\PlaceMacro{itd}
\PlaceMacro{bfx}\PlaceMacro{bfxx}\PlaceMacro{bfa}\PlaceMacro{bfb}\PlaceMacro{bfc}\PlaceMacro{bfd}
\PlaceMacro{bix}\PlaceMacro{bixx}\PlaceMacro{bia}\PlaceMacro{bib}\PlaceMacro{bic}\PlaceMacro{bid}
\PlaceMacro{slx}\PlaceMacro{slxx}\PlaceMacro{sla}\PlaceMacro{slb}\PlaceMacro{slc}\PlaceMacro{sld}
\PlaceMacro{bsx}\PlaceMacro{bsxx}\PlaceMacro{bsa}\PlaceMacro{bsb}\PlaceMacro{bsc}\PlaceMacro{bsd}

Příkazy, které mění styl nebo alternativu ve své dvoupísmenné verzi (\tex{tf}, \tex{it}, \tex{bf} atd.), umožňují řadu  {\em přípon}, které ovlivňují velikost písma. Přípony a, b, c a~d zvětšují velikost písma a~násobí ji o~$1,2$, $1,2^2$ ($=1,44$), $1,2^3$ ($=1,728$) nebo $1,2^4$ ($=2,42$). Viz příklad:

\type{\tf test, \tfa test, \tfb test, \tfc test, \tfd test}

{\color[red]{\tf test, \tfa test, \tfb test, \tfc test, \tfd test}}

přípony x a~xx zmenšují velikost písma a~násobí ji čísly 0,8 a~0,6.
v~tomto pořadí:

\type{\tf test, \tfx test, \tfxx test}

{\color[red]{\tf test, \tfx test, \tfxx test}}

Přípony \quote{x} a~\quote{xx} aplikované na \tex{tf} nám umožňují
zkrátit příkaz, takže \tex{tfx} lze zapsat jako
\PlaceMacro{tx}\tex{tx} a~\tex{tfxx} jako \PlaceMacro{txx}\tex{txx}.


Dostupnost těchto různých přípon závisí na aktuálním stavu.
implementaci písma. Podle reference příručky \ConTeXt{}u 2013
(určené především pro Mark~II) jediná přípona zaručená pro to
vždy fungovat, je \quote{x} a~ostatní mohou, ale nemusí být
implementovány; nebo mohou být jen pro některé alternativy.

Každopádně, abychom předešli pochybnostem, můžeme použít \tex{showbodyfont}, o~kterém jsem mluvil.
(v~\in{sekci}[sec:see-font]). Tento příkaz zobrazí
graf, který nám umožní nejen ocenit vzhled písma,
ale také vidět, jak písmo vypadá v~každém ze svých stylů
a~alternativách a~také jaké přípony pro změnu velikosti jsou k~dispozici.

Podívejme se ještě jednou na tabulku zobrazující \tex{showbodyfont}:

\showbodyfont[modern]


Pokud se na tabulku podíváme pozorněji, zjistíme, že první sloupec
obsahuje styly písma (\tex{rm}, \tex{ss} a~\tex{tt}). První
řádek obsahuje vlevo alternativy (\tex{tf}, \tex{sc},
\tex{sl}, \tex{it}, \tex{bf}, \tex{bs} a~\tex{bi}), zatímco vpravo
první řádek obsahuje ostatní dostupné přípony, ačkoli na druhé straně jsou přípony
pouze s~pravidelnou alternativou.

Je důležité si uvědomit, že změna velikosti písma provedená některým z~těchto typů
přípony změní pouze velikost písma v~užším slova smyslu, přičemž zůstane zachována
ostatní hodnoty obvykle spojené s~velikostí písma, jako je velikost řádku, zůstanou nedotčeny.
  
\subsubsubject{Přizpůsobení měřítka přípon}
  
To customise the scaling factor we can use
\PlaceMacro{definebodyfontenvironment}\tex{definebodyfontenvironment}
whose format can be:

Pro přizpůsobení měřítkového faktoru můžeme použít
\PlaceMacro{definebodyfontenvironment}\tex{definebodyfontenvironment}
jehož formát může být:

\starttyping
\definebodyfontenvironment[particular size][scaled]
\definebodyfontenvironment[default][scaled]
\stoptyping

V~první verzi bychom nadefinovali měřítko pro určitou velikost.
hlavního písma nastaveného pomocí \tex{setupbodyfont} nebo pomocí
\tex{switchtobodyfont}. Například:

\type{\definebodyfontenvironment[10pt][a=12pt,b=14pt,c=2, d=3]}


by zajistil, že když je hlavní písmo 10 bodů, přípona \quote{a}
by se změnila na 12 bodů, přípona \quote{b} na 14 bodů, přípona \quote{c}
by původní písmo vynásobila dvěma a~přípona \quote{d} třema. Poznámka
že pro a~a~b je uveden pevný rozměr, ale pro c a~d
je uveden násobek původní velikosti.

Pokud je však první argument \tex{definebodyfontenvironment} roven
\MyKey{default}, pak budeme předefinovávat hodnotu měřítka pro všechny hodnoty
možných velikostí písma a~jako hodnotu škálování můžeme zadat pouze hodnotu
násobícího číslo. Pokud tedy například napíšeme:

\type{\definebodyfontenvironment[default][a=1.3,b=1.6,c=2.5,d=4]}

naznačujeme, že bez ohledu na velikost hlavního písma
a~přípona by měla být vynásobena koeficientem 1,3, b koeficientem 1,6, c koeficientem 2 a~d koeficientem
4.

Stejně jako přípony xx, x, a, b, c a~d,
s~\tex{definebodyfontenvironment} můžeme přiřadit hodnotu škálování k~znakům
\MyKey{big}, \MyKey{small}, \MyKey{script} a~\MyKey{scriptscript} klíčová
slova. Tyto hodnoty jsou přiřazeny všem velikostem přidruženým k~těmto klíčovým
slovům v~\tex{setupbodyfont} a~\tex{switchtobodyfont}. Jsou také
použity v~následujících příkazech, jejichž užitečnost lze odvodit
z~jejich názvu:

\startitemize[1,packed]

\item \PlaceMacro{smallbold}\tex{smallbold}
\item \PlaceMacro{smallslanted}\tex{smallslanted}
\item \PlaceMacro{smallboldslanted}\tex{smallboldslanted}
\item \PlaceMacro{smallslantedbold}\tex{smallslantedbold}
\item \PlaceMacro{smallbolditalic}\tex{smallbolditalic}
\item \PlaceMacro{smallitalicbold}\tex{smallitalicbold}
\item \PlaceMacro{smallbodyfont}\tex{smallbodyfont}
\item \PlaceMacro{bigbodyfont}\tex{bigbodyfont}
  
\stopitemize

Pokud chceme zobrazit výchozí velikosti určitého písma, můžeme použít příkaz
\PlaceMacro{showbodyfontenvironment}\tex{showbodyfontenvironment[Font]}.
Tento příkaz použitý například na písmo {\tt modern} poskytne
následující výsledek:

\showbodyfontenvironment[modern]

\stopsubsection

\startsubsection
  [title={Definování příkazů a~klíčových slov pro velikosti, styly a~alternativy písma}]


Předdefinované příkazy pro změnu velikosti, stylů a~variant písma jsou
dostatečné. \ConTeXt\ nám navíc umožňuje:

\startitemize[n]

\item Přidání vlastního příkazu pro změnu stylu, velikosti nebo varianty písma.

\item Přidání synonym k~názvům stylů nebo variant rozpoznaných pomocí
  \tex{switchtobodyfont}.
  
\stopitemize

K~tomu slouží následující příkazy:

\startitemize

\item \PlaceMacro{definebodyfontswitch}\tex{definebodyfontswitch}:
umožňuje definovat příkaz pro změnu velikosti písma. Například pokud
chceme definovat příkaz \tex{osm} (nebo \tex{viii
příkaz}\footnote{Zapomeňte, že s~výjimkou řídicích symbolů,
\ConTeXt\ názvy příkazů mohou sestávat pouze z~písmen.}) pro nastavení velikosti 8 pt
potřebujeme napsat:

  {\tfx\type{\definebodyfontswitch[eight][8pt]}} nebo
  {\tfx\type{\definebodyfontswitch[viii][8pt]}}

\item \PlaceMacro{definefontstyle}\tex{definefontstyle}: umožňuje nám
definovat jedno nebo více slov, která lze použít v~\tex{setupbodyfont} nebo
\tex{switchtobodyfont} nastavit určitý styl písma; takže například,
kdybychom chtěli nazvat {\em sans serif} něčím jiným (např. v~příkazu
španělštině se nazývá \quotation{paloseco}), můžeme napsat

  \type{\definefontstyle[paloseco][ss]}

  Zvláštností \tex{definefontstyle} je, že umožňuje použít několik slov.
  současně přiřazeno stejnému stylu, takže, abychom mohli pokračovat
  španělským příkladem:

  \type{\definefontstyle[paloseco, sosa, sinrebordes][ss]}

\item \PlaceMacro{definealternativestyle}\tex{definealternativestyle}:
umožňuje přiřadit jméno k~variantě písma. Toto jméno může
fungovat jako příkaz nebo být rozpoznán jako volba {\tt stylu}
příkazů, které nám umožňují nastavit styl, který se má použít. Takže pro
příklad následující fragment

  \type{\definealternativestyle[strong][\bf][]}

  povolí příkaz \tex{strong} a~klíčové slovo \MyKey{strong}.
  které bude rozpoznáno volbou {\tt style} příkazů, které
  tuto volbu umožňují. Mohli jsme říci \quotation{bold}, ale toto slovo
  se již v~ConTeXtu používá, proto jsem zvolil výraz používaný v~HTML,
  a~to \quotation{strong} jako alternativu.

  \startSmallPrint

    Nevím, co znamená třetí argument
    \tex{definealternativestyle}. Není nepovinný, a~proto
    nemůže být \Doubt vynechán; ale jediné informace, které jsem o~něm našel, jsou tyto
    v~referenční příručce \ConTeXt\, kde se o~tomto třetím argumentu říká.
    že je relevantní pouze pro názvy kapitol a~oddílů \quotation{{\em
    kde se kromě \tex{cap} musíme řídit i~zde použitým písmem}}. (??)

  \stopSmallPrint

\stopitemize

\stopsubsection

\stopsection

\startsection
  [title=Další záležitosti týkající se používání některých alternativních řešení]

Mezi různými alternativami písma existují dvě, jejichž použití
vyžaduje určitá upřesnění:

\startsubsection
  [
    reference=sec:emphasis,
    title={Kurzíva, šikmé písmo a~zvýraznění},
  ]

Kurzíva i~šikmá písmena se používají hlavně pro typografické účely
zvýraznění části textu, aby se na něj upozornilo.
Jinými slovy, aby jej {zdůraznily}.

Text můžeme samozřejmě zdůraznit tím, že výslovně povolíme kurzívu nebo
šikmé písmo. \ConTeXt\ však nabízí alternativní příkaz, který je mnohem vhodnější.
užitečnější a~zajímavější a~je určen speciálně pro zvýraznění
fragmentu textu. Jedná se o~příkaz \PlaceMacro{em}\tex{em} ze slova
{\em emphasis}. Na rozdíl od \tex{it} a~\tex{sl}, které jsou čistě
typografické příkazy, \tex{em} je {\em konceptuální} příkaz; pracuje s~ním
jinak, takže je univerzálnější, a~to až do té míry, že \ConTeXt{}
dokumentace doporučuje používat \tex{em} před \tex{it} nebo
\tex{sl}. Při použití těchto dvou posledních příkazů říkáme
\ConTeXt{}u, jakou alternativu písma chceme použít; ale když použijeme příkaz
\tex{em}, říkáme mu, jaký efekt chceme vytvořit, a~necháváme to na
\ConTeXt{}u, aby rozhodl, jak toho dosáhnout.  Obvykle se pro dosažení efektu
zdůraznění nebo zvýraznění něčeho, povolíme kurzívu nebo
šikmé písmo, ale to závisí na kontextu. Pokud tedy použijeme \tex{em} v~příkazu
textu, který je již napsán kurzívou -- nebo je nakloněn --, příkaz bude
zvýrazní opačným způsobem -- v~tomto případě svislým textem.

Proto následující příklad:

\startDoubleExample

\startlines
\starttyping

{\em Jedna z~nejkrásnějších 
orchidejí na světě je 
{\em Thelymitra variegata}. 
neboli Jižní královna ze Sáby.}
\stoptyping
\stoplines


{\em Jedna z~nejkrásnějších 
orchidejí na světě je  {\em
 Thelymitra variegata} neboli Jižní královna ze Sáby}.
 
\stopDoubleExample

Všimněte si, že první \tex{em} umožňuje psaní kurzívou (ve skutečnosti šikmou, ale viz.
níže) a~že druhý \tex{em} ji vypíná a~místo toho vkládá
slova \quotation{Thelymitra variegata} normálním vzpřímeným stylem.

Další výhodou \tex{em} je to, že není alternativou, takž
nezakáže alternativu, kterou jsme měli předtím, a~tak například v~textu
který je tučně, pomocí \tex{em} získáme tučné šikmé písmo, aniž bychom potřebovali
explicitně volat \tex{bs}. Podobně, pokud se příkaz \tex{bf}
objeví v~textu, který je již zvýrazněn, toto zvýraznění
nepřestane.

Ve výchozím nastavení \tex{em} umožňuje spíše šikmou než kurzívu, ale můžeme.
změnit pomocí \tex{setupbodyfontenvironment[default][em=italic]}.

\stopsubsection

\startsubsection
  [
    reference=sec:smallcaps,
    title=Malé kapitálky a~falešne malé kapitálky,
  ]

Malé kapitálky jsou typografickým prostředkem, který je často mnohem lepší než
použití velkých písmen. Malé kapitálky nám dávají tvar
velkého písmene, ale zachovávají stejnou výšku jako malá písmena na
řádku. Proto jsou malé kapitálky stylistickou variantou malých písmen. Malé
kapitálky nahrazují v~určitých kontextech velká písmena a~jsou zvláště
užitečná při psaní římských číslic nebo názvů kapitol. V~akademických textech
je také zvykem používat malá písmena pro psaní jmen citovaných autorů.

Problémem je, že ne všechna písma používají malé kapitálky, a~ta, která je používají.
to ne vždy dělají pro některé ze svých stylů písma. Kromě toho,
protože malé kapitálky jsou alternativou kurzívy, tučného písma nebo šikmého písma,
v~souladu s~obecnými pravidly, která jsme uvedli v~této kapitole, se všechny
tyto typografické vlastnosti nemohly použit současně.

Tyto problémy lze vyřešit použitím {\em falešných malých kapitálek}, které
\ConTeXt\ umožňuje vytvořit pomocí příkazů \tex{cap} a~\tex{Cap};
v~tomto ohledu viz \in{sekce}[sekce:Upper-Lower-Fake].

\stopsubsection

\stopsection

\startsection
  [title=Použití a~konfigurace barev]


\ConTeXt\ poskytuje příkazy pro změnu barvy celého
dokumentu, některých jeho prvků nebo určitých částí textu. 
Poskytuje také příkazy pro nahrání stovek předdefinovaných barev
do paměti a~pro zobrazení jejich složek.

\startsubsection
  [title=Postupy pro barevnou sazbu fragmentů textu]


Většina konfigurovatelných příkazů \ConTeXt{}u umožňuje volbu nazvanou
\MyKey{color}, která nám umožňuje určit barvu, ve které
má být zapsán příkazem ovlivněný text. Tak například pro to
že názvy kapitol se mají psát modře, stačí napsat:

\vbox{\starttyping
  \setuphead
    [chapter]
    [color=blue]
\stoptyping}

Pomocí tohoto postupu můžeme obarvit názvy, nadpisy, poznámky pod čarou, okraje,
poznámky, sloupce a~čáry, tabulky, názvy tabulek nebo obrázků atd. Výhodou
tohoto postupu je to, že konečný výsledek bude konzistentní (všechny
texty, které plní stejnou funkci, budou napsány stejným způsobem.
barvou) a~snadněji se globálně mění. 


Část nebo fragment textu můžeme také přímo obarvit, ačkoli, abychom
se vyhli příliš pestrému použití barev, které nejsou příjemné z~pohledu
typografického hlediska, nebo nekonzistentnímu použití, je obecně vhodné
vyhnout se přímému vybarvování a~používat to, co bychom mohli nazvat {\em
sémantické barvení}, tj. místo psaní např.

\type{\color[red]{Velmi důležitý text}}

definujeme příkaz pro velmi důležitý text, kterému je přiřazena barva. Pro
například

\starttyping
\definehighlight[important][color=red]
\important{Velmi důležitý text}
\stoptyping

\stopsubsection

\startsubsection
  [title=Změna barvy pozadí a~popředí dokumentu]
\PlaceMacro{setupbackgrounds}\PlaceMacro{setupcolors}

Pokud chceme změnit barvu celého dokumentu v~závislosti na tom
zda chceme změnit barvu pozadí nebo barvu dokumentu, nebo zda chceme změnit barvu
popředí (textu), použijeme \tex{setupbackgrounds} nebo
\tex{setupcolors}. Takže například

\starttyping
\setupbackgrounds
  [page]
  [background=color,backgroundcolor=blue]
\stoptyping

Tento příkaz nastaví barvu pozadí stránek na modrou. Jako hodnota
pro \MyKey{backgroundcolor} můžeme použít název některého z~předdefinovaných názvů.
barvy.


Globální změna barvy popředí v~celém dokumentu (od
od místa vložení příkazu) použijte \tex{setupcolors}, kde
volba \MyKey{textcolor} řídí barvu textu. Například:

\type{\setupcolors[textcolor=red]}

uvidíte, že barva textu je červená.

\stopsubsection

\startsubsection
  [title=Příkazy pro obarvení jednotlivých textových\\ fragmentů]
\PlaceMacro{color}\PlaceMacro{colored}

Obecný příkaz pro obarvení malých částí textu je:

\type{\color[ColourName]{Text to zabarvení}}

Pro větší části textu je vhodnější používat

\type{\startcolor[ColourName] ... \stopcolor}

Obě jsou pojmenovány podle nějaké předem definované barvy. Pokud chceme definovat
barvu za běhu, můžeme použít příkaz \tex{colored}. Například:

\startDoubleExample

\starttyping
Tři \colored[r=0.1, g=0.8, b=0.8]
{zabarvené} kočky    
\stoptyping

Tři \colored[r=0.1, g=0.8, b=0.8]{zabarvené} kočky.
  
\stopDoubleExample

\stopsubsection

\startsubsection
  [
    reference=sec:predefined-colours,
    title=Předdefinované barvy,
  ]
  % This information comes from the reference manual. But I
  % suspect there are considerably more predefined colours. For
  % example the "maincolor" used in this document is based on
  % the orange not defined in the list of predefined colours.

\ConTeXt\ načte nejběžnější předdefinované barvy uvedené v~seznamu
  \in{tabulka}[tbl:predefined colors].\footnote{Tento seznam lze nalézt v~referenční příručce a~na wiki \ConTeXt{}u, ale jsem si docela jistý, že je to neúplný seznam, protože v~tomto dokumentu, aniž bychom načetli nějakou další barvu, používáme například \quotation{oranžová} - která není v~\in{tabulka}[tbl:predefined colors]-- pro názvy sekcí.}

{\switchtobodyfont[small]
\placetable
  [here]
  [tbl:predefined colours]
  {\ConTeXt's predefined colours}
{\starttabulate[|l|l|l|l|]
\HL
\NC{\bf Jméno}
\NC{\bf Světlý odstín}
\NC{\bf Střední odstín}
\NC{\bf Tmavý odstín}
\NR
\HL
\NC black
\NR
\NC white
\NR
\NC gray
\NC lightgray
\NC middlegray
\NC darkgray
\NR
\NC red
\NC lightred
\NC middlered
\NC darkred
\NR
\NC green
\NC lightgreen
\NC middlegreen
\NC darkgreen
\NR
\NC blue
\NC lightblue
\NC middleblue
\NC darkblue
\NR
\NC cyan
\NC\NC middlecyan
\NC darkcyan
\NR
\NC magenta
\NC\NC middlemagenta
\NC darmagenta
\NR
\NC yellow
\NC\NC middleyellow
\NC darkyellow
\NR
\HL
\stoptabulate
}}


Existují i~další kolekce barev, které se ve výchozím nastavení nenačítají, ale které mohu
načíst příkazem

\PlaceMacro{usecolors}\type{\usecolors[CollectionName]}

kde CollectionName může být 

\startitemize[packed]
\item \MyKey{crayola}, 235 barev imitujících odstíny fixů.
\item \MyKey{dem}, 91 barev.
\item \MyKey{ema}, 540 definice barev založené na barvách používaných v~systému Emacs..
\item \MyKey{rainbow}, 91 barvy pro použití v~matematických vzorcích.
\item \MyKey{ral}, 213 definice barev ze {\em Deutsches
  Institut für Gütesicherung und Kennzeichnungí} (Německý institut pro zajišťování kvality a~označování).
\item \MyKey{rgb}, 223 barev.
\item \MyKey{solarized}, 16 barev na základě solárního schématu.
\item \MyKey{svg}, 147 barev.
\item \MyKey{x11}, 450 standardních barev pro X11.
\item \MyKey{xwi}, 124 barev.
\stopitemize

\startSmallPrint

  Soubory s~definicemi barev jsou obsaženy
v~\MyKey{context/base/mkiv} adresář distribuce a~jeho název
  odpovídá schématu \MyKey{colo-imp-NOMBRE.mkiv}. Informace 
  o~různých kolekcích předdefinovaných barev, které jsem právě uvedl
  jsou založeny na mé konkrétní distribuci. Konkrétní kolekce, resp.
  počet barev v~nich definovaných, se mohou změnit v~budoucích
  verzích.

\stopSmallPrint

Abychom zjistili, jaké barvy obsahuje každá z~těchto kolekcí, můžeme použít příkaz
\PlaceMacro{showcolor}\tex{showcolor[CollectionName]} příkaz popsaný v~kapitole 2.
v~následujícím textu. Abychom mohli některé z~těchto barev použít, musíme je nejprve načíst
do paměti pomocí příkazu (\tex{usecolors[CollectionName]}) a~poté
musíme příkazem \tex{color} nebo
\PlaceMacro{startcolor}\tex{startcolor} přikázat název barvy.
Například následující sekvence: \page[preference]

\starttyping
\usecolors[xwi]
\color[darkgoldenrod]{Tweedledum a~Tweedledee}
\stoptyping
\page[no]

napíše
\usecolors[xwi]\color[darkgoldenrod]{Tweedledum a~Tweedledee}
\page[preference]

\stopsubsection

\startsubsection
  [title=Zobrazení dostupných barev]
\PlaceMacro{showcolor}

Příkaz \tex{showcolor} zobrazí seznam barev, ve kterých můžete vidět
vzhled barvy, její vzhled, když je barva použita
v~stupnici šedi, červené, zelené a~modré
a~název, pod kterým ji \ConTeXt\ zná. Používá se bez argumentu
\tex{showcolor} zobrazí barvy použité v~aktuálním dokumentu. Ale
jako argument můžeme uvést libovolnou z~předdefinovaných kolekcí barev
barev, které byly popsány v~\in{sekci}[sec:predefined-colours],
a~takže například \tex{showcolor[solarized]} nám zobrazí 16 barev.
solarizovaných v~této kolekci:

\showcolor[solarized]

Pokud chceme zobrazit složky rgb určité barvy, můžeme použít příkaz
\PlaceMacro{showcolorcomponents}\tex{showcolorcomponents[ColourName]}.
To je užitečné, pokud se snažíme definovat určitou barvu, abychom viděli
složení nějaké barvy, která je jí blízká. Například,
\tex{showcolorcomponents[darkgoldenrod]} nám ukáže:

\startframedtext[width=\textwidth]\switchtobodyfont[small]
\showcolorcomponents[darkgoldenrod]
\stopframedtext

\stopsubsection

\startsubsection
  [title=Definování vlastních barev]
\PlaceMacro{definecolor}


\tex{definecolor} nám umožňuje buď klonovat existující barvu, nebo definovat
novou barvu. Klonování existující barvy je stejně jednoduché jako vytvoření 
jejího alternativního názvu. K~tomu je třeba napsat:

\type{\definecolor[Nová barva][Stará barva]}

Tím se zajistí, že "{\em Nová barva}" bude mít přesně stejnou barvu jako
"{\em Stará barva}".

Hlavní použití \tex{definecolor} však spočívá ve vytváření nových barev. Chcete-li provést
je třeba příkaz použít následujícím způsobem:

\type{\definecolor[ColourName][Definition]}

kde {\em Definition} může být provádena použitím až šesti různých
barevných schémat:

\startitemize


\item {\bf RGB barvy}: Definice barev RGB je jedním
z~nejrozšířenější; vychází z~myšlenky, že barvy je možné
znázornit smícháním tří základních barev sčítáním:
červené (\quote{r} pro {\em červená}), zelené (\quote{g} pro {\em zelená})
a~modré (\quote{b} pro {\em modrá}). Každá z~těchto složek je označena
jako desetinné číslo mezi 0 a~1.

  \type{\definecolor[lime 1][r=0.75, g=1, b=0]}: \definecolor[lime 1]
       [r=0.75, g=1, b=0] \color[lime 1]{Text in “lime 1”}.

\item {\bf Hex barvy}: Tento způsob reprezentace barev je také
založen na schématu RGB, ale červená, zelená a~modrá složka jsou
označeny jako tříbajtové hexadecimální číslo, ve kterém první bajt
představuje hodnotu červené barvy, druhý hodnotu zelené barvy a~třetí 
hodnotu modré barvy. Například:

  \type{\definecolor[lime 2][x=BFFF00]}: \definecolor[lime 2][x=BFFF00]
  \color[lime 2]{Text in “lime 2”}.


\item {\bf Barvy CMYK}: Tento model generování barev je to, co
se nazývá \quotation{subtraktivní model} a~je založen na směsi
pigmentů těchto barev: azurová (\quote{c}), purpurová
(\quote{m}), žlutá (\quote{y}, z~{\em yellow}) a~černá (\quote{k},
z~{\em key}). Každá z~těchto složek je označena jako desetinné číslo.
mezi 0 a~1:

  \type{\definecolor[lime 3][c=0.25, m=0, y=1, k=0]}:
  \definecolor[lime 3][c=0.25, m=0, y=1, k=0] \color[lime 3]{Text in
   “lime 3”}.

\item {\bf HSL/HSV}: Tento barevný model je založen na měření odstínu.
(\quote{h}, z~{\em hue}), sytosti (\quote{s}) a~luminiscence
(\quote{l} nebo někdy \quote{v}, z~{\em value}). Odstín odpovídá
číslu v~rozmezí 0 až 360; sytost a~luminiscence musí odpovídat číslu
desetinným číslem mezi 0 a~1. Například:

  \type{\definecolor[lime 4][h=75, s=1, v=1]}: \definecolor[lime
    4][h=75, s=1, v=1] \color[lime 4]{Text in “lime 4”}


\item {\bf HWB barvy}: Model HWB je navrhovaným standardem pro CSS4.
který měří odstín (\quote{h}, z~{\em hue}) a~úroveň
bílé (\quote{w}, z~{\em whiteness}) a~černé (\quote{b}, z~{\em
blackness}). Odstín odpovídá číslu v~rozmezí 0 až 360, kdežto
bělost a~čerň jsou reprezentovány desetinným číslem v~rozmezí 0 až 360.
a~1.

  \type{\definecolor[Azure][h=75, w=0.2, b=0.7]}
  \definecolor[Azure][h=75, w=0.2, b=0.7]\color[Azure]{Text in
   “Azure”}.

  % \type{\definecolor[lima 5][h=75, w=0, b=0]}
  % \definecolor[lima 5][h=75, w=0.2, b=0.7]\color[lima 5]{Texto en
  %   color “lima 5”}.

\item {\bf Stupnice šedé}: vychází z~komponenty nazvané (\quote{s}, z~{\em scale}), která měří množství šedé. Musí to být číslo mezi 0 a~1. Například:

  \type{\definecolor[light grey][s=0.65]}: \definecolor[light grey][s=0.65] \color[light grey]{Text in “light grey”}.
  
\stopitemize

It is also possible to define a~new colour from another colour. For
example, the colour in which titles are written in this introduction is
defined as

Je také možné definovat novou barvu z~jiné barvy. 
Například barva, ve které jsou v~tomto úvodu napsány nadpisy, je
definována jako

\type{\definecolor[maincolour][0.6(orange)]}

\stopsubsection

\stopsection

\stopchapter

\stopcomponent


%%% Local Variables:
%%% mode: ConTeXt
%%% mode: auto-fill
%%% coding: utf-8-unix
%%% TeX-master: "../introCTX.mkiv"
%%% End:
%%% vim:set filetype=context tw=72 : %%%
